\documentclass{article}

\def\ParSkip{} 
% Packages
\usepackage{amssymb,amsmath,amsthm,bbm}
\usepackage{verbatim,float,url,dsfont}
\usepackage{graphicx,subfigure,psfrag}
\usepackage{algorithm,algorithmic}
\usepackage{mathtools,enumitem}
\usepackage{multirow}
\usepackage{ragged2e}
\usepackage{xr-hyper}

\usepackage[colorlinks=true,citecolor=blue,urlcolor=blue,linkcolor=blue]{hyperref}
\usepackage[margin=1in]{geometry}
\usepackage[round]{natbib}

\usepackage[utf8]{inputenc} % allow utf-8 input
\usepackage[T1]{fontenc}    % use 8-bit T1 fonts
\usepackage{booktabs}       % professional-quality tables
\usepackage{nicefrac}         % compact symbols for 1/2, etc.
\usepackage{microtype}      % microtypography

\ifdefined\TimesFont 
\usepackage{times} % use times font
\fi

\ifdefined\ParSkip 
\usepackage{parskip} % use par skip
\fi

% Theorems and such
\newtheorem{theorem}{Theorem}
\newtheorem{lemma}{Lemma}
\newtheorem{corollary}{Corollary}
\newtheorem{proposition}{Proposition}
\theoremstyle{definition}
\newtheorem{remark}{Remark}
\newtheorem{definition}{Definition}

% Assumption
\newtheorem*{assumption*}{\assumptionnumber}
\providecommand{\assumptionnumber}{}
\makeatletter
\newenvironment{assumption}[2]{
  \renewcommand{\assumptionnumber}{Assumption #1#2}
  \begin{assumption*}
  \protected@edef\@currentlabel{#1#2}}
{\end{assumption*}}
\makeatother

% Widebar
\makeatletter
\newcommand*\rel@kern[1]{\kern#1\dimexpr\macc@kerna}
\newcommand*\widebar[1]{%
  \begingroup
  \def\mathaccent##1##2{%
    \rel@kern{0.8}%
    \overline{\rel@kern{-0.8}\macc@nucleus\rel@kern{0.2}}%
    \rel@kern{-0.2}%
  }%
  \macc@depth\@ne
  \let\math@bgroup\@empty \let\math@egroup\macc@set@skewchar
  \mathsurround\z@ \frozen@everymath{\mathgroup\macc@group\relax}%
  \macc@set@skewchar\relax
  \let\mathaccentV\macc@nested@a
  \macc@nested@a\relax111{#1}%
  \endgroup
}
\makeatother

% Min and max
\newcommand{\argmin}{\mathop{\mathrm{argmin}}}
\newcommand{\argmax}{\mathop{\mathrm{argmax}}}
\newcommand{\minimize}{\mathop{\mathrm{minimize}}}
\newcommand{\maximize}{\mathop{\mathrm{maximize}}}
\newcommand{\st}{\mathop{\mathrm{subject\,\,to}}}

% Shortcuts
\def\R{\mathbb{R}}
\def\C{\mathbb{C}}
\def\Z{\mathbb{Z}}
\def\N{\mathbb{N}}
\def\E{\mathbb{E}}
\def\P{\mathbb{P}}
\def\T{\mathsf{T}}
\def\Cov{\mathrm{Cov}}
\def\Var{\mathrm{Var}}
\def\indep{\perp\!\!\!\perp}
\def\th{^{\text{th}}}
\def\tr{\mathrm{tr}}
\def\df{\mathrm{df}}
\def\dim{\mathrm{dim}}
\def\col{\mathrm{col}}
\def\row{\mathrm{row}}
\def\nul{\mathrm{null}}
\def\rank{\mathrm{rank}}
\def\nuli{\mathrm{nullity}}
\def\spa{\mathrm{span}}
\def\sign{\mathrm{sign}}
\def\supp{\mathrm{supp}}
\def\diag{\mathrm{diag}}
\def\aff{\mathrm{aff}}
\def\conv{\mathrm{conv}}
\def\dom{\mathrm{dom}}
\def\hy{\hat{y}}
\def\hf{\hat{f}}
\def\hmu{\hat{\mu}}
\def\halpha{\hat{\alpha}}
\def\hbeta{\hat{\beta}}
\def\htheta{\hat{\theta}}
\def\cA{\mathcal{A}}
\def\cB{\mathcal{B}}
\def\cD{\mathcal{D}}
\def\cE{\mathcal{E}}
\def\cF{\mathcal{F}}
\def\cG{\mathcal{G}}
\def\cK{\mathcal{K}}
\def\cH{\mathcal{H}}
\def\cI{\mathcal{I}}
\def\cL{\mathcal{L}}
\def\cM{\mathcal{M}}
\def\cN{\mathcal{N}}
\def\cP{\mathcal{P}}
\def\cS{\mathcal{S}}
\def\cT{\mathcal{T}}
\def\cW{\mathcal{W}}
\def\cX{\mathcal{X}}
\def\cY{\mathcal{Y}}
\def\cZ{\mathcal{Z}}


\title{Nonparametric Regression: Splines and RKHS Methods \\ \smallskip
\large Advanced Topics in Statistical Learning, Spring 2023 \\ \smallskip
Ryan Tibshirani}
\author{}
\date{}

\begin{document}
\maketitle
\RaggedRight
\vspace{-50pt}

Note: we're following the context, problem setup, notation, etc. from the last
lecture on nonparametric regression.

\section{Regression splines}

Regression splines and smoothing splines are motivated from a different
perspective than kernels and local polynomials; in the latter case, we started
off with a special kind of local averaging, and moved our way up to a
higher-order local models. With regression splines and smoothing splines, we
build up our estimate globally, from a set of select basis functions. 

(We note that, at a broader level, the latter is often called the
\emph{synthesis} framework for modeling, where we build up our estimate from a
set of atoms---here being basis functions.) 

\subsection{Splines}

These basis functions, as you might guess, are \emph{splines}. Let's assume that
$d=1$. We'll stick to the univariate case for a little while, because splines
are complex and interesting enough in dimension $d=1$. A spline $f$ of degree
$k$ with knots at $t_1 < \dots < t_r$ is a piecewise polynomial of degree $k$  
that is continuous and has continuous derivatives of orders $1,\dots,k-1$ at its
knots. To be clear: 
\begin{itemize}
\item $f$ is a polynomial of degree $k$ on each of $(-\infty,t_1], [t_1,t_2],
  \dots, [t_r,\infty)$; and
\item $D^\ell f$ is continuous at each of $t_1,\dots,t_r$, for all
  $\ell=0,\dots,k-1$. 
\end{itemize}

Splines have some special (some might say amazing) properties, and we will only
really scratch the surface here. They have been a topic of interest among
mathematicians and statisticians for a long time. Informally, a spline is a lot
smoother than a piecewise polynomial, and so modeling with splines can serve as
a way of reducing the variance of an estimator. See Figure \ref{fig:splines} for
an illustration.  

\begin{figure}[tb]
\centering
\includegraphics[width=0.8\textwidth]{splines.pdf}
\caption{\it Illustration of the effects of enforcing continuity at the knots
  for a cubic piecewise polynomial, and for various orders of the derivative,
  Credit: Chapter 5 of \citet{hastie2009elements}.}    
\label{fig:splines}
\end{figure}

A bit of statistical folklore: it is said that a cubic spline is so smooth, that
one cannot detect the locations of its knots by eye! 

\subsection{Spline bases}

How can we parametrize the space of $k\th$ degree splines with knots at 
$t_1,\dots,t_r$?  The most natural way is to use the \emph{truncated power
  basis}, $g_1, \dots, g_{r+k+1}$, defined as   
\begin{equation}
\begin{gathered}
\label{eq:tpb}
g_j(x) = \frac{1}{(j-1)!} x^j, \quad j=1,\dots,k+1, \\
g_{j+k+1}(x) = \frac{1}{k!} (x-t_j)^k_+, \quad j=1,\dots,r,
\end{gathered}
\end{equation}
Here $x_+=\max\{x,0\}$ denotes the positive part of $x$. From this we can see
that the linear space of $k\th$ degree splines with knots at $t_1,\dots,t_r$ has 
dimension $r+k+1$.

While \eqref{eq:tpb} gives us a simple and natural basis, calculations involving
truncated power bases functions can be troublesome because the ensuing basis 
matrix, $G \in \R^{n \times (r+k+1)}$ with entries
\[
G_{ij} = g_j(x_i)
\]
is generally very poorly conditioned.

A much better computational choice, both for speed and numerical stability, is 
called the \emph{B-spline basis}. This was a major development in spline theory 
and is now much the standard in software. The key idea is that B-splines have
local support: each B-spline basis function for a $k\th$ degree spline space is
supported on $k+2$ knots. Therefore the corresponding basis matrix is banded. It
also tends to be much better conditioned.  

Defining B-splines is certainly requires more nuance than the truncated power
basis, and the precise form of a B-spline is unimportant for the rest of this
lecture. For completeness, you can find the details behind their construction in
Appendix \ref{app:bs}. (You can also jump to Figure \ref{fig:bs} for a visual.) 

\subsection{Regress away!}

A first idea: let's just perform regression on a spline basis. In other words,
we use as our working model a $k\th$ degree spline with knots at some pre-fixed 
locations $t_1,\dots,t_r$. This means expressing $f$ as  
\[
f = \sum_{j=1}^{r+k+1} \beta_j g_j
\]
where $\beta_1,\dots,\beta_{r+k+1} \in \R$ are coefficients and
$g_1,\dots,g_{r+k+1}$ is a basis for the space of $k\th$ degree splines over the
knots $t_1,\dots,t_r$; for example, the truncated power basis or B-spline basis.    

Letting $Y=(y_1,\dots,y_n) \in \R^n$ denote the response vector, and defining
the basis matrix $G \in \R^{n \times (r+k+1)}$ as before, to have entries
\[
G_{ij} = g_j(x_i),
\]
we then just use least squares regression to determine estimate the
coefficients, defining \smash{$\hbeta = (\hbeta_1,\dots,\hbeta_{r+k+1})$} to
solve 
\[
\minimize_\beta \; \|Y - G\beta\|_2^2.
\]
This yields the \emph{regression spline} estimator, which makes predictions
according to   
\[
\hf(x) = \sum_{j=1}^{r+k+1} \hbeta_j g_j(x).
\]
Of course we know that \smash{$\hbeta = (G^\T G)^{-1} G^\T Y$}, so we can write
this as 
\[
\hf(x) = (g_1(x),\dots,g_{r+k+1}(x))^\T (G^\T G)^{-1} G^\T Y = w(x)^\T Y,  
\]
which reveals that regression splines are linear smoothers.

This is a classic method, and can work well provided that we choose ``good''
knots $t_1,\dots,t_r$; but in general choosing knots is a tricky business. There
is a large literature on knot selection for regression splines via greedy
methods like recursive partitioning. In practice, smoothing splines seem to be
more popular, which we cover next.

\section{Smoothing splines}

Before delving into smoothing splines, we need to introduce a variant on the
usual spline definition given above. To motivate it: a problem with spline
estimates is that they can have somewhat erractive behavior---translating into
high variance---at the boundaries of the input domain. (Recall that this is the
opposite problem to that with kernel smoothing, which had poor bias at the
boundaries.) This only gets worse as the polynomial order $k$ gets larger.   

\subsection{Natural splines}

One way to remedy this problem is to force the piecewise polynomial function to
have a lower degree to the left of the leftmost knot, and to the right of the
rightmost knot---this is exactly what \emph{natural splines} do. A natural
spline $f$ of degree $k$ with knots at $t_1 < \dots < t_r$ is a piecewise
polynomial of degree $k$ such that:
\begin{itemize}
\item $f$ is a polynomial of degree $k$ on each of $[t_1,t_2], \dots,
  [t_{r-1},t_r]$; 
\item $f$ is a polynomial of degree $(k-1)/2$ on $(-\infty,t_1]$ and
  $[t_r,\infty)$; and 
\item $D^\ell f$ is continuous at each of $t_1,\dots,t_r$, for all
  $\ell=0,\dots,k-1$. 
\end{itemize}
It is implicit here that natural splines are only defined for an odd degree $k$
(linear, cubic, etc.) The choice $k=3$ yields a natural cubic spline, by far the
most common case: this is just a cubic spline that reduces to linear beyond the 
leftmost and rightmost knots.     

What is the dimension of the span of $k\th$ degree natural splines with knots at
$t_1,\dots,t_r$? Recall for splines, this was $r+k+1$ (just count the number of
truncated power basis functions). For natural splines, we can compute this
dimension by counting as follows:  
\[
\underbrace{\vphantom{\Big(\Big)} (k+1)\cdot(r-1)}_{a} \;+\; 
\underbrace{\Big(\frac{(k-1)}{2}+1\Big) \cdot 2}_{b} \;-\;
\underbrace{\vphantom{\Big(\Big)} k \cdot r}_{c} \;=\; r. 
\]
In the above: 
\begin{itemize}
\item $a$ is the number of free parameters in the interior intervals $[t_1,t_2], 
  \dots, [t_{r-1},t_r]$;
\item $b$ is the number of free parameters in the exterior intervals $(-\infty,t_1],
  [t_r,\infty)$; and 
\item $c$ is the number of constraints at the knots $t_1,\dots,t_r$.  
\end{itemize}
The fact that the total dimension is $r$ is amazing; this is independent of $k$!  

We note that there are simple modifications the truncated power basis that gives
rise to a basis for natural splines, and similarly a modification of the
B-spline basis for natural splines. And again, B-splines are the preferred
parametrization for computational speed and stability. 

\subsection{Smooth away!}

Smoothing splines, at the end of the day, are given by an $\ell_2$-regularized
regression over a natural spline basis that places knots at all input points 
$x_1,\dots,x_n$. They circumvent the problem of knot selection because they
just use all inputs as knots, and they make this possible---starting with a 
saturated model and producing meaningful function estimates---by using
regularization to shrink the coefficients in the basis expansion.

Interestingly, we can motivate and define a smoothing spline directly from a
functional minimization perspective. With input points $x_1,\dots,x_n$ lying in 
an interval $[a,b]$, the \emph{smoothing spline} estimator \smash{$\hf$}, of a
given odd integer order $k \geq 1$, is defined to solve   
\begin{equation}
\label{eq:ss}
\minimize_f \; \sum_{i=1}^n (y_i - f(x_i))^2 + \lambda \int_a^b [D^m f(x)]^2 \,
dx, \quad \text{where $m=(k+1)/2$}.
\end{equation}
This is an infinite-dimensional optimization problem over all functions $f$ for
the which the criterion is well-defined and finite (this is called the $L^2$
Sobolev space of order $m$ on $[a,b]$, which we denote by $W^{m,2}([a,b])$, and
will study a bit later). The criterion in \eqref{eq:ss} trades off the squared
error of $f$ over $(x_i,y_i)$, $i=1,\dots,n$, with a penalty term that is large
when the $m\th$ derivative of $f$ is wiggly. The tuning parameter $\lambda \geq 
0$ governs the tradeoff between these two terms. 

By far the most commonly considered case is $k=3$, called the cubic smoothing
spline, and defined via
\begin{equation}
\label{eq:ss3}
\minimize_f \; \sum_{i=1}^n (y_i - f(x_i))^2 + \lambda \int_a^b [f''(x)]^2 \, dx.
\end{equation}

\subsection{Representer theorem}

\def\tf{\tilde{f}}

Remarkably, it so happens that the minimizer in the general smoothing spline
problem \eqref{eq:ss} is unique, and it is a natural $k\th$ degree spline with 
knots at the input points $x_1,\dots,x_n$! This is known as a \emph{representer
  theorem} for \eqref{eq:ss} (we will see more such results later). 

Proof: we'll stick to the cubic case, $k=3$, and follow Chapter 2.2 of
\citet{green1993nonparametric} for a nice direct proof. The result for general
$k$ will follow from RKHS theory.  

The key result can be stated as follows: if \smash{$\tf$} is any function on 
$[a,b]$ such that the penalty is well-defined (it has two derivatives, its
second derivative is square integrable), and $x_1,\dots,x_n \in [a,b]$ are
arbitrary, then there exists a natural cubic spline $f$ with knots at
$x_1,\dots,x_n$ such that:
\begin{itemize}
\item $f(x_i) = \tf(x_i)$, $i=1,\dots,n$; and 
\item $\int_a^b [f''(x)]^2 \, dx \leq \int_a^b [\tf''(x)]^2 \, dx$.
\end{itemize}
Note that this would in fact prove that we can restrict our attention in problem 
\eqref{eq:ss3} to natural splines with knots at $x_1,\dots,x_n$.

To prove the key result, we start with the fact that the cubic natural spline
space with knots at $x_1,\dots,x_n$ is $n$-dimensional, so given any $n$ points  
\smash{$z_i = \tf(x_i)$}, $i=1,\dots,n$, we can always find a natural spline $f$ 
with knots at $x_1,\dots,x_n$ that satisfies $f(x_i) = z_i$, $i=1,\dots,n$. Now 
define \smash{$h = \tf - f$}. Consider
\begin{align*}
\int_a^b f''(x) h''(x) \, dx 
&= f''(x) h'(x) \Big|_a^b - \int_a^b f'''(x) h'(x) \, dx \\
&= -\int_{x_1}^{x_n} f'''(x) h'(x) \, dx \\ 
&= -\sum_{i=1}^{n-1} f'''(x) h(x) \Big|_{x_i}^{x_{i+1}} +  
\int_{x_1}^{x_n} D^4 f(x) h'(x) \, dx \\
&= -\sum_{i=1}^{n-1} f'''(x_i^+) (h(x_{i+1}) - h(x_i)).
\end{align*}
In the first line we used integration by parts; in the second line we used the
fact that $f''(a) = f''(b) = 0$ and $f'''(x)=0$ for $x \leq x_1$ and $x \geq
x_n$, since $f$ is a natural spline; in the third line we used integration by
parts again; in the fourth we used the fact that $f'''$ is constant on each open
interval $(x_i,x_{i+1})$, and that $D^4 f=0$, again because $f$ is a natural
spline. Since each $h(x_i)=0$, we conclude from the last display that
\[
\int_a^b f''(x) h''(x) \, dx = 0.
\]
From this, it follows that
\[
\int_a^b [\tf''(x)]^2 \, dx 
= \int_a^b \big[ f''(x) + h''(x) \big]^2 \, dx 
= \int_a^b [f''(x)]^2 \, dx + \int_a^b [h''(x)]^2 \, dx
\]
since the cross term is zero, and therefore 
\[
\int_a^b [f''(x)]^2 \, dx \leq \int_a^b [\tf''(x)]^2 \, dx,
\]
with equality if and only if $h''(x)=0$ for all $x \in [a,b]$. Note that $h''=0$
implies that $h$ must be linear, and since we already know that $h(x_i)=0$ for
all $i=1,\dots,n$, this is equivalent to $h=0$. In other words, the last display
holds strictly except when \smash{$\tf=f$}, so the solution in \eqref{eq:ss3} is
uniquely a natural spline with knots at the inputs.

\subsection{Finite-dimensional form}

From the representer result, we can choose a basis $\eta_1,\dots,\eta_n$ for the
set of $k\th$ degree natural splines with knots at $x_1,\dots,x_n$, and
reparametrize the problem \eqref{eq:ss} as  
\begin{equation}
\label{eq:ss_finite_dim}
\minimize_\beta \; \sum_{i=1}^n \bigg( y_i - \sum_{j=1}^n \beta_j \eta_j(x_i)
\bigg)^2 + \lambda \int_a^b \bigg( \sum_{j=1}^n \beta_j D^m \eta_j(x) \bigg)^2
\, dx.  
\end{equation}
This is a finite-dimensional problem, and after we solve for the coefficients
\smash{$\hbeta \in \R^n$}, the smoothing spline estimator is simply given by  
\[
\hf(x)=\sum_{j=1}^n \hbeta_j \eta_j(x).
\]
Defining the basis matrix $N \in \R^{n \times n}$ and penalty matrix $\Omega \in
\R^{n \times n}$ to have entries
\[
N_{ij} = \eta_j(x_i), \quad 
\Omega_{ij} = \int_a^b D^m \eta_i(x) D^m \eta_j(x) \, dx,
\]
the problem in \eqref{eq:ss_finite_dim} can be written more succintly as
\[
\minimize_\beta \; \|Y - N\beta\|_2^2 + \lambda \beta \Omega \beta,  
\]
showing the smoothing spline problem to be a type of generalized ridge
regression problem. Its solution has the explicit form \smash{$\hbeta = (N^\T N
  + \lambda \Omega)^{-1} N^T Y$}, and thus, we can write the smoothing spline as   
\[
\hf(x) = (\eta_1(x),\dots,\eta_n(x))^\T (N^\T N + \lambda \Omega)^{-1} N^\T Y =
w(x)^\T Y,   
\]
which means, once again, smoothing splines are linear smoothers.

A remark on computation: the coefficients \smash{$\hbeta = (N^\T N + \lambda 
  \Omega)^{-1} N^\T Y$} can be computed in $O(n)$ operations. For this, we form  
$N$ using the B-spline basis (for natural splines), since then the matrix $N^\T
N + \Omega I$ will be banded. In fact, more specialized computations are
possible by taking advantage of more precise structure (beyond bandedness)
afforded by splines. Altogether, in practice, smoothing spline computations are 
extremely fast. 

\subsection{Reinsch form}

The vector of fitted values \smash{$\hat{Y} = (\hf(x_1),\dots,\hf(x_n)) \in
  \R^n$}  from the smoothing spline can be written as
\[
\hat{Y} = N(N^\T N + \lambda \Omega)^{-1} N^\T Y.
\]
It is informative to rewrite this in what is called Reinsch form,
\begin{align*}
\hat{Y} 
&= N\Big(N^\T (I + \lambda N^{-\T} \Omega N^{-1}) N \Big)^{-1} N^\T Y \\
&= (I+\lambda Q)^{-1} Y,
\end{align*}
where $Q = N^{-\T} \Omega N^{-1}$. Note that $Q$ does not depend on $\lambda$.
Thus if we compute an eigendecomposition $Q = U \Sigma U^\T$, then the
eigendecomposition of the smoother matrix \smash{$S = (I+\lambda Q)^{-1}$} is   
\[
S = \sum_{j=1}^n \frac{1}{1+\lambda \sigma_j} u_ju_j^\T,
\]
where $\Sigma=\mathrm{diag}(\sigma_1,\dots,\sigma_n)$. Therefore, once more, the
smoothing spline fitted values can be expressed as  
\[
\hat{Y} = \sum_{j=1}^n \frac{u_j^\T Y}{1+\lambda \sigma_j} u_j.
\]
The interpretation we may gather is as follows: smoothing splines perform a 
regression on the orthonormal set $u_1,\dots,u_n \in \R^n$, but they shrink
the coefficients, with more shrinkage applied to an eigenvector $u_j$ that
corresponds to a large eigenvalue $\sigma_j$.  

What exactly are these vectors $u_1,\dots,u_n$? These are referred to as the 
\emph{Demmler-Reinsch basis}, and some nice properties of theirs can be 
worked out analytically \citep{demmler1975oscillation}. In a nutshell: the
eigenvectors $u_j$ that correspond to smaller eigenvalues $\sigma_j$ are
smoother, and so with smoothing splines, we shrink less in their direction. Or
to put it differently, by increasing $\lambda$ in the smoothing spline
estimator, we are tuning out the more wiggly components. See Figure
\ref{fig:reinsch}. 

\begin{figure}[tbp]
\centering
\includegraphics[width=0.495\textwidth]{reinsch_basis.pdf}
\includegraphics[width=0.495\textwidth]{reinsch_weight.pdf}
\caption{\it Insights from the Reinsch form of the cubic smoothing spline, for
  an example with $n=50$ input points. Left: the bottom 8 eigenvectors of $Q$,
  drawn using a color gradient where purple means the smallest eigenvalue, and
  light blue the largest eigenvalue. We see that smoother eigenvectors
  correspond to lower eigenvalues. Right: the shrinkage weights
  $w_j=1/(1+\lambda \sigma_j)$, $j=1,\dots,n$ for 8 values of $\lambda$, again
  drawn using a color gradient where purple means the largest $\lambda$, and
  light blue the smallest $\lambda$. For larger $\lambda$, the weights decay
  faster, so the smoothing spline estimator more strongly filters out wiggly
  components.}         
\label{fig:reinsch}
\end{figure}

\subsection{Equivalent kernel}

Recall that we can write a smoothing spline prediction as \smash{$\hf(x) = 
  w(x)^\T Y$}, for a weight function  $w(x) = (w_1(x), \dots, w_n(x)) \in \R^n$.
We know the analytic form of this weight function, but how about its qualitative
behavior? To be more precise, if we denote each component function by $w_i(x) =
w(x, x_i)$ to emphasize that this weight gets attributed to $(x_i, y_i)$ in the
weighted sum \smash{$w(x)^\T Y = \sum_{i=1}^n w(x, x_i) y_i$} which gives us the
smoothing spline prediction at $x$, then we can ask the following questions:         
\begin{itemize}
\item What shape does $z \mapsto w(x, z)$ have? Is it kernel-like?
\item Does this shape change as we vary the test point $x$? 
\end{itemize}

It's easy to just read off the answers to these questions from the rows of the
smoother matrix $S = N(N^\T N + \lambda \Omega)^{-1} N^\T$, since in our
expanded notation, its elements are
\[
S_{ij} = w_j(x_i) = w(x_i, x_j).
\]
Now, something very interesting happens when we plot a few rows of $S$. For
evenly-spaced inputs, they look like the translations of the same kernel: see
the left panel of Figure \ref{fig:ss_kernel}. Even more interestingly, for
unevenly-spaced inputs, the rows still have a kernel shape but now the bandwidth
appears to adapt to the density of the input points: lower density, larger
bandwidth. See the right panel of Figure \ref{fig:ss_kernel}.

\begin{figure}[tb]
\centering
\includegraphics[width=0.495\textwidth]{ss_kernel1.pdf}
\includegraphics[width=0.495\textwidth]{ss_kernel2.pdf}
\caption{\it Three sample rows of the cubic smoothing spline operator $S$
  defined over $n=500$. Left: evenly-spaced inputs on a grid. The weights look
  like they are given by translations of the same kernel. Right: inputs drawn
  i.i.d.\ from $N(0.5, 0.01)$. The weights still look like kernels, but the
  bandwidth is now larger in low-density regions of the input domain.}     
\label{fig:ss_kernel}
\end{figure}

What we are seeing is an empirical validation of a beautiful asymptotic result
by \citet{silverman1984spline}, who proved that the cubic smoothing spline
estimator is asymptotically equivalent to a kernel regression estimator, with an 
unusual choice of kernel. Specifically, under suitable regularity conditions, we
have the large $n$ approximation,\footnote{Silverman actually shows that
  \smash{$w(x, z) \approx \frac{1}{h(z)} \frac{1}{p(z)} K\left( \frac{x-z}{h(z)}
    \right)$}, but the version we are stating has a more natural interpretation
  with respect to Figure \ref{fig:ss_kernel}, and is a consequence of symmetry
  of the smoother matrix $S$.} 
\[
w(x, z) \approx \frac{1}{h(x)} \frac{1}{p(x)} K\bigg( \frac{x-z}{h(x)} \bigg), 
\]
where $K$ is the ``Silverman kernel'': 
\[
K(t) = \frac{1}{2} \exp(-|t|/\sqrt{2}) \sin(|t|/\sqrt{2} + \pi/4), 
\]
and we have the local bandwidth: 
\[
h(x) = \bigg[ \frac{\lambda}{p(x)} \bigg]^{1/4},
\]
where $p(x)$ is the density of the input distribution at $x$. That is, the
bandwidth adapts to the local distribution of inputs. 

The Silverman kernel is ``kind of'' a higher-order kernel. It satisfies 
\[
\int K(t) \, dt = 1, \quad
\int t^j K(t) \, dt = 0, \; j=1,\dots,3, 
\quad \text{but} \quad 
\int t^4 K(t) \, dt = -24.
\]
So it lies outside the scope of usual kernel analysis. There is a lot more work
building off of Silverman's initial work that connects smoothing splines to
equivalent kernels. 

\section{Error analysis}

Compared to the analysis for kNN and kernel smoothing, the error analysis for
smoothing splines is a good deal more complex. Hence we'll dedicated an entire
section to learning the tools behind it. The upshot is that we'll be able to use
these tools to analyze a number of other estimators: 

[[ More to come ... sections below to be filled out ... ]]

\section{Multivariate splines}

\section{RKHS methods}

\section{Linear smoothers}

\bibliographystyle{plainnat}
\bibliography{../../common/ryantibs.bib}

\clearpage
\appendix

\section{B-splines}
\label{app:bs}

Note: this appendix is pretty much taken shamelessly from Appendix C of 
\citet{tibshirani2022divided}.

Though the truncated power basis \eqref{eq:tpb} is the simplest basis for
splines, the B-spline basis is just as fundamental, and it was ``there at the 
very beginning'', appearing in Schoenberg's original paper on splines
\citep{schoenberg1946contributions1}. Here we are quoting
\citet{deboor1976splines}, who gives a masterful survey of the history and
properties of B-splines (and points out that the name ``B-spline'' is derived
from Schoenberg's use of the term ``basic spline'', to further advocate for the
idea that B-splines can be seen as \emph{the} basis for splines).

\paragraph{Peano representation.}

\def\st{^{\text{st}}}

There are different ways to construct B-splines; here we cover a construction
based on what is called the \emph{Peano representation} for B-splines. If $f$ is
a $k+1$ times differentiable function $f$ on an interval $[a,b]$ (and its
$(k+1)\st$ derivative is integrable), then by Taylor expansion
\[
f(z) = \sum_{i=0}^k \frac{1}{i!} (D^i f)(a) (z-a)^i + 
\int_a^z \frac{1}{k!} (D^{k+1} f)(x) (z-x)^k \, dx.
\]
Note that we can rewrite this as
\begin{equation}
\label{eq:taylor}
f(z) = \sum_{i=0}^k \frac{1}{i!} (D^i f)(a) (z-a)^i + 
\int_a^b \frac{1}{k!} (D^{k+1} f)(x) (z-x)^k_+ \, dx. 
\end{equation}

Next we will take a particular divided difference on both sides of the above
display. First we recall the definition of a divided difference: with respect to
two centers $z_1,z_2$, it is defined by       
\[
f[z_1,z_2] =  \frac{f(z_2)-f(z_1)}{z_2-z_1},
\]
and more generally, with respect to $k+1$ centers $z_1,\dots,z_{k+1}$, for an
integer $k \geq 1$, it is defined by  
\[
f[z_1,\dots,z_{k+1}] = \frac{f[z_2,\dots,z_{k+1}] -
f[z_1,\dots,z_k]}{z_{k+1}-z_1}.
\]
(For this to reduce to the definition with two centers, when $k=1$, we take by 
convention $f[z]=f(z)$.)   

Returning back to our main thread, we take a divided difference in the Taylor
expansion \eqref{eq:taylor} with respect to arbitrary centers $z_1,\dots,z_{k+2}
\in [a,b]$, where we assume without a loss of generality that $z_1 < \cdots <
z_{k+2}$, and then we use linearity to exchange divided differencing with
integration, yielding   
\begin{equation}
\label{eq:peano}
k! \cdot f[z_1,\dots,z_{k+2}] = \int_a^b (D^{k+1} f)(x)
\underbrace{(\cdot-x)^k_+[z_1,\dots,z_{k+2}]}_{P^k(x; z_{1:(k+2)})} \, dx, 
\end{equation}
where we have also used the fact that a $(k+1)\st$ order divided difference (with
respect to any $k+2$ centers) of a $k\th$ degree polynomial is zero, and we
multiplied both sides by $k!$. To be clear, the notation \smash{$(\cdot -
  x)^k_+[z_1,\dots,z_{k+2}]$} means that we are taking the divided difference of
the function \smash{$z \mapsto (z - x)^k_+$} with respect to centers
$z_1,\dots,z_{k+2}$.    

\paragraph{B-spline definition.}  

The result in \eqref{eq:peano} shows that the $(k+1)\st$ divided difference 
of any (smooth enough) function $f$ can be written as a weighted average of 
its $(k+1)\st$ derivative, in a local neighborhood around the corresponding
centers, where the weighting is given by a universal kernel \smash{$P^k(\cdot;
  z_{1:(k+2)})$} (that does not depend on $f$), which is called the \emph{Peano 
  kernel} formulation for the B-spline; to be explicit, this is
\[
P^k(x; z_{1:(k+2)}) = (\cdot - x)^k_+[z_1,\dots,z_{k+2}].
\]
Since 
\[
(z-x)^k_+ - (-1)^{k+1} (x-z)^k_+ =  (z-x)^k,
\]
and any $(k+1)\st$ order divided difference of the $k\th$ degree polynomial $z
\mapsto (z-x)^k$ is zero, we can rewrite the second-to-last display as
\[
P^k(x; z_{1:(k+2)}) = (-1)^{k+1} (x - \cdot)^k_+[z_1,\dots,z_{k+2}].
\]
The function \smash{$P^k(\cdot; z_{1:(k+2)})$} is called a $k\th$ degree
\emph{B-spline} with knots $z_{1:(k+2)}$. It is a linear combination of $k\th$ 
degree truncated power functions and is hence indeed a $k\th$ degree spline.  

It is often more convenient to deal with the \emph{normalized B-spline}:
\[
M^k(x; z_{1:(k+2)}) = (-1)^{k+1} (z_{k+2}-z_1) 
(x - \cdot)^k_+[z_1,\dots,z_{k+2}]. 
\]
It is easy to show that 
\[
\text{$M^k(\cdot; z_{1:(k+2)})$ is supported on $[z_1,z_{k+2}]$, and  
$M^k(x; z_{1:(k+2)})>0$ for $x \in (z_1,z_{k+2})$}.  
\]
To see the support result, note that for $x > z_{k+2}$, we are taking a divided
difference of all zeros, which of course zero, and for $x < z_1$, we are taking 
a $(k+1)\st$ order divided difference of a polynomial of degree $k$, which is
again zero. To see the positivity result, we can, for example, appeal to 
induction on $k$ and the recursion to come later.

\paragraph{B-spline basis.}  

To build a local basis the space of $k\th$ degree splines with knots
$t_1,\dots,t_r$, which we assume lie in the interior of $[a,b]$, we first define
boundary knots   
\[
t_{-k} < \cdots < t_{-1} < t_0 = a, \quad \text{and} \quad 
b = t_{r+1} < t_{r+2} < \cdots < t_{r+k+1}. 
\]
(Any such values for $t_{-k},\dots,t_0$ and $t_{r+1},\dots,t_{r+k+1}$ will
suffice to produce a basis; in fact, setting $t_{-k}=\cdots=t_0$ and
$t_{r+1}=\cdots=t_{r+k+1}$ would suffice, though this would require us to 
understand how to properly interpret divided differences with repeated centers;
as in Definition 2.49 of \citet{schumaker2007spline}.)  We then define the
normalized B-spline basis \smash{$M^k_j$}, $j=1,\dots,r+k+1$ 
\[
M^k_j = M^k(\cdot ; t_{(j-k-1):j}) \Big|_{[a,b]}, 
\quad j=1,\dots,r+k+1. 
\]
It is clear that each \smash{$M^k_j$}, $j=1,\dots,r+k+1$ is a $k\th$ degree    
spline with knots in $t_1,\dots,t_r$; hence to verify that they are a basis for
this space we only need to show their linear independence, which is
straightforward using the structure of their supports. 

For concreteness, we note that the $0\th$ degree normalized B-splines basis are
simply indicator functions,
\[
M^0_j = 1_{I_j}, \quad j=1,\dots,r+1.
\]
Here $I_0=[t_0,t_1]$ and $I_i=(t_i,t_{i+1}]$, $i=1,\dots,r$, and we use
$t_{r+1}=b$ for notational convenience. We note that this particular choice for
the half-open intervals (left- versus right-side open) is arbitrary, but
consistent with our definition of the truncated power basis \eqref{eq:tpb}
when $k=0$. 

Figure \ref{fig:bs} shows example normalized B-splines of degrees 0 through 3.    

\begin{figure}[tb]
\centering
\includegraphics[width=0.475\textwidth]{bs0.pdf} 
\includegraphics[width=0.475\textwidth]{bs1.pdf} 
\includegraphics[width=0.475\textwidth]{bs2.pdf} 
\includegraphics[width=0.475\textwidth]{bs3.pdf} 
\caption{\it B-splines of degrees 0 through 3. The knot points are marked by
  dashed blue vertical lines.}
\label{fig:bs}
\end{figure}

\paragraph{Recursive formulation.} 

B-splines satisfy a recursion relation that can be seen directly from the
recursive nature of divided differences: for any $k \geq 1$ and centers $z_1 <
\cdots < z_{k+2}$,  
\begin{align*}
(x - \cdot)^k_+ [z_1,\dots,z_{k+2}]
&= \frac{(x - \cdot)^k_+[z_2,\dots,z_{k+2}] - 
(x -\cdot)^k_+[z_1,\dots,z_{k+1}]}{z_{k+2} - z_1} \\
&= \frac{(x-z_{k+2})(x - \cdot)^{k-1}_+[z_2,\dots,z_{k+2}] 
- (x-z_1) (x - \cdot)^{k-1}_+[z_1,\dots,z_{k+1}] }{z_{k+2} - z_1},  
\end{align*} 
where in the second line we applied the Leibniz rule for divided differences 
\[
fg [z_1,\dots,z_{k+1}] = \sum_{i=1}^{k+1} f[z_1,\dots,z_i] g[z_i,\dots,z_{k+1}] 
\]
to conclude that
\begin{align*}
(x - \cdot)^k_+ [z_1,\dots,z_{k+1}] &= (x-z_1) \cdot 
(x - \cdot)^{k-1}_+ [z_1,\dots,z_{k+1}] \\
(x - \cdot)^k_+ [z_2,\dots,z_{k+2}] &= (x - \cdot)^{k-1}_+ 
  [z_2,\dots,z_{k+2}] \cdot (x-z_{k+2}). 
\end{align*}
Translating the above recursion over to normalized B-splines, we get 
\[
M^k(x; z_{1:(k+2)}) = \frac{x-z_1}{z_{k+1}-z_1} \cdot 
M^{k-1}(x; z_{1:(k+1)}) + \frac{z_{k+2}-x}{z_{k+2}-z_2} \cdot 
M^{k-1}(x; z_{2:(k+2)}),  
\]
which means that for the normalized basis, 
\[
M^k_j(x) = \frac{x-t_{j-k-1}}{t_{j-1}-t_{j-k-1}} \cdot
M^{k-1}_{j-1}(x) + \frac{t_j-x}{t_j-t_{j-k}} \cdot M^{k-1}_j(x), 
\quad j=1,\dots,r+k+1.  
\]
Above, we naturally interpret \smash{$M^{k-1}_0 = M^{k-1}(\cdot; 
  t_{-k:0})|_{[a,b]}$} and \smash{$M^{k-1}_{r+k+1} = M^{k-1}(\cdot;   
  t_{(r+1):(r+k+1)})|_{[a,b]}$}. 

The above recursions are very important, both for verifying numerous properties
of B-splines and for computational purposes. In fact, many authors prefer to
use recursion to define a B-spline basis in the first place.

\end{document}

