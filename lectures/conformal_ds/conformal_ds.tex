\documentclass{article}

\def\ParSkip{} 
% Packages
\usepackage{amssymb,amsmath,amsthm,bbm}
\usepackage{verbatim,float,url,dsfont}
\usepackage{graphicx,subfigure,psfrag}
\usepackage{algorithm,algorithmic}
\usepackage{mathtools,enumitem}
\usepackage{multirow}
\usepackage{ragged2e}
\usepackage{xr-hyper}

\usepackage[colorlinks=true,citecolor=blue,urlcolor=blue,linkcolor=blue]{hyperref}
\usepackage[margin=1in]{geometry}
\usepackage[round]{natbib}

\usepackage[utf8]{inputenc} % allow utf-8 input
\usepackage[T1]{fontenc}    % use 8-bit T1 fonts
\usepackage{booktabs}       % professional-quality tables
\usepackage{nicefrac}         % compact symbols for 1/2, etc.
\usepackage{microtype}      % microtypography

\ifdefined\TimesFont 
\usepackage{times} % use times font
\fi

\ifdefined\ParSkip 
\usepackage{parskip} % use par skip
\fi

% Theorems and such
\newtheorem{theorem}{Theorem}
\newtheorem{lemma}{Lemma}
\newtheorem{corollary}{Corollary}
\newtheorem{proposition}{Proposition}
\theoremstyle{definition}
\newtheorem{remark}{Remark}
\newtheorem{definition}{Definition}

% Assumption
\newtheorem*{assumption*}{\assumptionnumber}
\providecommand{\assumptionnumber}{}
\makeatletter
\newenvironment{assumption}[2]{
  \renewcommand{\assumptionnumber}{Assumption #1#2}
  \begin{assumption*}
  \protected@edef\@currentlabel{#1#2}}
{\end{assumption*}}
\makeatother

% Widebar
\makeatletter
\newcommand*\rel@kern[1]{\kern#1\dimexpr\macc@kerna}
\newcommand*\widebar[1]{%
  \begingroup
  \def\mathaccent##1##2{%
    \rel@kern{0.8}%
    \overline{\rel@kern{-0.8}\macc@nucleus\rel@kern{0.2}}%
    \rel@kern{-0.2}%
  }%
  \macc@depth\@ne
  \let\math@bgroup\@empty \let\math@egroup\macc@set@skewchar
  \mathsurround\z@ \frozen@everymath{\mathgroup\macc@group\relax}%
  \macc@set@skewchar\relax
  \let\mathaccentV\macc@nested@a
  \macc@nested@a\relax111{#1}%
  \endgroup
}
\makeatother

% Min and max
\newcommand{\argmin}{\mathop{\mathrm{argmin}}}
\newcommand{\argmax}{\mathop{\mathrm{argmax}}}
\newcommand{\minimize}{\mathop{\mathrm{minimize}}}
\newcommand{\maximize}{\mathop{\mathrm{maximize}}}
\newcommand{\st}{\mathop{\mathrm{subject\,\,to}}}

% Shortcuts
\def\R{\mathbb{R}}
\def\C{\mathbb{C}}
\def\Z{\mathbb{Z}}
\def\N{\mathbb{N}}
\def\E{\mathbb{E}}
\def\P{\mathbb{P}}
\def\T{\mathsf{T}}
\def\Cov{\mathrm{Cov}}
\def\Var{\mathrm{Var}}
\def\indep{\perp\!\!\!\perp}
\def\th{^{\text{th}}}
\def\tr{\mathrm{tr}}
\def\df{\mathrm{df}}
\def\dim{\mathrm{dim}}
\def\col{\mathrm{col}}
\def\row{\mathrm{row}}
\def\nul{\mathrm{null}}
\def\rank{\mathrm{rank}}
\def\nuli{\mathrm{nullity}}
\def\spa{\mathrm{span}}
\def\sign{\mathrm{sign}}
\def\supp{\mathrm{supp}}
\def\diag{\mathrm{diag}}
\def\aff{\mathrm{aff}}
\def\conv{\mathrm{conv}}
\def\dom{\mathrm{dom}}
\def\hy{\hat{y}}
\def\hf{\hat{f}}
\def\hmu{\hat{\mu}}
\def\halpha{\hat{\alpha}}
\def\hbeta{\hat{\beta}}
\def\htheta{\hat{\theta}}
\def\cA{\mathcal{A}}
\def\cB{\mathcal{B}}
\def\cD{\mathcal{D}}
\def\cE{\mathcal{E}}
\def\cF{\mathcal{F}}
\def\cG{\mathcal{G}}
\def\cK{\mathcal{K}}
\def\cH{\mathcal{H}}
\def\cI{\mathcal{I}}
\def\cL{\mathcal{L}}
\def\cM{\mathcal{M}}
\def\cN{\mathcal{N}}
\def\cP{\mathcal{P}}
\def\cS{\mathcal{S}}
\def\cT{\mathcal{T}}
\def\cW{\mathcal{W}}
\def\cX{\mathcal{X}}
\def\cY{\mathcal{Y}}
\def\cZ{\mathcal{Z}}

\usepackage[normalem]{ulem}

\title{Conformal Prediction Under Distribution Shift \\ \smallskip
\large Advanced Topics in Statistical Learning, Spring 2023 \\ \smallskip
Ryan Tibshirani }
\date{}

\begin{document}
\maketitle
\RaggedRight
\vspace{-50pt}

Note: we're following the context, problem setup, notation, etc. from the last
lecture on conformal prediction.

In this lecture we cover conformal methods that apply beyond the i.i.d.\
setting. This is a very active and recent topic of research, and it's possible
(or even likely) that what's considered fundamental in this area will change in
the next few years. Until then, there will be numerous topics from which we can
``pick and choose'' for a lecture like this one. We've chosen three such
topics---and shamelessly (shamefully?), for two of these, you'll notice an 
overlap between the authors list and the author of these lecture notes. We
should be clear that there is plenty of other interesting work out there that we
can't cover in just one lecture. 

\section{Likelihood-weighted conformal prediction}

\def\tP{\tilde{P}}

We first cover a likelihood-weighted conformal prediction method, due to
\citet{tibshirani2019conformal}. A primary motivation will be the setting of  
\emph{covariate shift}, where 
\begin{equation}
\label{eq:cov_shift}
\begin{gathered}
(X_i, Y_i) \sim P = P_X \times P_{Y|X}, \; \text{independently, for
  $i=1,\dots,n$}, \\ 
(X_{n+1}, Y_{n+1}) \sim \tP = \tP_X \times P_{Y|X}, \; \text{independently}. 
\end{gathered}
\end{equation}
Notice that the conditional distribution of $Y|X$ is assumed to be the same for
both the training and test data, but the distribution of $X$ is allowed to
change, i.e., we allow \smash{$\tP_X \not= P_X$}. This is a general framework of
great interest, because it encompasses may important problem settings. For
example, we could have done some kind of structured covariate sampling for our
training set (demographically, geographically, etc.) but then we do prediction
``in the wild'', where the mix of covariates is different.

The first thing we could ask is: does this even matter for conformal prediction?
That is, if we observed data according to \eqref{eq:cov_shift}, and computed the 
usual conformal prediction intervals, then would we see a problem with coverage?
The top row of Figure \ref{fig:airfoil} provides an answer, empirically. This is 
taken from \citet{tibshirani2019conformal}, and shows the results of an
experiment in which, over 5000 repetitions, two test sets are drawn: one without  
covariate shift (results in red), and one with covariate shift (in blue). The
top left panel shows the test coverage of split conformal prediction intervals 
(drawn as histograms, over the 5000 repetitions). We can see that coverage fails
quite noticeably in the covariate shift setting.

\begin{figure}[htb]
\hspace{-25pt}
\includegraphics[width=0.525\textwidth]{airfoil_cov.pdf}
\includegraphics[width=0.525\textwidth]{airfoil_len.pdf}
\caption{\it Experiments for split conformal prediction under covariate
  shift. All results are aggregated over 5000 repetitions, each of which
  randomly forms training and test sets. Top row: the usual split conformal
  prediction with and without covariate shift. Middle row: weighted split
  conformal using the true (and in general unknown) likelihood ratio between
  test and training covariate feature distributions, compared to ordinary split 
  conformal without covariate shift but in a problem with a comparable effective 
  sample size. Bottom row: weighted split conformal using an estimated
  likelihood ratio from running classification with logistic regression or
  random forests. Credit: \citet{tibshirani2019conformal}.}       
\label{fig:airfoil}
\end{figure}

To remedy this, we are going to work with a weighed empirical distribution of 
conformity scores, rather than the usual (unweighted) empirical distribution. 
And to approach this argument, it helps to build intuition by looking back at
the first key idea behind conformal, which recall, used ranks in order to
construct adjusted empirical quantiles. 

\paragraph{Revisiting the first key idea: rank-based quantiles.}

\def\Quantile{\mathrm{Quantile}}

The last lecture proved the following fact. If $R_1,\dots,R_{n+1}$ are
exchangeable random variables, then for any $\alpha \in (0,1)$, 
\[
\P\bigg\{ R_{n+1} \leq \Quantile\bigg( \frac{\lceil (1-\alpha)(n+1) \rceil}{n};
\, \frac{1}{n} \sum_{i=1}^n \delta_{R_i} \bigg) \bigg\} \geq 1-\alpha.
\]
We can equivalently express this as
\[
\P\bigg\{ R_{n+1} \leq \Quantile\bigg( \frac{\lceil (1-\alpha)(n+1) \rceil}{n+1}; 
\, \frac{1}{n+1} \sum_{i=1}^{n+1} \delta_{R_i} \bigg) \geq 1-\alpha,
\]
because the event in each of the last two displays is equivalent to the
statement that $R_{n+1}$ is among the \smash{$\lceil (1-\alpha)(n+1) \rceil$}
smallest of $R_1,\dots,R_{n+1}$. The last display is itself equivalent to
\[
\P\bigg\{ R_{n+1} \leq \Quantile\bigg( 1-\alpha; \, \frac{1}{n+1}
\sum_{i=1}^{n+1} \delta_{R_i} \bigg) \bigg\} \geq 1-\alpha, 
\]
because the quantile function of the empirical distribution of
$R_1,\dots,R_{n+1}$, only changes in increments of $1/(n+1)$ (and will
automatically round up to the nearest increment until it captured sufficient
probability mass to exceed $1-\alpha$). Finally, it turns out that we can
equivalently express the last display as  
\begin{equation}
\label{eq:quantile}
\P\bigg\{ R_{n+1} \leq \Quantile\bigg( 1-\alpha; \, \frac{1}{n+1} \sum_{i=1}^n
\delta_{R_i} + \frac{1}{n+1} \delta_\infty \bigg) \bigg\} \geq 1-\alpha.
\end{equation}
This can be seen by applying the following fact to the complements of the two
events in the previous two displays: for a discrete distribution $F$ with
support points $a_1,\dots,a_k \in \R$, denoting $q=\Quantile(\beta; F)$, if we
reassign the points $a_i > q$ to arbitrary values strictly larger than $q$,
yielding a new distribution $F'$, then the level $\beta$ quantile remains
unchanged, $\Quantile(\beta; F) = \Quantile(\beta; F')$.  

\paragraph{Alternate proof of the quantile result \eqref{eq:quantile}.}

We will now prove \eqref{eq:quantile} from a new perspective (no longer by
reducing it to a statement about the rank of $R_{n+1}$ among
$R_1,\dots,R_{n+1}$) that will enable us to extend this result to a more general
setting. The basic idea is to condition on the unlabeled collection of values
obtained by our random variables $R_1,\dots,R_{n+1}$, then inspect the
probabilities that the last random variable $R_{n+1}$ attains each one of these
values.  

Denote by $f$ the probability density function (or mass function, or more
generally, Radon-Nikodym derivative with respect to an arbitrary base measure)
of the joint sample $R_1,\dots,R_{n+1}$. Exchangeability means
\[
f(r_1,\dots,r_{n+1}) = f(r_{\sigma(1)}, \dots, r_{\sigma(n+1)}), \quad \text{for 
  all permutations $\sigma$}.
\] 
For simplicity, and without loss of generality, assume that there are almost
surely no ties among the scores $R_1,\dots,R_{n+1}$. Let $E_r$ be the event that 
$\{R_1,\dots,R_{n+1}\}=\{r_1,\dots,r_{n+1}\}$. Then for each $i$,  
\begin{align*}
\P(R_{n+1}=r_i \,|\, E_r)
&= \frac{\sum_{\sigma : \sigma(n+1)=i} f(r_{\sigma(1)}, \dots, r_{\sigma(n+1)})}  
{\sum_\sigma f(r_{\sigma(1)}, \dots, r_{\sigma(n+1)})} \\
&= \frac{\sum_{\sigma: \sigma(n+1)=i} f(r_1, \dots, r_{n+1})}  
{\sum_\sigma f(r_1, \dots, r_{n+1})} \\
&= \frac{n!}{(n+1)!} = \frac{1}{n+1}. 
\end{align*}
This shows that the distribution of $R_{n+1}|E_r$ is uniform on the set 
$\{r_1,\dots,r_{n+1}\}$, that is,
\[
R_{n+1}|E_r \sim \frac{1}{n+1}\sum_{i=1}^{n+1}\delta_{r_i},
\]
and it follows, since $F(Q(t)) \geq t$ for any cumulative distribution function
$F$ and corresponding quantile function $Q$, that   
\[
\P\bigg\{ R_{n+1} \leq \Quantile\bigg( 1-\alpha; \,
\frac{1}{n+1}\sum_{i=1}^{n+1}\delta_{r_i} \bigg) \, \bigg| \, 
E_r\bigg\} \geq 1-\alpha,
\]
This is the same as
\[
\P\bigg\{ R_{n+1} \leq \Quantile\bigg( 1-\alpha; \,
\frac{1}{n+1}\sum_{i=1}^{n+1}\delta_{R_i}\bigg) \, \bigg| \,  
E_r\bigg\} \geq 1-\alpha,
\]
and we can marginalize to obtain   
\[
\P\bigg\{ R_{n+1} \leq \Quantile\bigg( 1-\alpha; \,
\frac{1}{n+1}\sum_{i=1}^{n+1}\delta_{R_i}\bigg) \bigg\} \geq 1-\alpha.
\]
This is the display right above \eqref{eq:quantile}, and by the same argument as
given above, it is equivalent to \eqref{eq:quantile}.

\subsection{Weighted exchangeability: quantile lemma}

Though the alternate proof we just gave is a bit longer than the standard
reduction to ranks, it is important because it allows us to move past the
setting of exchangeable scores $R_1,\dots,R_{n+1}$. In words, after revealing
(conditioning on) the set of values obtained by the scores, we need to be able
to answer the following question: \emph{what is the probability with which any
  given value is that of the test score?}   

This question still has a relatively clean answer when $R_1,\dots,R_{n+1}$ are
\emph{weighted exchangeable}, which is a generalization of exchangeability, and
specifies that the random variables have a density (or mass function, or more
generally, Radon-Nikodym derivative with respect to an arbitrary base measure)
of the form   
\begin{equation}
\label{eq:weighted_exch}
f(r_1,\dots,r_{n+1}) = \prod_{i=1}^{n+1} w_i(r_i) \cdot g(r_1,\dots,r_{n+1}),  
\end{equation}
where $g$ is any function that is permutation invariant, i.e.,
\smash{$g(r_1,\dots,r_{n+1}) = g(r_{\sigma(1)}, \dots, r_{\sigma(n+1)})$}, for
any permutation $\sigma$.

We now have the following extension of \eqref{eq:quantile}, stated as a lemma,
for concreteness.  

\begin{lemma}
\label{lem:weighted_quant}
Let $Z_i$, $i=1,\dots,n+1$ be weighted exchangeable random variables, with
respect to weight functions $w_1,\dots,w_{n+1}$. Assume without loss of
generality that these are distinct almost surely. Let 
\[
R_i = V(Z_i; \, Z_1,\dots,Z_{n+1}), \quad i=1,\dots,n+1, 
\]
where $V$ is an arbitrary score function that is symmetric in its last $n+1$
arguments, and define
\begin{equation}
\label{eq:pw}
p^w_i(z_1,\dots,z_{n+1}) = 
\frac{\sum_{\sigma : \sigma(n+1)=i} \prod_{j=1}^{n+1} w_j(z_{\sigma(j)})}
{\sum_\sigma \prod_{j=1}^{n+1} w_j(z_{\sigma(j)})}, \quad i=1,\dots,n+1,   
\end{equation} 
where the sums are over permutations $\sigma$ of the numbers $1,\dots,n+1$.
Then for any $\alpha \in (0,1)$,  
\begin{equation}
\label{eq:weighted_quant}
\P\bigg\{ R_{n+1} \leq \Quantile\bigg( 1-\alpha; \, \sum_{i=1}^n 
p^w_i(Z_1,\dots,Z_{n+1}) \delta_{R_i} + p^w_{n+1}(Z_1,\dots,Z_{n+1}) 
\delta_\infty \bigg) \bigg\} \geq 1-\alpha.
\end{equation}
\end{lemma}

\begin{proof}
We follow the same general strategy from the alternate proof of
\eqref{eq:quantile}. Let $E_z$ denote the event that
$\{Z_1,\dots,Z_{n+1}\}=\{z_1,\dots,z_{n+1}\}$, and let $r_i=V(z_i; \,
z_1,\dots,z_{n+1})$, for $i=1,\dots,n+1$. Let $f$ be the density function 
of the joint sample $Z_1,\dots,Z_{n+1}$. For each $i$, we have
\begin{equation}
\label{eq:general_calc}
\P(R_{n+1} = r_i \,|\, E_z) = \P(Z_{n+1} = z_i \,|\, E_z)
= \frac{\sum_{\sigma : \sigma(n+1)=i} f(z_{\sigma(1)},\dots, z_{\sigma(n+1)})} 
{\sum_\sigma f(z_{\sigma(1)},\dots, z_{\sigma(n+1)})},
\end{equation}
and as $Z_1,\dots,Z_{n+1}$ are weighted exchangeable,
\begin{align*}
\frac{\sum_{\sigma : \sigma(n+1)=i} f(z_{\sigma(1)},\dots, z_{\sigma(n+1)})}  
{\sum_\sigma f(z_{\sigma(1)},\dots, z_{\sigma(n+1)})} &= 
\frac{\sum_{\sigma : \sigma(n+1)=i} \prod_{j=1}^{n+1} w_j(z_{\sigma(j)}) \cdot 
  g(z_{\sigma(1)},\dots,z_{\sigma(n+1)})}
{\sum_\sigma \prod_{j=1}^{n+1} w_j(z_{\sigma(j)}) \cdot
  g(z_{\sigma(1)},\dots,z_{\sigma(n+1)})}  \\
&= \frac{\sum_{\sigma : \sigma(n+1)=i} \prod_{j=1}^{n+1} w_j(z_{\sigma(j)})
  \cdot g(z_1,\dots,z_{n+1})} 
{\sum_\sigma\prod_{j=1}^{n+1} w_j(z_{\sigma(j)}) \cdot g(z_1,\dots,z_{n+1})} \\ 
&= p^w_i(z_1,\dots,z_{n+1}).
\end{align*}
In other words,  
\[
R_{n+1}|E_z \sim \sum_{i=1}^{n+1} p^w_i(z_1,\dots,z_{n+1})\delta_{r_i},
\]
which implies that
\[
\P\bigg\{ R_{n+1} \leq \Quantile\bigg( 1-\alpha; \, \sum_{i=1}^{n+1}
p^w_i(z_1,\dots,z_{n+1})\delta_{r_i}\bigg) \, \bigg| \, E_z\bigg\} \geq
1-\alpha.  
\]
This is equivalent to  
\[
\P\bigg\{ R_{n+1} \leq \Quantile\bigg( 1-\alpha; \, \sum_{i=1}^{n+1} 
p^w_i(Z_1,\dots,Z_{n+1}) \delta_{R_i}\bigg) \, \bigg| \, E_z\bigg\} \geq
1-\alpha,  
\]
and after marginalizing, 
\[
\P\bigg\{ R_{n+1} \leq \Quantile\bigg( 1-\alpha; \, \sum_{i=1}^{n+1} 
p^w_i(Z_1,\dots,Z_{n+1}) \delta_{R_i}\bigg) \bigg\} \geq 1-\alpha. 
\]
Finally, by the same arguments as before, we can change the point mass at 
$R_{n+1}$ to one at $\infty$, which proves \eqref{eq:weighted_quant} as
desired. 
\end{proof}

We remark that computation of the probability weights in \eqref{eq:pw} is very
difficult in general, due to the combinatorial form (note that this actually
reduces to computing what is known as a \emph{matrix permanent}, which is known
to be hard). However, for cetain weighted exchangeable structures, it can be
easy, as we will see a bit later for covariate shift. 

\subsection{Weighted exchangeability: conformal prediction}

\def\hC{\hat{C}}

A weighted version of conformal prediction follows from Lemma
\ref{lem:weighted_quant}, which we state next as a theorem, for concreteness. 

\begin{theorem}
\label{thm:weighted_conf}
Assume that $Z_i=(X_i,Y_i) \in \cX \times \cY$, $i=1,\dots,n+1$ are weighted   
exchangeable with weight functions $w_1,\dots,w_{n+1}$. Define a weighted
conformal set (based on the first $n$ samples) at a point $x \in \cX$, with
nominal error level $\alpha \in (0,1)$ as follows. Let   
\begin{equation}
\label{eq:scores}
\begin{aligned}
R_i^{(x,y)} &= V\Big( (X_i,Y_i); \, Z_1,\dots,Z_n,(x,y)\Big), \quad
  i=1,\dots,n, \\ 
R_{n+1}^{(x,y)} &= V\Big( (x,y); \, Z_1,\dots,Z_n,(x,y) \Big), 
\end{aligned}
\end{equation}
for an arbitrary score function $V$ that is symmetric in its last $n+1$
arguments, and 
\begin{equation}
\label{eq:weighted_conf}
\hC^w_n(x) = \bigg\{ y : R_{n+1}^{(x,y)} \leq \Quantile\bigg( 1-\alpha; 
\, \sum_{i=1}^n p^w_i\big( Z_1,\dots,Z_n,(x,y) \big) \delta_{R_i^{(x,y)}} + 
p^w_{n+1}\big( Z_1,\dots,Z_n,(x,y) \big) \delta_\infty \bigg) \bigg\},  
\end{equation}
where \smash{$p^w_i$}, $i=1,\dots,n+1$ are as in \eqref{eq:pw}. Then
\smash{$\hC^w_n$} satisfies     
\begin{equation}
\label{eq:weighted_cov}
\P\big( Y_{n+1} \in \hC^w_n(X_{n+1}) \big) \geq 1-\alpha.  
\end{equation}
\end{theorem}

\begin{proof}
Abbreviate \smash{$R_i=R_i^{(X_{n+1},Y_{n+1})}$}, $i=1,\dots,n+1$. By
construction 
\[
Y_{n+1} \in \hC^w_n(X_{n+1}) \iff 
R_{n+1} \leq \Quantile\bigg( 1-\alpha; \, \sum_{i=1}^n p^w_i(Z_1,\dots,Z_{n+1}) 
\delta_{R_i} + p^w_{n+1}(Z_1,\dots,Z_{n+1}) \delta_\infty \bigg),
\]
and applying Lemma \ref{lem:weighted_quant} gives the result.  
\end{proof}

\paragraph{Split version.}

The split conformal version of the above result can be viewed as a special case  
where the score function relies on a point predictor that has been fit on an
external data set. For example, if we take it to be $V(x,y) = |y-\mu_0(x)|$,
where $\mu_0$ has been fit on a data set $Z_0$, then \eqref{eq:weighted_conf} 
simplifies to       
\[
\hC^w_n(x) = \mu_0(x) \pm \Quantile\bigg(1-\alpha; \, 
\sum_{i=1}^n p^w_i\big( Z_1,\ldots,Z_n,(x,y) \big) \delta_{|Y_i - \mu_0(X_i)|}
+ p^w_{n+1}\big( Z_1,\ldots,Z_n,(x,y) \big) \delta_\infty \bigg),
\]
and by \eqref{eq:weighted_cov}, this has coverage at least $1-\alpha$,
conditional on $Z_0$.   

\paragraph{CDF form.}

\def\hF{\hat{F}}

The analogous CDF form of the conformal set in \eqref{eq:weighted_conf} is as
follows: 
\begin{equation}
\label{eq:weighted_cdf}
\hC^w_n(x) = \bigg\{ y : \sum_{i=1}^{n+1} p^w_i\big( Z_1,\dots,Z_n,(x,y) \big) 
1\Big\{ R_i^{(x,y)} \leq R^{(x,y)}_{n+1} \Big\} \leq \lceil 1-\alpha \rceil_w
\bigg\},   
\end{equation}
where \smash{$\lceil 1-\alpha \rceil_w = \min\{ \tau \in \mathrm{range}(\hF^w_n)
  : \tau \geq 1-\alpha\}$} and we use \smash{$\hF^w_n$} to denote the (random)
CDF of the discrete distribution \smash{$\sum_{i=1}^n p^w_i(Z_1,\dots,Z_n,(x,y))
  \delta_{R^{(x,y)}_i} + p^w_{n+1}(Z_1,\ldots,Z_n,(x,y)) \delta_\infty$}. 
Compared to the CDF form of the ordinary unweighted conformal prediction set,
from the last lecture, the form in \eqref{eq:weighted_conf} is more
complicated---we need to adjust the nominal level of $1-\alpha$ upwards so that
it lies in the range of the CDF of the weighted score distribution, and here
this distribution is random, so the adjustment is itself random. This is the
main reason we worked with the quantile form in \eqref{eq:weighted_conf} in the
first place, since we can always use the unadjusted level $1-\alpha$ and
completely avoid any such complications.

\paragraph{Auxiliary randomization for exact coverage.}

It is also worth noting that we can achieve exact coverage by using auxiliary
randomization, either in CDF or quantile form. Applying our previous
randomization trick (from the last lecture) to the CDF form
\eqref{eq:weighted_cdf} gives 
\begin{multline*}
\hC_n^{w,*}(x) = \bigg\{ y : \sum_{i=1}^n p^w_i\big( Z_1,\dots,Z_n,(x,y)
\big) 1\Big\{ R_i^{(x,y)} < R^{(x,y)}_{n+1} \Big\} +{} \\ U \sum_{i=1}^{n+1} 
p^w_i\big( Z_1,\dots,Z_n,(x,y) \big) 1\Big\{ R_i^{(x,y)} = R^{(x,y)}_{n+1}
\Big\} \leq 1-\alpha \bigg\},
\end{multline*}
where $U \sim \mathrm{Unif}(0,1)$, independent of everything else. This is
fairly simple and intuitive---it is free of any level adjustments needed in the   
unrandomized CDF-based set in \eqref{eq:weighted_cdf}. Meanwhile, we can also 
randomize the quantile form \eqref{eq:weighted_conf}, as in  
\begin{multline*}
\hC^{w,*}_n(x) = \bigg\{ y : R_{n+1}^{(x,y)} \leq {} \\
B^w \cdot \Quantile\bigg( 1-\alpha; \, \sum_{i=1}^n p^w_i\big(
Z_1,\dots,Z_n,(x,y) \big) \delta_{R_i^{(x,y)}} + p^w_{n+1}\big(
Z_1,\dots,Z_n,(x,y) \big) \delta_\infty \bigg) +{} \\ 
(1-B^w) \cdot \Quantile\bigg( \lfloor 1-\alpha \rfloor_w; \, 
\sum_{i=1}^n p^w_i\big( Z_1,\dots,Z_n,(x,y) \big) \delta_{R_i^{(x,y)}} +
p^w_{n+1}\big( Z_1,\dots,Z_n,(x,y) \big) \bigg\},     
\end{multline*}
where now \smash{$\lfloor 1-\alpha \rfloor_w = \max\{ \tau \in
  \mathrm{range}(\hF^w_n) : \tau \leq 1-\alpha\}$} with \smash{$\hF^w_n$}
denoting the (random) CDF of the discrete distribution \smash{$\sum_{i=1}^n
  p^w_i(Z_1,\dots,Z_n,(x,y))  \delta_{R^{(x,y)}_i} +
  p^w_{n+1}(Z_1,\ldots,Z_n,(x,y)) \delta_\infty$} as before, and  
\[
B^w \sim \mathrm{Bernoulli}\bigg( \frac{1-\alpha - \lfloor 1-\alpha \rfloor_w}
{\lceil 1-\alpha \rceil_w - \lfloor 1-\alpha \rfloor_w} \bigg),
\]
independent of everything else. Arguably, the randomized conformal set from 
the second-to-last display is actually less simple and intuitive---we must
introduce level adjustments that were not needed in
\eqref{eq:weighted_conf}. Ultimately, either version of the set
\smash{$\hC^{w,*}_n$} defined above (it is not clear that the two are 
equivalent) satisfies       
\[
\P\big( Y_{n+1} \in \hC^{w,*}_n(X_{n+1}) \big) = 1-\alpha.  
\]

\subsection{Conformal prediction for covariate shift}

We now show how to apply the above results to get a version of conformal
prediction for covariate shift problems, as developed in
\citet{tibshirani2019conformal}. However, we note that Theorem 
\ref{thm:weighted_conf} can also be used as a basis for developing conformal
methods in other non-i.i.d.\ settings, such as label shift
\citep{podkopaev2021distribution}, causal inference \citep{lei2021conformal},
experimental design \citep{fannjiang2022conformal}, and survival analysis
\citep{candes2023conformalized}. 

\begin{corollary}
\label{cor:weighted_conf_cs}
Assume that $Z_i=(X_i,Y_i)$, $i=1,\dots,n+1$ obey the model
\eqref{eq:cov_shift}.  Assume that \smash{$\tP_X$} is absolutely continuous with
respect to $P_X$, and denote \smash{$w=d\tP_X/dP_X$}.  Define a weighted
conformal set (based on the first $n$ samples) at a point $x \in \cX$, with
nominal error level $\alpha \in (0,1)$, by 
\begin{equation}
\label{eq:weighted_conf_cs}
\hC^w_n(x) = \bigg\{ y : R_{n+1}^{(x,y)} \leq \Quantile\bigg( 1-\alpha; 
\, \sum_{i=1}^n \pi^w_i(x) \delta_{R_i^{(x,y)}} + \pi^w_{n+1}(x) \delta_\infty 
\bigg) \bigg\}, 
\end{equation}
where \smash{$R_i^{(x,y)}$}, $i=1,\ldots,n+1$ are conformity scores as in
\eqref{eq:scores}, for an arbitrary score function $V$ that is symmetric in its
last $n+1$ arguments, and
\begin{equation}
\label{eq:pw_cs}
\pi^w_i(x) = \frac{w(X_i)}{\sum_{j=1}^n w(X_j) + w(x)}, \; i=1,\ldots,n, 
\quad \text{and} \quad 
\pi^w_{n+1}(x) = \frac{w(x)}{\sum_{j=1}^n w(X_j) + w(x)}. 
\end{equation}
Then \smash{$\hC^w_n$} satisfies 
\begin{equation}
\label{eq:weighted_cov_cs}
\P\big( Y_{n+1} \in \hC^w_n(X_{n+1}) \big) \geq 1-\alpha. 
\end{equation}
\end{corollary}

\begin{proof}
It is straightforward to see that the independent draws $Z_i = (X_i,Y_i)$,
$i=1,\ldots,n+1$ are weighted exchangeable \eqref{eq:weighted_exch} with
$w_i\equiv 1$ for $i=1,\ldots,n$, and $w_{n+1}((x,y))=w(x)$. In this special
case, the probabilities in \eqref{eq:pw} simplify to  
\[
p^w_i(z_1,\ldots,z_{n+1}) = \frac{\sum_{\sigma : \sigma(n+1)=i}
  w(x_i)}{\sum_\sigma w(x_{\sigma(n+1)})} =
\frac{w(x_i)}{\sum_{j=1}^{n+1}w(x_j)}, \quad i=1,\ldots,n+1,
\]
in other words, \smash{$p^w_i(Z_1,\ldots,Z_n,(x,y)) = \pi^w_i(x)$},
$i=1,\ldots,n+1$, where the latter are as in \eqref{eq:pw_cs}. Applying Theorem 
\ref{thm:weighted_conf} gives the result.  \end{proof}  

The same remarks as before apply here: a split conformal version follows as a
special case (via a particular score function) and exact coverage in
\eqref{eq:weighted_cov_cs} can be achieved by randomizing the quantile in 
\eqref{eq:weighted_conf_cs}.

Looking back at Figure \ref{fig:airfoil}, the middle row provides an example of
the (split version of the) conformal set in \eqref{eq:weighted_conf_cs}, with
oracle knowledge (in orange) of the likelihood ratio weight function
\smash{$w=d\tP_X/dP_X$}. We can see from the middle left panel that its coverage
is restored compared to the naive application of conformal in the covariate
shift problem, from the top row. However, we also see that the dispersion in the
coverage histogram from weighted conformal (over the 5000 repetitions of the
experiment) is larger than that of ordinary conformal without covariate shift
(in red), from the top row. This is because, with non-uniform weights due to
covariate shift, we are effectively operating at a lower sample size. The middle
row thus also displays the results (in purple) of usual conformal prediction in
a problem without covariate shift but at the same effective sample size, defined
as
\[
\hat{n} = \bigg( \frac{\sum_{i=1}^n |w(X_i)|}{\sqrt{\sum_{i=1}^n |w(X_i)|^2}}
\bigg)^2 = \bigg( \frac{\|w(X_{1:n})\|_1}{\|w(X_{1:n})\|_2} \bigg)^2,
\]
where we abbreviate $w(X_{1:n})=(w(X_1),\ldots,w(X_n)) \in \R^n$. We see that
its coverage dispersion is about the same. Interestingly (and unfortunately for
the likelihood-weighted method), even with the effective sample size correction, 
the usual conformal prediction intervals are shorter than the weighted conformal
prediction intervals, as shown in the middle right panel.

\subsection{Estimating the likelihood ratio from unlabeled data}

Here we describe how to estimate \smash{$w=d\tP_X/dP_X$}, the likelihood ratio
of interest, when we have access to unlabeled data $X_{n+1},\dots,X_{n+m} \in
\cX$ at prediction time. (This is sometimes called the transductive or
semi-supervised setting in machine learning.) We can use any classifier that
estimated probabilities of class membership, such as logistic regression or
random forests. We proceed as follows: we train the classifier on feature-class
pairs $(X_i,C_i)$, $i=1,\ldots,n+m$,  where $C_i=0$ for $i=1,\ldots,n$ and
$C_i=1$ for $i=n+1,\ldots,n+m$. Noting that   
\[
\frac{\P(C=1 | X=x)}{\P(C=0 | X=x)}
= \frac{\P(C=1)}{\P(C=0)} \frac{d\tP_X}{dP_X}(x),
\]
we can thus view the conditional odds ratio $w(x)=\P(C=1|X=x)/\P(C=0|X=x)$ as an
equivalent representation for the oracle weight function---since we actually
only need to know the likelihood ratio up to a proportionality 
constant. Therefore, if \smash{$\hat{p}(x)$} is an estimate of $\P(C=1|X=x)$
obtained by fitting a probabilistic classifier to the data $(X_i,C_i)$,
$i=1,\ldots,n+m$, then we can use   
\begin{equation}
\label{eq:w_estimate}
\hat{w}(x) = \frac{\hat{p}(x)}{1-\hat{p}(x)}
\end{equation}
as our estimated weight function for the calculation of probabilities
\eqref{eq:pw_cs}, needed for the weighted conformal set
\eqref{eq:weighted_conf_cs}. The better calibrated the classifier, the better
the estimated weighted in \eqref{eq:w_estimate} will be.  

Looking back once again at Figure \ref{fig:airfoil}, the bottom row shows the
results of using this method to estimate the weights using logistic regression
(in gray) and random forests (in green). Both classifiers provide reasonably
good prediction sets in the end (logistic regression is actually well-specified 
in this experiment, so its favorable performance should not be surprising). 

\subsection{Conformal prediction for structured-X settings}

We saw that a particularly simple and computationally efficient application of
Theorem \ref{thm:weighted_conf} was the covariate shift problem. Now we go in
the opposite direction: make it even more general and add the same time, even
more computationally intractable (at least at face value). The next result
essentially already follows from what we proved in Lemma
\ref{lem:weighted_quant} and Theorem \ref{thm:weighted_conf}: we just stop at
\eqref{eq:general_calc}, without simplifying further. 

\begin{theorem}
\label{thm:weighted_conf_gen}
Assume that $Z_i=(X_i,Y_i)$, $i=1,\dots,n+1$ are distributed according to:
\begin{gather*}
(X_1,\dots,X_{n+1}) \sim \Lambda, \\
Y_i|X_i \sim P_{Y|X}, \; \text{independently, for $i=1,\dots,n$}.
\end{gather*}
Let $\lambda$ denote the density (or mass function, or more generally,
Radom-Nikodym derivative with respect to an arbitrary base measure) of
$Q$. Define a weighted conformal set (based on the first $n$ samples) at a point
$x \in \cX$, with nominal error level $\alpha \in (0,1)$, by   
\begin{equation}
\label{eq:weighted_conf_gen}
\hC^\lambda_n(x) = \bigg\{ y : R_{n+1}^{(x,y)} \leq \Quantile\bigg( 1-\alpha;  
\, \sum_{i=1}^n p^\lambda_i\big( Z_1,\dots,Z_n,(x,y) \big) \delta_{R_i^{(x,y)}}
+ p^\lambda_{n+1}\big( Z_1,\dots,Z_n,(x,y) \big) \delta_\infty \bigg) \bigg\},  
\end{equation}
where \smash{$R_i^{(x,y)}$}, $i=1,\ldots,n+1$ are conformity scores as in
\eqref{eq:scores}, for an arbitrary score function $V$ that is symmetric in its 
last $n+1$ arguments, and
\begin{equation}
\label{eq:pw_gen}
p^\lambda_i(z_1,\dots,z_{n+1}) = 
\frac{\sum_{\sigma : \sigma(n+1)=i} \lambda(z_{\sigma(1)},\dots,
  z_{\sigma(n+1)})} {\sum_\sigma \lambda(z_{\sigma(1)},\dots, z_{\sigma(n+1)})},
\quad i=1,\dots,n+1.  
\end{equation}
Then \smash{$\hC^\lambda_n$} satisfies 
\begin{equation}
\label{eq:weighted_cov_gen}
\P\big( Y_{n+1} \in \hC^\lambda_n(X_{n+1}) \big) \geq 1-\alpha. 
\end{equation}
\end{theorem}

Computation of the weights in \eqref{eq:pw_gen} is now even more difficult than
\eqref{eq:pw} in the weighted exchangeable setting (even more difficult than a
matrix permanent, since $\lambda$ could in principle depend in a complicated way
on the order of its inputs). That said, the above theorem still produces a
conformal set \eqref{eq:weighted_conf_gen} with the very general guarantee
\eqref{eq:weighted_cov_gen}, which is interesting. This may be useful (and
computable) in certain structured-X settings, for example, where the sequence
$X_1,\dots,X_{n+1}$ has some kind of Markov structure.

\section{Custom-weighted conformal prediction}

\def\tw{\tilde{w}}
\def\dtv{\mathsf{d}_{\mathsf{TV}}}

We next cover a custom-weight conformal prediction method, due to
\citet{barber2022conformal}. In comparison to the likelihood-weighted method
in the previous section, the weights considered in the current section will be
\emph{fixed} (not a function of the data) but \emph{arbitrary}. The theory, as
we'll see, is also quite different; in a sense, it is more general in
scope. We'll only cover this at a relatively high level (no proof details). 

To state the main theorems, we'll need introduce a few additional pieces of  
notation. As before, let $Z_i=(X_i,Y_i)$, $i=1,\dots,n+1$ be data points (with
the last one $Z_{n+1}=(X_{n+1},Y_{n+1})$ serving as the test point), and $V$ 
a score function. The additional notation is as follows.   

\begin{itemize}
\item Denote by $Z = (Z_1,\dots,Z_{n+1})$ the data vector (an ordered
  sequence). 
\item Denote by $Z^i$ the data vector after swapping components $i$ and
  $n+1$. 
\item Denote by $R(Z)$ the score vector, with components $R(Z)_j = V(Z_j; Z)$. 
\item Denote by $R(Z^i)$ the score vector had the data vector been $Z^i$, with
  components \smash{$R(Z^i)_j = V(Z^i_j; Z^i)$}.     
\end{itemize}

\subsection{Custom weights, symmetric score function}

Note that in the current notation, symmetry of $V$ in its last $n+1$
arguments---which is the typical assumption for the conformity score 
function---implies that we have $V(Z_j; Z) = V(Z_j; Z_\sigma)$, for any
permutation $\sigma$, where \smash{$Z_\sigma = (Z_{\sigma(1)}, \dots,
  Z_{\sigma(n+1)})$}. Thus under this symmetry condition, we may write $R(Z^i) =
R(Z)^i$.   
% \[
% R(Z^i)_j = 
% \begin{cases}
% R(Z)_j & j \not= i,n+1 \\
% R(Z)_i & j = n+1 \\
% R(Z)_{n+1} & j = i.
% \end{cases}
% \]

Now we can state the first main result, for custom-weighted conformal prediction.

\begin{theorem}
\label{thm:weighted_conf_nex}
Let $w_i \in [0,1]$, $i=1,\dots,n$ be fixed and arbitrary weights, and define 
\begin{equation}
\label{eq:tw}
\tw_i = \frac{w_i}{w_1 + \cdots + w_n + 1}, \; i=1,\ldots,n, 
\quad \text{and} \quad 
\tw_{n+1} = \frac{1}{w_1 + \cdots + w_n + 1}.
\end{equation}
Define a weighted conformal set (based on the first $n$ samples) at a point
$x \in \cX$, with nominal error level $\alpha \in (0,1)$, by   
\begin{equation}
\label{eq:weighted_conf_nex}
\hC^w_n(x) = \bigg\{ y : R_{n+1}^{(x,y)} \leq \Quantile\bigg( 1-\alpha; 
\, \sum_{i=1}^n \tw_i \, \delta_{R_i^{(x,y)}} + \tw_{n+1} \, \delta_\infty
\bigg) \bigg\},   
\end{equation}
where \smash{$R_i^{(x,y)}$}, $i=1,\ldots,n+1$ are conformity scores as in
\eqref{eq:scores}, for an arbitrary score function $V$ that is symmetric in its
last $n+1$ arguments. Then with \emph{no} assumptions on the joint distribution
of $Z_i=(X_i,Y_i)$, $i=1,\dots,n+1$, the set \smash{$\hC^w_n$} satisfies
\begin{equation}
\label{eq:weighted_cov_nex}
\P\big( Y_{n+1} \in \hC^w_n(X_{n+1}) \big) \geq 1-\alpha - \sum_{i=1}^n 
\tw_i \cdot \dtv(R(Z), R(Z^i)),
\end{equation}
where $\dtv(A,B)$ is the total variation (TV) distance between the distributions
of random variables $A,B$.  
\end{theorem}

Observe that the normalization step in \eqref{eq:tw} assigns a unit weight to 
the test point, and then renormalizes (so that the new weights have unit
sum). As the initial weights were all between 0 and 1, this means
\smash{$\tw_{n+1} \geq \tw_i$}, for all $i=1,\dots,n$. The set in
\eqref{eq:weighted_conf_nex} reduces to the (unweighted) conformal prediction
method set when $w_1 = \cdots = w_n = 1$.  

We can interpret the result in \eqref{eq:weighted_cov_nex} as follows. If the
distribution of the test data point $Z_{n+1}$ drifts from that of the training
data, but we are able identify a priori which training data points will be most
\emph{representative} of the test distribution, then we can upweight these
points and downweight the others. This would result in a small coverage gap, 
where  
\begin{equation}
\label{eq:coverage_gap}
\text{coverage gap} = \sum_{i=1}^n \tw_i \cdot \dtv(R(Z), R(Z^i))
\end{equation}
(since we have small weights multiplying large TV distances, and large weights 
multiplying small TV distances.) Of course, choosing a good weights
scheme---identifying a priori which training points are representative of the
test data distribution---is an important problem unto itself. In certain
structured data settings, such as problems with time series or spatial
structure, progress towards general methodology for crafting weights seems more
tangible than others. Figure \ref{fig:electricity} gives an example using an
exponentially decaying weight scheme in a time series problem, from
\citet{barber2022conformal}.  

\begin{figure}[htb]
\centering
\includegraphics[width=0.925\textwidth]{electricity.pdf}
\caption{\it Experiments for split conformal prediction under distribution
  drift. Top row: coverage and width in a time series problem, for the usual
  conformal prediction (CP), weighted conformal with exponentially decaying
  weights and least squares as the prediction algorithm (NexCP+LS), and the same
  method but now with weighted least squares as the prediction algorithm 
  (NexCP+WLS). Bottom row: the same metrics and methods on a permuted version of 
  the time series data set. Credit: \citet{barber2022conformal}.}        
\label{fig:electricity}
\end{figure}

We now make several further remarks. 

\paragraph{Split version.}

As before, the split conformal version of the above result can be viewed as a
special case where the score function relies on a point predictor fit on an
external data set. For example, if we take it to be $V(x,y) = |y-\mu_0(x)|$,
where $\mu_0$ has been fit on a data set $Z_0$ then \eqref{eq:weighted_conf_nex}
simplifies to            
\[
\hC^w_n(x) = \mu_0(x) \pm \Quantile\bigg(1-\alpha; \, \sum_{i=1}^n \tw_i \,
\delta_{|Y_i - \mu_0(X_i)|} + \tw_{n+1} \, \delta_\infty \bigg),
\]
and \eqref{eq:weighted_cov_nex} becomes
\[
\P\big( Y_{n+1} \in \hC^w_n(X_{n+1}) \,\big|\, Z_0 \big) \geq 1-\alpha -
\sum_{i=1}^n \tw_i \cdot \dtv\big( R(Z), R(Z^i) \,\big|\, Z_0 \big), 
\]
where $\dtv(A, B \,|\, C)$ is the TV distance between the conditional
distributions of $A|C$ and $B|C$. If $Z_0$ is independent of $Z$, then the
coverage gap is just a before, in \eqref{eq:coverage_gap}.

\paragraph{The i.i.d.\ setting.}

When $Z_i=(X_i,Y_i)$, $i=1,\dots,n+1$ are i.i.d.\ (or more generally,
exchangeable), we are back to the traditional setting for conformal
prediction. In this case, there is no slack in \eqref{eq:weighted_cov_nex},
since exchangeability implies \smash{$R(Z) \overset{d}{=} R(Z^i)$} and thus
\smash{$\dtv(R(Z), R(Z^i)) = 0$} for each $i$, and hence
\eqref{eq:weighted_cov_nex} collapses to an exact $1-\alpha$ coverage
guarantee. This not only reproduces the standard result for ordinary conformal,
when we take $w_1 = \cdots = w_n = 1$, but also shows us something  
new: in the i.i.d.\ (or exchangeable) setting, we can use arbitrary weights and
still get exact coverage with weighted conformal prediction.

\paragraph{Coverage gap bounds.}

It is worth noting a few upper bounds for the coverage gap in
\eqref{eq:coverage_gap}. First, 
\[
\text{coverage gap} \leq \sum_{i=1}^n \tw_i \cdot \dtv(Z, Z^i),
\]
since the TV distance between $f(A),f(B)$ is always less than that between
$A,B$. The bound in the above display is easier to interpret, but can also be
much larger than that in \eqref{eq:coverage_gap}. (For example, think about the
case of a high-dimensional feature space, and a score function that ignores all
but a few relevant features for prediction.) Second, if $Z_i$, $i=1,\dots,n+1$
are independent (but not identically distributed), then it can be shown that 
\[
\text{coverage gap} \leq 2\sum_{i=1}^n \tw_i \cdot \dtv(Z_i, Z_{n+1}).
\]
This lends more direct intuition to the idea that we want to assign large
weights to training points whose distributions are representative of the test 
distribution. 

\paragraph{CDF form.}

The analogous CDF form of the conformal set in \eqref{eq:weighted_conf_nex} is
as follows: 
\begin{equation}
\label{eq:weighted_cdf_nex}
\hC^w_n(x) = \bigg\{ y : \sum_{i=1}^{n+1} \tw_i \, 1\Big\{ R_i^{(x,y)} \leq
R^{(x,y)}_{n+1} \Big\} \leq \lceil 1-\alpha \rceil_w \bigg\},
\end{equation}
where \smash{$\lceil 1-\alpha \rceil_w = \min\{ \tau \in \mathrm{range}(\hF^w_n)
  : \tau \geq \beta\}$} and \smash{$\hF^w_n$} denotes the (random) CDF of the
discrete distribution \smash{$\sum_{i=1}^n \tw_i \, \delta_{R^{(x,y)}_i} +
  \tw_{n+1} \, \delta_\infty$}. We note that, as before (for the
likelihood-weighted case), the adjustment of the probability level needed here
is random.  

\paragraph{Auxiliary randomization for exact coverage.}

We can randomize the CDF form in \eqref{eq:weighted_cdf_nex} as follows: 
\[
\hC_n^{w,*}(x) = \bigg\{ y : \sum_{i=1}^n \tw_i \, 1\Big\{ R_i^{(x,y)} <
R^{(x,y)}_{n+1} \Big\} + U \sum_{i=1}^{n+1} \tw_i \, 1\Big\{ R_i^{(x,y)} =
R^{(x,y)}_{n+1} \Big\} \leq 1-\alpha \bigg\},
\]
where $U \sim \mathrm{Unif}(0,1)$, independent of everything else. This is
fairly simple and intuitive---it is free of any level adjustments needed in the   
unrandomized CDF-based set in \eqref{eq:weighted_cdf_nex}. The set
\smash{$\hC^{w,*}_n$} satisfies
\[
\P\big( Y_{n+1} \in \hC^{w,*}_n(X_{n+1}) \big) = 1-\alpha.  
\]
We can also randomize the quantile form in \eqref{eq:weighted_conf_nex} in order
to obtain exact coverage (similar to what we did in the likelihood-weighted
case) but we omit the details. 

\subsection{Custom weights, nonsymmetric score function}

\section{Adaptive conformal inference}

Lastly we cover a conformal-like method, for sequential prediction problems, due 
to \citet{gibbs2021adaptive}. In a way, this is a signifcant departure from the
methods we've seen thus far, since the base idea isn't specific to conformal
prediction at all. Its core guarantee is quite simple to prove, yet at the same
time, quite strong. We'll only cover this material at a high level, and will
skip a lot of the details.

\bibliographystyle{plainnat}
\bibliography{../../common/ryantibs}

\end{document}
