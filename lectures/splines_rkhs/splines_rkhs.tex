\documentclass{article}

\def\ParSkip{} 
% Packages
\usepackage{amssymb,amsmath,amsthm,bbm}
\usepackage{verbatim,float,url,dsfont}
\usepackage{graphicx,subfigure,psfrag}
\usepackage{algorithm,algorithmic}
\usepackage{mathtools,enumitem}
\usepackage{multirow}
\usepackage{ragged2e}
\usepackage{xr-hyper}

\usepackage[colorlinks=true,citecolor=blue,urlcolor=blue,linkcolor=blue]{hyperref}
\usepackage[margin=1in]{geometry}
\usepackage[round]{natbib}

\usepackage[utf8]{inputenc} % allow utf-8 input
\usepackage[T1]{fontenc}    % use 8-bit T1 fonts
\usepackage{booktabs}       % professional-quality tables
\usepackage{nicefrac}         % compact symbols for 1/2, etc.
\usepackage{microtype}      % microtypography

\ifdefined\TimesFont 
\usepackage{times} % use times font
\fi

\ifdefined\ParSkip 
\usepackage{parskip} % use par skip
\fi

% Theorems and such
\newtheorem{theorem}{Theorem}
\newtheorem{lemma}{Lemma}
\newtheorem{corollary}{Corollary}
\newtheorem{proposition}{Proposition}
\theoremstyle{definition}
\newtheorem{remark}{Remark}
\newtheorem{definition}{Definition}

% Assumption
\newtheorem*{assumption*}{\assumptionnumber}
\providecommand{\assumptionnumber}{}
\makeatletter
\newenvironment{assumption}[2]{
  \renewcommand{\assumptionnumber}{Assumption #1#2}
  \begin{assumption*}
  \protected@edef\@currentlabel{#1#2}}
{\end{assumption*}}
\makeatother

% Widebar
\makeatletter
\newcommand*\rel@kern[1]{\kern#1\dimexpr\macc@kerna}
\newcommand*\widebar[1]{%
  \begingroup
  \def\mathaccent##1##2{%
    \rel@kern{0.8}%
    \overline{\rel@kern{-0.8}\macc@nucleus\rel@kern{0.2}}%
    \rel@kern{-0.2}%
  }%
  \macc@depth\@ne
  \let\math@bgroup\@empty \let\math@egroup\macc@set@skewchar
  \mathsurround\z@ \frozen@everymath{\mathgroup\macc@group\relax}%
  \macc@set@skewchar\relax
  \let\mathaccentV\macc@nested@a
  \macc@nested@a\relax111{#1}%
  \endgroup
}
\makeatother

% Min and max
\newcommand{\argmin}{\mathop{\mathrm{argmin}}}
\newcommand{\argmax}{\mathop{\mathrm{argmax}}}
\newcommand{\minimize}{\mathop{\mathrm{minimize}}}
\newcommand{\maximize}{\mathop{\mathrm{maximize}}}
\newcommand{\st}{\mathop{\mathrm{subject\,\,to}}}

% Shortcuts
\def\R{\mathbb{R}}
\def\C{\mathbb{C}}
\def\Z{\mathbb{Z}}
\def\N{\mathbb{N}}
\def\E{\mathbb{E}}
\def\P{\mathbb{P}}
\def\T{\mathsf{T}}
\def\Cov{\mathrm{Cov}}
\def\Var{\mathrm{Var}}
\def\indep{\perp\!\!\!\perp}
\def\th{^{\text{th}}}
\def\tr{\mathrm{tr}}
\def\df{\mathrm{df}}
\def\dim{\mathrm{dim}}
\def\col{\mathrm{col}}
\def\row{\mathrm{row}}
\def\nul{\mathrm{null}}
\def\rank{\mathrm{rank}}
\def\nuli{\mathrm{nullity}}
\def\spa{\mathrm{span}}
\def\sign{\mathrm{sign}}
\def\supp{\mathrm{supp}}
\def\diag{\mathrm{diag}}
\def\aff{\mathrm{aff}}
\def\conv{\mathrm{conv}}
\def\dom{\mathrm{dom}}
\def\hy{\hat{y}}
\def\hf{\hat{f}}
\def\hmu{\hat{\mu}}
\def\halpha{\hat{\alpha}}
\def\hbeta{\hat{\beta}}
\def\htheta{\hat{\theta}}
\def\cA{\mathcal{A}}
\def\cB{\mathcal{B}}
\def\cD{\mathcal{D}}
\def\cE{\mathcal{E}}
\def\cF{\mathcal{F}}
\def\cG{\mathcal{G}}
\def\cK{\mathcal{K}}
\def\cH{\mathcal{H}}
\def\cI{\mathcal{I}}
\def\cL{\mathcal{L}}
\def\cM{\mathcal{M}}
\def\cN{\mathcal{N}}
\def\cP{\mathcal{P}}
\def\cS{\mathcal{S}}
\def\cT{\mathcal{T}}
\def\cW{\mathcal{W}}
\def\cX{\mathcal{X}}
\def\cY{\mathcal{Y}}
\def\cZ{\mathcal{Z}}

\DeclareMathOperator*{\esssup}{ess\,sup}

\title{Nonparametric Regression: Splines and RKHS Methods \\ \smallskip
\large Advanced Topics in Statistical Learning, Spring 2023 \\ \smallskip
Ryan Tibshirani}
\author{}
\date{}

\begin{document}
\maketitle
\RaggedRight
\vspace{-50pt}

Note: we're following the context, problem setup, notation, etc. from the last
lecture on nonparametric regression.

\section{Regression splines}

Regression splines and smoothing splines are motivated from a different
perspective than kernels and local polynomials; in the latter case, we started
off with a special kind of local averaging, and moved our way up to a
higher-order local models. With regression splines and smoothing splines, we
build our estimator globally, from a set of select basis functions. 

(We note that, at a broader level, the latter is often called the
\emph{synthesis} framework for modeling, where we build our estimator from a  
set of atoms---here being basis functions.) 

\subsection{Splines}

These basis functions, as you might guess, are \emph{splines}. Let's assume that
$d=1$. We'll stick to the univariate case for a little while, because splines
are complex and interesting enough in dimension $d=1$. A spline $f$ of degree
$k$ with knots at $t_1 < \dots < t_r$ is a piecewise polynomial of degree $k$  
that is continuous and has continuous derivatives of orders $1,\dots,k-1$ at its
knots. To be clear: 
\begin{itemize}
\item $f$ is a polynomial of degree $k$ on each of $(-\infty,t_1], [t_1,t_2],
  \dots, [t_r,\infty)$; and
\item $D^\ell f$ is continuous at each of $t_1,\dots,t_r$, for all
  $\ell=0,\dots,k-1$. 
\end{itemize}

Splines have some special (some might say amazing) properties, and we will only
really scratch the surface here. They have been a topic of interest among
mathematicians and statisticians for a long time. Informally, a spline is a lot
smoother than a piecewise polynomial, and so modeling with splines can serve as
a way of reducing the variance of an estimator. See Figure \ref{fig:splines} for
an illustration.  

\begin{figure}[tb]
\centering
\includegraphics[width=0.8\textwidth]{splines.pdf}
\caption{\it Illustration of the effects of enforcing continuity at the knots
  for a cubic piecewise polynomial, and for various orders of the derivative.
  Credit: Chapter 5.2 of \citet{hastie2009elements}.}    
\label{fig:splines}
\end{figure}

A bit of statistical folklore: it is said that a cubic spline is so smooth, that
one cannot detect the locations of its knots by eye! 

\subsection{Spline bases}

How can we parametrize the space of $k\th$ degree splines with knots at 
$t_1,\dots,t_r$?  The most natural way is to use the \emph{truncated power
  basis}, $g_1, \dots, g_{r+k+1}$, defined as   
\begin{equation}
\begin{gathered}
\label{eq:tpb}
g_j(x) = \frac{1}{(j-1)!} x^{j-1}, \quad j=1,\dots,k+1, \\
g_{j+k+1}(x) = \frac{1}{k!} (x-t_j)^k_+, \quad j=1,\dots,r,
\end{gathered}
\end{equation}
Here $x_+=\max\{x,0\}$ denotes the positive part of $x$. From this we can see
that the linear space of $k\th$ degree splines with knots at $t_1,\dots,t_r$ has 
dimension $r+k+1$.

While \eqref{eq:tpb} gives us a simple and natural basis, calculations involving
truncated power bases functions can be troublesome because the ensuing basis 
matrix, $G \in \R^{n \times (r+k+1)}$ with entries
\[
G_{ij} = g_j(x_i)
\]
is generally very poorly conditioned.

A much better computational choice, both for speed and numerical stability, is 
called the \emph{B-spline basis}. This was a major development in spline theory 
and is now much the standard in software. The key idea is that B-splines have
local support: each B-spline basis function for a $k\th$ degree spline space is
supported on $k+2$ knots. Therefore the corresponding basis matrix is banded. It
also tends to be much better conditioned.  

Defining B-splines is certainly requires more nuance than the truncated power
basis, and the precise form of a B-spline is unimportant for the rest of this
lecture. For completeness, you can find the details behind their construction in
somes supplementary notes posted on the course website.

\subsection{Regress away!}

A first idea: let's just perform regression on a spline basis. In other words,
we use as our working model a $k\th$ degree spline with knots at some pre-fixed 
locations $t_1,\dots,t_r$. This means expressing $f$ as  
\[
f = \sum_{j=1}^{r+k+1} \beta_j g_j
\]
where $\beta_1,\dots,\beta_{r+k+1} \in \R$ are coefficients and
$g_1,\dots,g_{r+k+1}$ is a basis for the space of $k\th$ degree splines over the
knots $t_1,\dots,t_r$; for example, the truncated power basis or B-spline basis.    

Letting $Y=(y_1,\dots,y_n) \in \R^n$ be the response vector, and $G \in \R^{n
  \times (r+k+1)}$ the basis matrix with entries   
\[
G_{ij} = g_j(x_i),
\]
as before, we then just use least squares to determine estimate the
coefficients, defining \smash{$\hbeta = (\hbeta_1,\dots,\hbeta_{r+k+1})$} to 
solve 
\[
\minimize_\beta \; \sum_{i=1}^n \bigg(y_i -\sum_{j=1}^{r+k+1} \beta_j g_j(x_i) 
\bigg)^2 \iff \minimize_\beta \; \|Y - G\beta\|_2^2.
\]
This yields the \emph{regression spline} estimator, which makes predictions
according to   
\[
\hf(x) = \sum_{j=1}^{r+k+1} \hbeta_j g_j(x).
\]
Of course we know that \smash{$\hbeta = (G^\T G)^{-1} G^\T Y$}, so we can write
this as 
\[
\hf(x) = (g_1(x),\dots,g_{r+k+1}(x))^\T (G^\T G)^{-1} G^\T Y = w(x)^\T Y,  
\]
which reveals that regression splines are linear smoothers.

This is a classic method, and can work well provided that we choose ``good''
knots $t_1,\dots,t_r$; but in general choosing knots is a tricky business. There
is a large literature on knot selection for regression splines via greedy
methods like recursive partitioning. In practice, smoothing splines seem to be
more popular, which we cover next.

\section{Smoothing splines}

Before delving into smoothing splines, we need to introduce a variant on the
usual spline definition given above. To motivate it: a problem with spline
estimates is that they can have somewhat erractive behavior---translating into
high variance---at the boundaries of the input domain. (Recall that this is the
opposite problem to that with kernel smoothing, which had poor bias at the
boundaries.) This only gets worse as the polynomial order $k$ gets larger.   

\subsection{Natural splines}

One way to remedy this problem is to force the piecewise polynomial function to
have a lower degree to the left of the leftmost knot, and to the right of the
rightmost knot---this is exactly what \emph{natural splines} do. A natural
spline $f$ of degree $k$ with knots at $t_1 < \dots < t_r$ is a piecewise
polynomial of degree $k$ such that:
\begin{itemize}
\item $f$ is a polynomial of degree $k$ on each of $[t_1,t_2], \dots,
  [t_{r-1},t_r]$; 
\item $f$ is a polynomial of degree $(k-1)/2$ on $(-\infty,t_1]$ and
  $[t_r,\infty)$; and 
\item $D^\ell f$ is continuous at each of $t_1,\dots,t_r$, for all
  $\ell=0,\dots,k-1$. 
\end{itemize}
It is implicit here that natural splines are only defined for an odd degree $k$
(linear, cubic, etc.) The choice $k=3$ yields a natural cubic spline, by far the
most common case: this is just a cubic spline that reduces to linear beyond the 
leftmost and rightmost knots.     

What is the dimension of the span of $k\th$ degree natural splines with knots at
$t_1,\dots,t_r$? Recall for splines, this was $r+k+1$ (just count the number of
truncated power basis functions). For natural splines, we can compute this
dimension by counting as follows:  
\[
\underbrace{\vphantom{\Big(\Big)} (k+1)\cdot(r-1)}_{a} \;+\; 
\underbrace{\Big(\frac{(k-1)}{2}+1\Big) \cdot 2}_{b} \;-\;
\underbrace{\vphantom{\Big(\Big)} k \cdot r}_{c} \;=\; r. 
\]
In the above: 
\begin{itemize}
\item $a$ is the number of parameters in the interior intervals $[t_1,t_2],
  \dots, [t_{r-1},t_r]$; 
\item $b$ is the number of parameters in the exterior intervals $(-\infty,t_1],
  [t_r,\infty)$; and 
\item $c$ is the number of constraints at the knots $t_1,\dots,t_r$.  
\end{itemize}
The fact that the total dimension is $r$ is pretty remarkable; this is
independent of $k$!   

We note that there are simple modifications the truncated power basis that gives
rise to a basis for natural splines, and similarly a modification of the
B-spline basis for natural splines. And again, B-splines are the preferred
parametrization for computational speed and stability. 

\subsection{Smooth away!}

Smoothing splines, at the end of the day, are given by an $\ell_2$-regularized
regression over a natural spline basis that places knots at all input points 
$x_1,\dots,x_n$. They circumvent the problem of knot selection because they
just use all inputs as knots, and they make this possible---starting with a 
saturated model and producing meaningful function estimates---by using
regularization to shrink the coefficients in the basis expansion.

Interestingly, we can motivate and define a smoothing spline directly from a
functional minimization perspective. With input points $x_1,\dots,x_n$ lying in 
an interval $[a,b]$, the \emph{smoothing spline} estimator \smash{$\hf$}, of a
given odd integer order $k \geq 1$, is defined to solve   
\begin{equation}
\label{eq:ss}
\minimize_f \; \sum_{i=1}^n (y_i - f(x_i))^2 + \lambda \int_a^b [D^m f(x)]^2 \,
dx, \quad \text{where $m=(k+1)/2$}.
\end{equation}
This is an infinite-dimensional optimization problem over all functions $f$ for
the which the criterion is well-defined and finite. The criterion in
\eqref{eq:ss} trades off the squared error of $f$ over $(x_i,y_i)$,
$i=1,\dots,n$, with a penalty term that is large when the $(m-1)^{\text{st}}$
derivative of $f$ is wiggly. The tuning parameter $\lambda \geq 0$ governs the
tradeoff between these two terms.  

By far the most commonly considered case is $k=3$, the cubic smoothing spline,
defined via 
\begin{equation}
\label{eq:ss_cubic}
\minimize_f \; \sum_{i=1}^n (y_i - f(x_i))^2 + \lambda \int_a^b [f''(x)]^2 \, dx.
\end{equation}
This is so common that often you'll just hear this called \emph{the} smoothing
spline. 

\subsection{Representer theorem}

Remarkably, it so happens that the minimizer in the general smoothing spline
problem \eqref{eq:ss} is unique, and it is a natural $k\th$ degree spline with 
knots at the input points $x_1,\dots,x_n$! This is known as a \emph{representer
  theorem} for \eqref{eq:ss} (we will see more such results later). 

Proof: we'll just consider the cubic case, $k=3$, and follow Chapter 2.2 of
\citet{green1993nonparametric} who give a nice direct proof.   
% The result for general $k$ will follow from RKHS theory. 

The key result can be stated as follows: if $g$ is any function on $[a,b]$ such
that the penalty is well-defined (it has two derivatives, its second derivative
is square integrable), and $x_1,\dots,x_n \in [a,b]$ are arbitrary, then there
exists a natural cubic spline $f$ with knots at $x_1,\dots,x_n$ such that: 
\begin{itemize}
\item $f(x_i) = g(x_i)$, $i=1,\dots,n$; and 
\item $\int_a^b [f''(x)]^2 \, dx \leq \int_a^b [g''(x)]^2 \, dx$.
\end{itemize}

This would imply that we can restrict our attention in problem
\eqref{eq:ss_cubic} to natural splines with knots at $x_1,\dots,x_n$. 

To prove
the key result, we start with the fact that the cubic natural spline space with
knots at $x_1,\dots,x_n$ is $n$-dimensional, so given any $n$ points $z_i =
g(x_i)$, $i=1,\dots,n$, we can always find a natural spline $f$ with knots at
$x_1,\dots,x_n$ that satisfies $f(x_i) = z_i$, $i=1,\dots,n$. Now define $h =
g-f$. Consider 
\begin{align*}
\int_a^b f''(x) h''(x) \, dx 
&= f''(x) h'(x) \Big|_a^b - \int_a^b f'''(x) h'(x) \, dx \\
&= -\int_{x_1}^{x_n} f'''(x) h'(x) \, dx \\ 
&= -\sum_{i=1}^{n-1} f'''(x) h(x) \Big|_{x_i}^{x_{i+1}} +  
\int_{x_1}^{x_n} D^4 f(x) h(x) \, dx \\
&= -\sum_{i=1}^{n-1} f'''(x_i^+) (h(x_{i+1}) - h(x_i)).
\end{align*}
In the first line we used integration by parts; in the second line we used the
fact that $f''(a) = f''(b) = 0$ and $f'''(x)=0$ for $x \leq x_1$ and $x \geq
x_n$, since $f$ is a natural spline; in the third line we used integration by
parts again; in the fourth we used the fact that $f'''$ is constant on each open
interval $(x_i,x_{i+1})$, and that $D^4 f=0$, again because $f$ is a natural
spline. Since each $h(x_i)=0$, we conclude from the last display that
\[
\int_a^b f''(x) h''(x) \, dx = 0.
\]
From this, it follows that
\[
\int_a^b [g''(x)]^2 \, dx 
= \int_a^b \big[ f''(x) + h''(x) \big]^2 \, dx 
= \int_a^b [f''(x)]^2 \, dx + \int_a^b [h''(x)]^2 \, dx
\]
since the cross term is zero, and therefore 
\[
\int_a^b [f''(x)]^2 \, dx \leq \int_a^b [g''(x)]^2 \, dx,
\]
with equality if and only if $h''(x)=0$ for all $x \in [a,b]$. Note that $h''=0$
implies that $h$ must be linear, and since we already know that $h(x_i)=0$ for
all $i=1,\dots,n$, this is equivalent to $h=0$. In other words, the last display
holds strictly except when $g=f$, so the solution in \eqref{eq:ss_cubic} is
uniquely a natural spline with knots at the inputs. 

\subsection{Finite-dimensional form}

From the representer result, we can choose a basis $\eta_1,\dots,\eta_n$ for the
set of $k\th$ degree natural splines with knots at $x_1,\dots,x_n$, and
reparametrize the problem \eqref{eq:ss} as  
\begin{equation}
\label{eq:ss_finite_dim1}
\minimize_\beta \; \sum_{i=1}^n \bigg( y_i - \sum_{j=1}^n \beta_j \eta_j(x_i)
\bigg)^2 + \lambda \int_a^b \bigg( \sum_{j=1}^n \beta_j D^m \eta_j(x) \bigg)^2
\, dx.  
\end{equation}
This is a finite-dimensional problem, and after we solve for the coefficients
\smash{$\hbeta \in \R^n$}, the smoothing spline estimator is simply given by  
\[
\hf(x) = \sum_{j=1}^n \hbeta_j \eta_j(x).
\]
Defining the basis matrix $N \in \R^{n \times n}$ and penalty matrix $\Omega \in
\R^{n \times n}$ to have entries
\[
N_{ij} = \eta_j(x_i), \quad 
\Omega_{ij} = \int_a^b D^m \eta_i(x) D^m \eta_j(x) \, dx,
\]
the problem in \eqref{eq:ss_finite_dim1} can be written more succintly as 
\begin{equation}
\label{eq:ss_finite_dim2}
\minimize_\beta \; \|Y - N\beta\|_2^2 + \lambda \beta^\T \Omega \beta,   
\end{equation}
which a type of generalized ridge regression problem. From
\eqref{eq:ss_finite_dim2}, it is not hard to see that the solution has the
explicit form \smash{$\hbeta = (N^\T N + \lambda \Omega)^{-1} N^\T Y$}, and
thus, we can write the smoothing spline as    
\[
\hf(x) = (\eta_1(x),\dots,\eta_n(x))^\T (N^\T N + \lambda \Omega)^{-1} N^\T Y =
w(x)^\T Y,   
\]
which means, once again, smoothing splines are linear smoothers.

A remark on computation: the coefficients \smash{$\hbeta = (N^\T N + \lambda 
  \Omega)^{-1} N^\T Y$} can be computed in $O(n)$ operations. For this, we form  
$N$ using the B-spline basis (for natural splines), since then the matrix $N^\T
N + \Omega I$ will be banded. In fact, more specialized computations are
possible by taking advantage of more precise structure (beyond bandedness)
afforded by splines. Altogether, in practice, smoothing spline computations are 
extremely fast. 

\subsection{Reinsch form}

The vector of fitted values \smash{$\hat{Y} = (\hf(x_1),\dots,\hf(x_n)) \in
  \R^n$}  from the smoothing spline can be written as
\[
\hat{Y} = N(N^\T N + \lambda \Omega)^{-1} N^\T Y.
\]
It is informative to rewrite this in what is called Reinsch form,
\begin{align*}
\hat{Y} 
&= N\Big(N^\T (I + \lambda N^{-\T} \Omega N^{-1}) N \Big)^{-1} N^\T Y \\
&= (I+\lambda Q)^{-1} Y,
\end{align*}
where $Q = N^{-\T} \Omega N^{-1}$. Note that $Q$ does not depend on $\lambda$.
Thus if we compute an eigendecomposition $Q = U \Sigma U^\T$, then the
eigendecomposition of the smoother matrix \smash{$S = (I+\lambda Q)^{-1}$} is   
\[
S = \sum_{j=1}^n \frac{1}{1+\lambda \sigma_j} u_ju_j^\T,
\]
where $\Sigma=\mathrm{diag}(\sigma_1,\dots,\sigma_n)$. Therefore, once more, the
smoothing spline fitted values can be expressed as  
\[
\hat{Y} = \sum_{j=1}^n \frac{u_j^\T Y}{1+\lambda \sigma_j} u_j.
\]
The interpretation we may gather is as follows: smoothing splines perform a 
regression on the orthonormal set $u_1,\dots,u_n \in \R^n$, but they shrink
the coefficients, with more shrinkage applied to an eigenvector $u_j$ that
corresponds to a large eigenvalue $\sigma_j$.  

What exactly are these vectors $u_1,\dots,u_n$? These are referred to as the 
\emph{Demmler-Reinsch basis}, and some nice properties of theirs can be 
worked out analytically \citep{demmler1975oscillation}. In a nutshell: the
eigenvectors $u_j$ that correspond to smaller eigenvalues $\sigma_j$ are
smoother, and so with smoothing splines, we shrink less in their direction. Or
to put it differently, by increasing $\lambda$ in the smoothing spline
estimator, we are tuning out the more wiggly components. See Figure
\ref{fig:reinsch}. 

\begin{figure}[tbp]
\centering
\includegraphics[width=0.495\textwidth]{reinsch_basis.pdf}
\includegraphics[width=0.495\textwidth]{reinsch_weight.pdf}
\caption{\it Insights from the Reinsch form of the cubic smoothing spline, for
  an example with $n=50$ input points. Left: the bottom 8 eigenvectors of $Q$,
  drawn using a color gradient where purple means the smallest eigenvalue, and
  light blue the largest eigenvalue. We see that smoother eigenvectors
  correspond to lower eigenvalues. Right: the shrinkage weights
  $w_j=1/(1+\lambda \sigma_j)$, $j=1,\dots,n$ for 8 values of $\lambda$, again
  drawn using a color gradient where purple means the largest $\lambda$, and
  light blue the smallest $\lambda$. For larger $\lambda$, the weights decay
  faster, so the smoothing spline estimator more strongly filters out wiggly
  components.}         
\label{fig:reinsch}
\end{figure}

\subsection{Equivalent kernel}

Recall that we can write a smoothing spline prediction as \smash{$\hf(x) = 
  w(x)^\T Y$}, for a weight function  $w(x) = (w_1(x), \dots, w_n(x)) \in \R^n$.
We know the analytic form of this weight function, but how about its qualitative
behavior? To be more precise, if we denote each component function by $w_i(x) =
w(x, x_i)$ to emphasize that this weight gets attributed to $(x_i, y_i)$ in the
weighted sum \smash{$w(x)^\T Y = \sum_{i=1}^n w(x, x_i) y_i$} which gives us the
smoothing spline prediction at $x$, then we can ask the following questions:         
\begin{itemize}
\item What shape does $z \mapsto w(x, z)$ have? Is it kernel-like?
\item Does this shape change as we vary the test point $x$? 
\end{itemize}

It's easy to just read off the answers to these questions from the rows of the
smoother matrix $S = N(N^\T N + \lambda \Omega)^{-1} N^\T$, since in our
expanded notation, its elements are
\[
S_{ij} = w_j(x_i) = w(x_i, x_j).
\]
Now, something very interesting happens when we plot a few rows of $S$. For
evenly-spaced inputs, they look like the translations of the same kernel: see
the left panel of Figure \ref{fig:ss_kernel}. Even more interestingly, for
unevenly-spaced inputs, the rows still have a kernel shape but now the bandwidth
appears to adapt to the density of the input points: lower density, larger
bandwidth. See the right panel of Figure \ref{fig:ss_kernel}.

\begin{figure}[tb]
\centering
\includegraphics[width=0.495\textwidth]{ss_kernel1.pdf}
\includegraphics[width=0.495\textwidth]{ss_kernel2.pdf}
\caption{\it Three sample rows of the cubic smoothing spline operator $S$
  defined over $n=500$. Left: evenly-spaced inputs on a grid. The weights look
  like they are given by translations of the same kernel. Right: inputs drawn
  i.i.d.\ from $N(0.5, 0.01)$. The weights still look like kernels, but the
  bandwidth is now larger in low-density regions of the input domain.}     
\label{fig:ss_kernel}
\end{figure}

What we are seeing is an empirical validation of a beautiful asymptotic result
by \citet{silverman1984spline}, who proved that the cubic smoothing spline
estimator is asymptotically equivalent to a kernel regression estimator, with an 
unusual choice of kernel. Specifically, under suitable regularity conditions, we
have the large $n$ approximation,\footnote{Silverman actually shows that
  \smash{$w(x, z) \approx \frac{1}{h(z)} \frac{1}{p(z)} K( \frac{x-z}{h(z)})$},
  but the version we are stating has a more natural interpretation with respect
  to Figure \ref{fig:ss_kernel}, and is a consequence of symmetry of the
  smoother matrix $S$.}  
\[
w(x, z) \approx \frac{1}{h(x)} \frac{1}{p(x)} K\bigg( \frac{x-z}{h(x)} \bigg), 
\]
where $K$ is the ``Silverman kernel'': 
\[
K(t) = \frac{1}{2} \exp(-|t|/\sqrt{2}) \sin(|t|/\sqrt{2} + \pi/4), 
\]
and we have the local bandwidth: 
\[
h(x) = \bigg[ \frac{\lambda}{p(x)} \bigg]^{1/4},
\]
where $p(x)$ is the density of the input distribution at $x$. That is, the
bandwidth adapts to the local distribution of inputs. 

The Silverman kernel is ``kind of'' a higher-order kernel. It satisfies 
\[
\int K(t) \, dt = 1, \quad
\int t^j K(t) \, dt = 0, \; j=1,\dots,3, 
\quad \text{but} \quad 
\int t^4 K(t) \, dt = -24.
\]
So it lies outside the scope of usual kernel analysis. There is a lot more work
building off of Silverman's initial work that connects smoothing splines to
equivalent kernels. 

\subsection{Rate of convergence} 

Compared to that for kNN and kernel smoothing, the error analysis for smoothing
splines is more nuanced. Hence we'll dedicate a whole lecture to learning the
tools behind it, a bit later in the course. The upshot is that we'll learn a
general framework, and accompanying probabilistic tools, that can be used 
to analyze numerous other nonparametric estimators (such as other penalized 
empirical risk minimizers).        

The punchline for smoothing splines will be as follows. If we assume that $f_0
\in W^{m,2}(L; [0,1])$ for a constant $L>0$, we write to mean that $D^m f_0$
exists in the weak sense (to be defined precisely below) and \smash{$\int_0^1
  [D^m f_0(x)]^2 \, dx \leq L^2$}, and we assume mild conditions on the input 
points and sub-Gaussian noise, then the smoothing spline estimator of degree
$k=2m-1$ with tuning parameter \smash{$\lambda \asymp n^{1/(2m+1)}$} satisfies    
\[
\max \Big\{ \|\hf - f_0\|_n^2 , \, \|\hf - f_0\|_2^2 \Big\} \lesssim
n^{-2m/(2m+1)} \quad \text{in probability},
\]
with respect to randomness in the draws of $(x_i,y_i)$, $i=1,\dots,n$. 

\section{Sobolev theory*}

In somewhat of an interlude, we use this section to introduce Sobolev
spaces. This material can be mostly skipped without interrupting the flow of
understanding the main ideas in the rest of this lecture, hence the
asterisk. Well actually, you should probably read the definition of weak 
differentiability, but you can treat the rest as optional. Yes, optional, but
also super interesting and fairly fundamental in many ways, which is why we
include some of the core details of Sobolev spaces here. To learn more, an
excellent reference is \citet{evans2010partial}.  

\subsection{Weak derivatives}

First we introduce what is known as \emph{weak differentiability}. A function $f
: U \to \R$, where $U \subseteq \R^d$ is an open set, is called weakly
differentiable provided that there is some function $g : U \to \R^d$ such that
\begin{equation}
\label{eq:weak_derivative}
\int_U f(x) D \phi(x) \, dx = - \int_U g(x) \phi(x) \, dx,
\quad \text{for all $\phi \in C_c^\infty(U)$}, 
\end{equation}
where $C_c^\infty(U)$ is the set of all infinitely differentiable functions with
compact support in $U$, and $D\phi$ is the derivative of one such function
$\phi$. The function $g$ satisfying \eqref{eq:weak_derivative} is unique almost
everywhere, and denoted by $Df$ henceforth, called the weak derivative of $f$.

The motivation for this definition of weak differentiability is integration by
parts: if $f,\phi$ are differentiable in the classical (usual) sense, then
\smash{$\int_U f(x) D \phi(x) \, dx = - \int_U Df(x) \phi(x) \, dx$} by
integration by parts (applied componentwise), where there are no boundary terms
because $\phi$ has compact support. Thus, we define the weak derivative $g = Df$
such that this holds for all infinitely differentiable compactly-supported
$\phi$.

Higher-order derivatives follow similarly. For a multi-index $\alpha =
(\alpha_1,\dots,\alpha_d) \in \Z^d_+$, recall, we write $|\alpha| = \alpha_1 +
\cdots + \alpha_d$ and we denote   
\[
D^\alpha \phi = \frac{\partial^{|\alpha|} \phi}{\partial x_1^{\alpha_1} \partial 
  x_2^{\alpha_2} \dots \partial x_d^{\alpha_d}},
\]
interpreted in the classical (usual) sense for a function $\phi$. In the weak
sense, $f$ is said to have $\alpha\th$ weak differentiable $g$ if 
\begin{equation}
\label{eq:weak_derivative_alpha}
\int_U f(x) D^\alpha \phi(x) \, dx = - (1)^{|\alpha|} \int_U g(x) \phi(x) \, dx,
\quad \text{for all $\phi \in C_c^\infty(U)$}.   
\end{equation}
Again, the function $g$ satisfying \eqref{eq:weak_derivative_alpha} is unique
almost everywhere, and we denote by $D^\alpha f$.

Of course, the weak derivative reduces to the classical derivative exists when
the latter exists. However, weak differentiability is more general. An example
of a function that is weakly differentiable but not classically differentiable
is
\[
f(x) = x_+, \quad \text{with weak derivative $Df(x) = 1\{x > 0\}$.} 
\]
An example of a function that is \emph{not} weakly differentiable is $f(x) =
1\{x > 0\}$. For this, one can check that the condition
\eqref{eq:weak_derivative} cannot hold: taking $\phi$ to be any function whose
support includes 0, we learn that
\[
\phi(0) = \int g(x) \phi(x) \, dx.
\]
This cannot possibly be true for all infinitely differentiable $\phi$. This
cannot possibly be true for all $\phi$ unless $g=0$. But if $g=0$, then we
cannot recover $\phi(0)$ through integration, since the right-hand side above
will always be 0.   

\subsection{Sobolev spaces}

We are now equipped to define what are called \emph{Sobolev spaces}. For an
integer $k \geq  0$, and $1 \leq p \leq \infty$, and an open domain $U \subseteq 
\R^d$, we define the Sobolev space        
\[
W^{k,p}(U) = \Big\{ f : U \to \R \,:\, \text{$D^\alpha f$ exists in the weak
  sense, and $\|D^\alpha f\|_p < \infty$ for all $|\alpha| \leq k$} \Big\}. 
\]
Here $\|\cdot\|_p$ denotes the $L^p$ norm on $U$: 
\[
\|f\|_p^p = \int_U |f(x)|^p \, dx \quad \text{for $p < \infty$}, 
\]
and \smash{$\|f\|_\infty = \esssup_{x \in U} \, |f(x)|$} (the essential
supremum, the smallest upper bound possible over all subsets of $U$ of full
measure).

Sobolev spaces are central in the study of partial differential equations, which
is commonly where you'll find people (mathematicians) referring to them. But
they also play an important role in nonparametric statistics. For example, if
you recall the smoothing spline problem \eqref{eq:ss}, where we said that ``the
minimization is all functions $f$ for the which the criterion is well-defined
and finite'', we can actually take the domain here to be the $L^2$ Sobolev 
space\footnote{We'll ignore here, and in several other places, the fact that 
  the domain is closed. Recall, we defined Sobolev spaces on open domains, 
  and we will do the same for H{\"o}lder spaces as well. Hence, we are being 
  slightly imprecise by abruptly allowing closed domains but it's easiest to do 
  so for simplicity of exposition.}    
\[
W^{m,2}([a,b]) = \Big\{ f : [a,b] \to \R \,:\, \text{$D^\alpha f$ exists in the
  weak sense, and $\|D^\alpha f\|_2^2 < \infty$, for all $\alpha \leq m$}
\Big\}.     
\]

We note that, generally speaking, when working in Sobolev spaces (just like
$L^p$ spaces), we identify two functions that are equal almost everywhere. In 
other words, each element $f$ in a Sobolev space is really an equivalence class
of functions, any pair of which agree except on a set of measure zero.

Now we recall \emph{H{\"o}lder spaces}. We touched on these in the lecture on 
kernels. Formally, for an integer $r \geq 0$, and $0 < \gamma \leq 1$, we define
the H{\"o}lder space
\begin{multline*}
C^{r+\gamma}(U) = \Bigg\{ f : U \to \R \,:\, \text{$D^\alpha f$ exists,
  $\|D^\alpha f\|_\infty < \infty$ for all $|\alpha| \leq r$, and} \\   
  \sup_{x \not= z} \, \frac{|D^\alpha f(x) - D^\alpha f(z)|}{\|x-z\|_2^\gamma} < 
  \infty \; \text{for all $|\alpha| = r$} \bigg\}.   
\end{multline*}
In the above, we can interpret $D^\alpha f$ as a classical derivative, and
$\|\cdot\|_\infty$ as the supremum norm (we don't need the essential supremum
norm). This may all seem more restrictive than Sobolev spaces, but as we'll
discuss just below, we can actually identify the H{\"o}lder space with a
particular Sobolev space.       

Let's pause to examine the relationship between the Sobolev space
$W^{k,\infty}(U)$ and the H{\"o}lder space $C^k(U)$. The former contains all
functions $f$ such that the $\alpha\th$ weak derivative of $f$ is bounded for
all $|\alpha| \leq k$. The latter contains all functions $f$ such that the
$\alpha\th$ derivative of $f$ is bounded for all $|\alpha| < k$, and $D^\alpha
f$ Lipschitz for $|\alpha| = k$. It turns out that these two conditions are
really saying the same thing. Trying to make this statement precise for general
domains is probably not worth it, so we'll just say that for a bounded domain
$U$ with smooth boundary, we can identify $W^{k,\infty}(U)$ with $C^k(U)$. To be
precise, in doing so, we identify each $f$ in the Sobolev space with its
classically differentiable version.  

% Theorem 4 in Chapter 5.8 of Evans (2010), requires U to be open and bounded
% with C^1 boundary

Much more can be said about the connection between Sobolev and H{\"o}lder
spaces, which is covered next.

\subsection{Embedding theorems}

Among the whole optional part of material on Sobolev spaces, this next bit   
really is the \emph{most} optional, but it's too cool not to cover. We start by
defining norms on Sobolev and H{\"o}lder spaces; namely,
\[
\|f\|_{W^{k,p}(U)} = 
\begin{dcases} 
\Bigg( \sum_{|\alpha| \leq k} \|D^\alpha f\|_p^p \Bigg)^{1/p} & \text{if $p <
  \infty$} \\
\;\;\; \sum_{|\alpha| \leq k} \|D^\alpha f\|_\infty & \text{if $p = \infty$}   
\end{dcases}
\qquad \text{for $f \in W^{k,p}(U)$},
\]
and 
\[
\|f\|_{C^{r+\gamma}(U)} = \sum_{|\alpha| \leq r} \|D^\alpha f\|_\infty +
\sum_{|\alpha| = r} \sup_{x \not= z} \, \frac{|D^\alpha f(x) - D^\alpha f(z)|} 
{\|x-z\|_2^\gamma}
\qquad \text{for $f \in C^{r+\gamma}(U)$}. 
\] 
Equipped with these norms, $W^{k,p}(U)$ and $C^{r+\gamma}(U)$ are Banach 
spaces (complete normed linear spaces). In fact, when $p=2$, the space
$W^{k,p}(U)$ is a Hilbert space (complete inner product space) under the inner
product \smash{$\langle f, g \rangle_{W^{k,p}(U)}   
= \sum_{|\alpha| \leq k} \langle D^\alpha f, D^\alpha g \rangle_2 
= \sum_{|\alpha| \leq k} \int_U D^\alpha f(x) D^\alpha g(x) \, dx$.}

Now here comes the cool part, which are special inequalities\footnote{In an
  unfortunate clash of nomenclature with statistics, you'll often hear
  mathematicians calling these ``estimates''.} involving Sobolev norms that lead 
to what are known as \emph{embedding theorems}. There are really several
embedding theorems that were developed over many years, by different authors
(Gagliardo, Nirenberg, Sobolev, Morrey, others), but now the totality of them is
usually just called ``the Sobolev embedding theorem''. In a nutshell, this is
what it says (recall $d$ is the dimension of the domain $U$; and we're
intentionally vague about what conditions it needs to satisfy).  

\paragraph{The case $pk < d$, subcritical regime.}

In this case we a have lower smoothness-to-dimension ratio. Let $0 \leq \ell <
k$ be an integer, and $p < q < \infty$, such that the pair satisfies   
\[
\frac{1}{q} - \frac{\ell}{d} = \frac{1}{p} - \frac{k}{d}.
\]
Then we have, for a constant $C>0$ depending only on $k,p,\ell,q$ and $U$, 
\[
\|f\|_{W^{\ell,q}(U)} \leq C \|f\|_{W^{k,p}(U)}, \quad \text{for all $f \in
  W^{k,p}(U)$}.
\]
This means that $W^{k,p}(U) \subseteq W^{\ell,q}(U)$, and moreover we have what 
is known as a \emph{continuous embedding} of normed spaces. From the above    
inequality, we learn that any norm ball of radius $\rho$ in the former space
is contained in a normal ball of the latter, whose radius is at most $C\rho$.

Note that in the special case where we take $\ell = 0$, we get that for  
\[
q = \frac{pd}{d-pk},
\]
and a constant $C>0$ depending only on $k,p,q$ and $U$, it holds that 
\[
\|f\|_q \leq C \|f\|_{W^{k,p}(U)}, \quad \text{for all $f \in W^{k,p}(U)$}. 
\]
This means that $W^{k,p}(U) \subseteq L^q(U)$, and the embedding is continuous.  

% Theorems 2 and 6 in Chapter 5.6 of Evans (2010), requires U to be open and
% bounded with C^1 boundary, doesn't actually state the result for general \ell,
% only for \ell = 0 

% However, I think boundedness is only needed for compactness. Because Chapter
% 11 of Leoni (2009) proves continuous embeddings on all on \R^d

\paragraph{The case $pk > d$, supercritical regime.}

In this case we a have higher smoothness-to-dimension ratio. Let $0 \leq r < k$
be an integer, and $0 < \tau \leq 1$, such that the pair satisfies 
\[
\frac{r+\tau}{d} = \frac{k}{d} - \frac{1}{p}.
\]
A technical detail: if the value of $r,\tau$ satisfying the above results in
$\tau = 1$, then we need to ``downgrade'' it to a value less than 1. That is,
let $\gamma = \tau$ if $\tau < 1$, and otherwise let $0 < \gamma < 1$. Then we
have, for a constant $C>0$ depending only on $k,p,r,\gamma$ and $U$,       
\[
\|f\|_{C^{r+\gamma}(U)} \leq C \|f\|_{W^{k,p}(U)}, \quad \text{for all $f \in
  W^{k,p}(U)$}. 
\]
This means that $W^{k,p}(U) \subseteq C^{r+\gamma}(U)$, and the embedding is
continuous. In view of this, we can always identify each function $f \in
W^{k,p}(U)$ with its $r$ times classically differentiable version.     

% Theorems 5 and 6 in Chapter 5.6 of Evans (2010), requires U to be open and 
% bounded with C^1 boundary, doesn't quite state the result the same way we do 

% But same comment applies about \R^d being fine for continuous embedding 

\medskip
(The critical regime $pk = d$ is actually a bit more subtle and we'll skip over
it.)  

How do we interpret the Sobolev embedding theorem? In the subcritical regime, it
says that we can trade off smoothness in the differential sense with smoothness
in the integral sense. Given a $k$ times (weakly) differentiable function, whose
derivatives are in $L^p(U)$, the embedding theorem tells us that its derivatives
of order $\ell < k$ are less ``peaky'' and more ``evenly spread out'', since
they are $L^q(U)$ with $q > p$.

In the supercritical regime, the conclusion is arguably even more
fascinating---it says that knowing a function's derivatives are smooth in a
global (integrated) sense tells us something about local smoothness of lower
order derivatives, at each point in the domain. Given a $k$ times (weakly)
differentiable function, the embedding theorem starts off with a statement about
the size of \smash{$\sum_{|\alpha| \leq k} \int_U |D^\alpha f(x)|^p \, dx$}, and
translates this into a statement about the H{\"o}lder constant of $D^\alpha f$
for $|\alpha| < k$.      

Figure \ref{fig:sobolev} gives an illustration of the tradeoffs navigated by the
Sobolev embedding theorem (these types of plots are sometimes called ``DeVore
diagrams'' in honor of Ron DeVore).   

\begin{figure}[tb]
\centering
\includegraphics[width=\textwidth]{sobolev.pdf}
\caption{\it Illustrations of the Sobolev embedding theorem. In each plot, we 
  parametrize the x-axis by $1/p$ and the y-axis by $k$. Hence, as we move
  right-to-left, we increase the index of the $L^p$ norm that is used to measure 
  integral smoothness, and functions get less ``locally peaked'' and more 
  ``evenly spread out''. As we move top-to-bottom, we decrease the number of
  (weak) derivatives under consideration. Left: subcritical regime with $pk <
  d$. We traverse a line of slope $d$, moving down and to the left, in order to
  visit the different Sobolev spaces into which $W^{k,p}(U)$ embeds, culminating
  in an intersection with the x-axis ($k=0$) at $q = pd/(d-pk)$. Right:
  supercritical regime with $pk > d$. Similarly, we traverse a line of slope $d$
  to find the H{\"o}lder space into which $W^{k,p}(U)$ embeds, which is given by
  an intersection with the y-axis ($q = \infty$). The open white circles denote
  cases in which the embedding fails for $\gamma = 1$.}  
\label{fig:sobolev}
\end{figure}

\section{Multivariate splines}

We move on to methods for fitting multivariate splines in nonparametric
regression. In a sense, both of the following statements are true. 

\begin{enumerate}
\item There are several multivariate extensions of spline estimators: e.g,
  tensor product splines and thin plate splines (among others). We'll cover
  these. 
\item There are no real multivariate extensions of spline estimators: tensor
  product splines and thin plate splines are not ``really'' splines. In fact,
  even defining a multivariate spline is generally very tricky. 
\end{enumerate}

How can this be? Let's go about this backwards, and start by explaining the very
last part of the second point: suppose you were to (reasonably) insist that a 
multivariate spline should be, like a univariate spline, a piecewise polynomial
of a degree $k$ that is $C^{k-1}$, which means it has continuous derivatives of
all orders less than $k$. Then in general this is going to be tricky to fulfill
when $d \geq 2$. This is essentially due to the number in the number of
constraints such continuity imposes in the multivariate setting. Recall that a
$k\th$ degree polynomial in $d$ dimensions is of the form (using multi-index
notation $\alpha = (\alpha_1,\dots,\alpha_d) \in \Z^d_+$):
\begin{equation}
\label{eq:polynomial}
f(x) = \sum_{|\alpha| \leq k} \beta_\alpha \, x_1^{\alpha_1} x_2^{\alpha_2}
\dots x_d^{\alpha_d},
\end{equation}
for coefficients $\beta_\alpha$, $|\alpha| \leq k$. The number of coefficients,
and hence the dimension of the space of $k\th$ degree polynomials in $d$
dinensions, is
\[
\sum_{\ell=0}^k {d-1+\ell \choose \ell} = {d + k \choose k}.
\]
Now let's just think about cubic degree $k=3$ in dimension $d=2$, and try to  
understand the claim that it's tricky to construct a $C^2$ piecewise cubic,
when we consider the ``pieces'' to be triangles. In fact, it's already going to
be hard enough to construct a $C^1$ piecewise cubic. Suppose that we have two  
triangles sharing an edge $e$. Then on each triangle we have \smash{${5 \choose  
    2} = 10$} parameters to define the cubic, and hence 20 parameters in
total. Denoting by $f,g$ the two cubics on our adjacent triangles, the $C^1$ 
condition says that
\[
D^\alpha f(x) = D^\alpha g(x), \quad \text{for all $|\alpha| \leq 1$ and all $x
  \in e$}. 
\] 
It can be shown that this reduces to 10 constraints on the coefficients; but
importantly, the structure of these constraints is such that, for a given set of
values at the vertices of the triangles, it is unclear whether there will exist
a set of coefficients that satisfies the constraints \emph{and} meets the
prescribed values at the vertices. This basically happens because the linear 
constraints are all entangled. Figure \ref{fig:c1_constraints} gives an
illustration.    

To get around this problem we have to increase the degree of the polynomial. As
it turns out, for quintic degree $k=5$, this problem doesn't occur for two
adjacent triangles in dimension $d=2$, and we get what is sometimes called a
``free'' $C^1$ smoothness condition, where this kind of entanglement of
constraints doesn't happen. One can ask: in dimension $d$, what is the
requirement on the degree $k$ (for a generic partition of points into simplices)
such that we get a ``free'' $C^s$ smoothness condition? The answer is,
disappointingly,              
\[
k \geq s(d+1) + d.
\]
We see that for ``free'' $C^1$ smoothness when $d=2$, the minimal degree is
indeed $k=5$, and when $d=3$ it is $k=7$. This seems very wasteful.   

\begin{figure}[tb]
\centering
\includegraphics[width=0.5\textwidth]{c1_cubic.png}
\caption{\it Illustration of the constraints for $C^1$ smoothness, for a cubic
  on each triangle. Credit: \citet{deboor2009way}. A polynomial of an arbitrary
  degree on a triangle can be written be in what is called a B-form (as an
  expansion of what are called Bernstein basis polynomials), whose coefficients
  can be assigned to anchor points, drawn above as blue circles. The set of
  coefficients that are tied together in a linear equation needed for $C^1$
  smoothness are highlighted as a red quadrilateral. We can see that the linear
  systems are entangled, since pairs of them overlap at an anchor point.}     
\label{fig:c1_constraints}
\end{figure}

This is not a complete show-stopper for multivariate splines and there is much
more to the story than this but we'll not travel down this path. To learn more,
two definitive references are \citet{deboor1993box, lai2007spline}, and
\citet{deboor2009way} is a lighter, enjoyable review.    

\subsection{Tensor product splines}

One way forward in the multivariate setting is just to take tensor
products. Recall, a tensor product of two univariate functions $f_1,f_2$ on
$[a,b]$ is denoted $f_1 \otimes f_2$, which is the function on $[a,b]^2$ defined
as  
\[
(f_1 \otimes f_2)(x_1, x_2) = f_1(x_1) f_2(x_2). 
\]
Similarly, for any collection of $q \geq 2$ functions $f_1,\dots,f_q$, we can
define the tensor product $f_1 \otimes f_2 \otimes \cdots \otimes f_q$ as a
function 
\[
(f_1 \otimes f_2 \otimes \cdots \otimes f_q)(x_1, x_2, \dots, x_q) = f_1(x_1) 
f_2(x_2) \dots f_q(x_q).
\]

Thus given a $k\th$ degree spline basis with $r$ knots $t_1,\dots,t_r$, which we
denote by $g_1,\dots,g_N$ for $N=r+k+1$, we can take all $d$-way tensor
products to form collection of functions on $[a,b]^d$:
\[
g_{j_1} \otimes g_{j_2} \otimes \cdots \otimes g_{j_d}, \quad \text{for all
  $\ell = (j_1, j_2, \dots, j_d) \in [N]^d$}, 
\]
where we use the abbreviation $[N] = \{1,\dots,N\}$. By construction the linear
span of this collection, i.e., functions of the form    
\[
f(x) = \sum_{j \in [N]^d} \beta_j \, g_{j_1}(x_1) \otimes \cdots \otimes 
g_{j_d}(x_d),   
\]
for coefficients $\beta_j$, $j \in [N]^d$, generates the space of tensor 
product splines 
\[
\Big\{ f_1 \otimes \cdots \otimes f_d : \text{each $f_\ell$ is a $k\th$ degree
  spline with knots in $t_1,\dots,t_r$, for $\ell=1,\dots,d$} \Big\}.  
\]
Figure \ref{fig:tensor} displays a few tensor products of bivariate B-splines.

\begin{figure}[!t]
\centering
\includegraphics[width=0.925\textwidth]{tensor.png}
\caption{\it Bivariate tensor products of cubic B-splines. Credit: Chapter 5.7
  of \citet{hastie2009elements}.}    
\label{fig:tensor}
\end{figure}

To fit a tensor product spline from data, the simplest thing to do would be the 
\emph{tensor product regression spline}, whose coefficients are given by solving  
\[
\minimize_\beta \; \sum_{i=1}^n \bigg(y_i -  \sum_{j \in [N]^d} \beta_j \,
g_{j_1}(x_{i1}) \otimes \cdots \otimes g_{j_d}(x_{id}) \bigg)^2. 
\]
Denoting by $A \otimes B$ the Kronecker product of matrices $A,B$, and denoting
by $G_\ell \in \R^{N \times N}$ the basis matrix for dimension $\ell$, with
entries $[G_\ell]_{pq} = g_q(x_{p\ell})$, we can express the above in more
compact form as  
\[
\minimize_\beta \; \big\| Y - (G_1 \otimes \cdots G_d) \beta \big\|_2^2. 
\]
As usual, given the solution \smash{$\hbeta$},  we make predictions according to
\smash{$\hf(x) = \sum_{j \in [N]^d} \hbeta_j \, g_{j_1}(x_1) \otimes \cdots
  \otimes g_{j_d}(x_d)$}, and since \smash{$\hbeta$} is linear in $Y$, the
tensor product spline \smash{$\hf$} is a linear smoother.

Few notes: there is no reason that the knots have to be the same along each 
dimension, that is only done here for simplicity. We could have also used
regularization, though for ``variationally-motivated'' regularization we'd turn 
to thin plate splines (covered next).

\subsubsection{What are these functions?} 

We should be clear that tensor product construction \emph{does not ``really'' 
  give us a multivariate spline}, if we maintain that a spline is a $k\th$
piecewise polynomial that is also $C^{k-1}$. Why? Note that a tensor product of
univariate $k\th$ degree splines produces a function that is indeed $C^{k-1}$,
and indeed a piecewise polynomial: it is a polynomial on each hypercube of the
form        
\[
[t_{j_1}, t_{j_1+1}] \times \cdots \times [t_{j_d}, t_{j_d+1}]. 
\]
But it is \emph{not} necessarily a $k\th$ degree polynomial on each
hypercube; it is a tensor product of univariate $k\th$ degree
polynomials. Technically, out of this we can get a polynomial of degree up to 
$k^d$, and at the same time, we can't actually generate any polynomial of degree 
$k^d$. We are restricted to functions of the form   
\[
f(x) = \sum_{\alpha_1,\dots,\alpha_d \leq k} \beta_\alpha \, x_1^{\alpha_1}
x_2^{\alpha_2} \dots x_d^{\alpha_d}.
\]
There is important difference here, compared to \eqref{eq:polynomial}. As a
concrete example, take $k=1$ and $d=2$: then the tensor product of univariate
linear functions give us (what are called bilinear) functions of the form
\[
1 + a x_1 + b x_2 + c x_1 x_2,
\]
which is \emph{not} linear, because of the cross term $x_1 x_2$. At the same
time, we can't get any quadratic out of this, because we're missing the terms
$x_1^2$ and $x_2^2$. 

This seems to bothers some people (researchers, practitioners), but doesn't
bother others---it really must depend on the problem setting or application one
has in mind. Sometimes the highly anisotropic (coordinate-aligned) nature of a
tensor product spline may be desirable, and other times it may not.    

\subsection{Thin plate splines}

An alternative route is to try to extend the variational problem \eqref{eq:ss}
that defined the univariate smoothing spline. At the outset, let's aim for the
most general extension, which is to define an estimator by solving
\begin{equation}
\label{eq:ss_multivariate}
\minimize_f \; \sum_{i=1}^n (y_i - f(x_i))^2 + \lambda \int_U \sum_{|\alpha| =
  m} [D^\alpha f(x)]^2 \, dx,   
\end{equation}
for an integer order $m \geq 1$. Here $U$ is a subset of $\R^d$ that contains
the input points $x_i$, $i=1,\dots,n$, and the minimization is over all
functions $f$ for the which the criterion is well-defined and finite. For
example, we can consider the minimization to be over the Sobolev space
$W^{m,2}(U)$,\footnote{There is a subtlety here about what we actually mean here
  for unbounded domain like $U = \R^d$. The problem is that the Sobolev space 
  $W^{m,2}(\R^d)$ doesn't admit polynomials of degree $m-1$, whereas we'd like
  all of these to be allowed in the minimization. We may follow up later with a
  separate note about this.}   
which recall contains $m$ times weakly differentiable functions with $\alpha\th$
derivative in $L^2(U)$, for $|\alpha| \leq m$. 

% RJT note: I don't love this, because when we take U = \R^d, then W^{m,2}(\R^d)
% doesn't contain polynomials of degree m, since these don't have finite L2 norm
% of lower order derivatives, but we really do want to allow polynomials in the
% minimization. So considering the domain to be W^{m,2}(U) is really only the
% right thing to do when U is bounded. But then that spoils the RKHS connection.    

Is this a good idea? It depends. When $2m > d$, problem
\eqref{eq:ss_multivariate} is well-defined in the sense that it admits a
solution. This solution is what we'll eventually call the thin plate spline
estimator (in a particular case). But when $2m \leq d$, there is a fundamental
issue with this approach: problem \eqref{eq:ss_multivariate} is
\emph{ill-defined in the sense that it has no solution}. This is a manifestation
of the Sobolev embedding theorem.

\subsubsection{Return of the Sobolev embedding theorem}

Let's recall what we learned from the Sobolev embedding theorem earlier (if you
skipped it, since we did say it it was optional, then also that's fine---the
following explanation should still be pretty self-contained). When $2m > d$,
which we called the supercritical regime, the Sobolev space $W^{m,2}(U)$ embeds
continuously into a H{\"o}lder space $C^{r+\gamma}(U)$, so in particular it
embeds continuously into $C^0(U)$: this is simply the space of continuous
functions on $U$ equipped with the $L^\infty$ norm. %\footnote{On $C^0(U)$, 
  % as with all H{\"o}lder spaces, we take $L^\infty$ to be the sup norm, and
  % \emph{not} the essential sup norm.}     
Why is this important? Well, given a sequence such that $f_k \to f$ as $k \to
\infty$ in \smash{$\|\cdot\|_{W^{m,2}(U)}$} norm, the continuous embedding
property implies $f_k \to f$ as $k \to \infty$ in $L^\infty$ norm, which implies
$f_k(x) \to f(x)$ for all $x$. Thus we have shown a critical fact:
\begin{quote}
\centering\it
For $2m > d$, the point evaluation operator is continuous on $W^{m,2}(U)$.   
\end{quote}
This means we can do things like solve the variational problem
\eqref{eq:ss_multivariate}. 

But when $2m \leq d$, we do not get an embedding into $C^0(U)$, and it turns out
that:   
\begin{quote}
\centering\it
For $2m \leq d$, the point evaluation operator is not continuous on
$W^{m,2}(U)$.  
\end{quote}
This means that there can be a sequence such that $f_k \to f$ as $k \to \infty$
in \smash{$\|\cdot\|_{W^{m,2}(U)}$} norm but $f_k(x) \not\to f(x)$ for some
$x$. In fact, it's not hard to see this from first principles when $2m < d$.
Let $f$ to be an infinitely differentiable ``bump'' function that is symmetric
about the origin, with $f(0) = 1$, and zero outside of the unit $\ell_2$ ball
$\{x : \|x\|_2 \leq 1\} \subseteq U$. Now define $f_k(x) = f(k x)$, for
$k=1,2,3,\dots$, so that $f_k$ collapses smoothly to a spike at the origin as $k 
\to \infty$. Then
\[
\int \sum_{|\alpha| = m} [D^\alpha f_k(x)]^2 \, dx = 
k^{d-2m} \int \sum_{|\alpha| = m} [D^\alpha f(u)]^2 \, du, 
\]
simply by a change of variables $u = k x$. If $d > 2m$, then as $k \to \infty$,
we get that $f_k \to 0$ (the zero function) in \smash{$\|\cdot\|_{W^{m,2}(U)}$}
norm. However, recall $f_k(0) = 1$ for all $k$, so clearly we do not get
pointwise convergence at the origin. Altogether, this means we \emph{cannot} do
things like solve the variational problem \eqref{eq:ss_multivariate}.  

\subsubsection{Restrictions to the rescue}

To circumvent the above issue, we can just restrict $2m > d$. Note that this
forces us to take more derivatives as $d$ grows, which is not really desirable
(more on this later). In any case, when $d=2$, it is valid to take $m=2$, which
yields the \emph{thin plate spline} estimator, defined by solving   
\begin{equation}
\label{eq:thin_plate1}
\minimize_f \; \sum_{i=1}^n (y_i - f(x_i))^2 + \lambda \int_U \|\nabla^2 
f(x)\|_F^2 \, dx, 
\end{equation}
where $\nabla^2 f$ denotes the weak Hessian (matrix of weak second derivatives)
of $f$, and $\|\cdot\|_F$ is the Frobenius norm. This can be equivalently
written as  
\begin{equation}
\label{eq:thin_plate2}
\minimize_f \; \sum_{i=1}^n (y_i - f(x_i))^2 + \lambda \int_U \bigg[ \bigg( 
\frac{\partial^2 f(x)}{\partial x_1^2}\bigg)^2 +  2\bigg( \frac{\partial^2
  f(x)}{\partial x_1 \partial x_2} \bigg)^2 +  \bigg( \frac{\partial^2
  f(x)}{\partial x_2^2} \bigg)^2 \bigg] \, dx. 
\end{equation}
Like the cubic smoothing spline problem \eqref{eq:ss_cubic}, the thin plate
spline problem \eqref{eq:thin_plate2} (or equivalently, \eqref{eq:thin_plate1}) 
has a representer theorem for its solution. Unlike the smoothing spline, the
solution actually depends on the choice of the domain $U$. For $U = \R^d$,  
% RJT note: again I don't love this because when U = \R^d, we're not really
% doing minimization over the Sobolev space below. So this is kind of murky
% territory. 
one can show that it suffices to consider functions of the form   
\begin{equation} 
\label{eq:polyharmonic}
f(x) = a + b^\T x + \sum_{j=1}^n \beta_j \eta(\|x - x_j\|_2), 
\end{equation}
where $\eta$ is the so-called \emph{radial basis function}
\[
\eta(r) = \frac{1}{16 \pi} r^2 \log r^2,
\]
(As $\log 0$ is undefined, we adopt a continuous extention at zero, and set
$\eta(0) = 0$.) In order for the penalty in the problem \eqref{eq:thin_plate2}
to be finite, a necessary and sufficient condition on the coefficients is that 
\[
0 = \sum_{i=1}^n \beta_j = \sum_{j=1}^n \beta_j x_j.
\]
Note that there are 3 linear constraints here (since each $x_i$ is in $\R^2$), 
and there are $n+3$ parameters in \eqref{eq:polyharmonic}, which means that in
total, subject the constraints, we have $n$ free parameters.  

Using this constrained parametrization, we can translate \eqref{eq:thin_plate2}
into a finite-dimensional problem. One can show that it is a constrained
generalized ridge regression, thus its solution, the fitted coefficients
\smash{$\hat{a}, \hat{b}, \hbeta$}, are linear in $Y$. As we make predictions
according to \smash{$\hf(x) = \hat{a} + \hat{b}^\T x + \sum_{j=1}^n \hbeta_j
  \eta(\|x - x_i\|_2)$}, we see that the thin plate spline is a linear smoother.
The story for what happens on a bounded domain $U$ is similar, but a bit more
complex. See Chapters 7.4--7.7 of \citet{green1993nonparametric} for precise
details.      

The function $f$ in \eqref{eq:polyharmonic} often called a \emph{polyharmonic
  spline}. Again, it is \emph{not ``really'' a multivariate spline}, because it
is not a piecewise polynomial. But let's emphasize an important property: $\eta$
is symmetric around the origin. In fact, if we just look back at the criterion
\eqref{eq:thin_plate1}, we see that if we rotated the coordinate system by an
orthogonal transform $V \in \R^{2 \times 2}$, and take our new coordinates to be
$z=Vx$, then we get  
\[
\minimize_f \; \sum_{i=1}^n (y_i - f(z_i))^2 + \lambda \int \underbrace{\| V
  \nabla^2 f(z) V^\T\|_F^2}_{= \|\nabla^2 f(z)\|_F^2} \, dz,   
\]
where we used the fact that $\|V A V^\T\|_F = \|A\|_F$ for any matrix $A$ due to
orthogonality of $V$. This means that the thin plate spline estimator is
\emph{rotationally invariant}: if we rotated the coordinate system of the input
points, then the new solution would just be a rotation of the old solution. This
is \emph{not} true of tensor product splines. 

What about more than 2 dimensions? When $d=3$, the choice $m=2$ is still
valid, and the corresponding estimator is often still called the thin plate
spline. The details are similar to the above. In fact, whenever $2m > d$, the 
variational problem \eqref{eq:ss_multivariate} has a representer theorem and
admits a finite-dimensional solution which is a polyharmonic spline, and the
problem reduces to a generalized ridge regression. See Chapter 7.9 of
\citet{green1993nonparametric}. Instead of pursuing these details, we'll simply
move on to RKHS theory, as this will provide us a more general perspective. 

\subsubsection{Can we do anything else?}

Before moving on to RKHS theory, it's worth emphasizing restrictive the
condition $2m > d$ is. Though it's a very simple calculation, Table
\ref{tab:supercritical} lists the requirement on $m$ for $d=1$ through 5.  

\begin{table}[htb]
\centering
\begin{tabular}{c|c}
Dimension & Restriction \\
\hline
$d=1$ & $m \geq 1$ \\
$d=2$ & $m \geq 2$ \\
$d=3$ & $m \geq 2$ \\
$d=4$ & $m \geq 3$ \\
$d=5$ & $m \geq 3$ 
\end{tabular}
\caption{\it Translating the supercritical condition $2m > d$, for each $d 
  \leq 5$.} 
\label{tab:supercritical} 
\end{table}

This is obviously not desirable: the choice of smoothness order $m$ (which then
dictates the degree of the polyharmonic spline) should be up to the modeler, and
not dictated by the ambient dimension.

There is another way. We can ``discretize'' the variational problem
\eqref{eq:ss_multivariate}, specifically, we can substitute the penalty with a
discrete approximation based on a graph Laplacian, where the graph is built 
using the input points $x_1,\dots,x_n$. This has the advantage of being (i)
always well-defined, and (ii) more computationally efficient than solving
\eqref{eq:ss_multivariate} even when the latter is well-posed. In fact, beyond
just being well-defined, the graph-based approach can even be minimax rate
optimal (over norm balls in Sobolev spaces) in regimes where the variational
problem \eqref{eq:ss_multivariate} is ill-posed. Of course, we're skirting a lot
of details, but the takeaway you should know is this:  
\begin{quote}
\centering\it
Consistent estimation over $W^{m,2}([0,1]^d)$---even minimax rate optimal
estimation---is still possible when $2m \leq d$. We can't use variational
methods like \eqref{eq:ss_multivariate}, but we can do other things.
\end{quote}
However, to qualify the above statement, we \emph{must} be in the Random-X
setting, where $x_i$, $i=1,\dots,n$ are random. If the input points were fixed,
then when $2m \leq d$, a construction using ``bump'' functions (just like the
one we gave above to show that the point evaluation operator is not continuous)
will show that consistent estimation is impossible. To learn more, see 
\citet{green2021minimax1}. See also \citet{green2021minimax2} for a related
approach that forms a nonparametric estimator by projecting onto eigenvectors of 
the graph Laplacian (rather than penalizing according to the graph Laplacian)
and has even stronger theory.  

\section{RKHS methods}

Lastly, we turn to RKHS methods. This will require covering some theory first in
order to properly understand what's going on, but RKHS theory is arguably easier
than Sobolev theory. Two great references on the former are
\citet{scholkopf2002learning, christmann2008support}.  

A \textbf{warning} before we begin: don't confuse any of what follows with
kernel smoothing (as covered in the last lecture) or the kernel functions used
therein! This is a completely separate concept. 

\subsection{Kernels, Hilbert spaces, RKHS}

We start off by defining the ingredients needed for an RKHS. 

\paragraph{Inner products.}

Let $\cH$ be a vector space over $\R$ (for our purposes: think of this as a
space of real-valued functions). An \emph{inner product} on $\cH$ is a function
$\langle \,\cdot\,, \,\cdot\, \rangle_\cH : \cH \times \cH \to \R$ satisfying
three properties, which must hold for all $f,g,h \in \cH$ and all $a,b \in \R$: 
\begin{enumerate}
\item $\langle af + bg, h \rangle_\cH = a \langle f,h \rangle_\cH + b \langle
  g,h \rangle_\cH$;
\item $\langle f,g \rangle_\cH = \langle g,f \rangle_\cH$;
\item $\langle f,f \rangle_\cH \geq 0$ with equality iff $f=0$. 
\end{enumerate}
We can always define a norm based on this inner product, denoted
$\|\cdot\|_\cH$, by 
\[
\|f\|_\cH^2 = \langle f, f \rangle_\cH.
\]
Equipped with its inner product $\langle \,\cdot\,, \,\cdot\, \rangle_\cH$, we
call $\cH$ an \emph{inner product space}. 

\paragraph{Hilbert spaces.} 

An inner product space $\cH$ is called a \emph{Hilbert space} if it is
complete. Recall that completeness means that Cauchy sequences converge to   
limits in the space.\footnote{Concretely, given any sequence $f_n$,
  $n=1,2,3,\dots$ in $\cH$ such that for any $\epsilon>0$, we have $\|f_n - 
  f_m\|_\cH < \epsilon$ for large enough $n,m$, it must hold that $\|f_n - 
  f\|_\cH \to 0$ as $n \to \infty$ for some $f \in \cH$.}

\paragraph{Kernels.}

There are different ways to define kernels, and in fact different ways to arrive 
at an RKHS. One way is this. A \emph{kernel} is function $k : \cX \times \cX \to 
\R$ for which there exists a map $\phi : \cX \to \cH$ (for our purposes: think
of this as a feature map) such that for any $x,y \in \cX$,  
\[
k(x,y) = \langle \phi(x), \phi(y) \rangle_\cH.
\]
Note that $k$ is always symmetric and nonnegative, because $\langle \,\cdot\,, 
\,\cdot\, \rangle_\cH$ is an inner product. Furthermore, it is easy to check
from first principles that $k$ satisfies the following property: for any $n \geq
1$ and $x_1,\dots,x_n \in \cX$, if we define a matrix $K \in \R^{n \times n}$ to 
have entries $K_{ij} = k(x_i, x_j)$, then  
\[
a^\T K a \geq 0, \quad \text{for all $a \in \R^n$}.
\]
A function $k$ that is symmetric and satisfies the above property is called 
\emph{positive semidefinite}. 

Thus, from what we have just argued, a kernel function is always positive
semidefinite. Remarkably, the converse is also true: any positive semidefinite
function $k : \cX \times \cX \to \R$ is a kernel. This means that \emph{in order
  to define a kernel $k$, we don't have to explicitly specify the feature map 
  $\phi$; we just need to satisfy positive semidefiniteness}.  

Examples of kernels on $\cX = \R^d$ include the polynomial kernel:   
\[
k(x,y) = (1 + x^\T y)^m,
\]
the exponential kernel:
\[
k(x,y) = \exp(x^\T y),
\]
and the Gaussian kernel: 
\[
k(x,y) = \exp(-\|x-y\|_2^2 / \sigma^2).
\]

Importantly, kernels need not be restricted to $\cX \subseteq \R^d$; they can be
defined over much more exotic spaces $\cX$, like spaces of images or text
strings, we just need a suitable (positive semidefinite) notion of similarity
between objects.   

\paragraph{Reproducing kernel Hilbert spaces.} 

A Hilbert space $\cH$ of functions is called a \emph{reproducing kernel Hilbert
  space} (RKHS), with kernel $k : \cX \times \cX \to \R$, provided the following
two conditions are satisfied:  
\begin{enumerate}
\item for any $x \in \cX$, the function $k(\cdot, x)$ is an element of $\cH$;   
\item for any function $f \in \cH$ and $x \in \cX$, it holds that $\langle f,
  k(\cdot, x) \rangle_\cH = f(x)$. 
\end{enumerate}
The functions $k(\cdot, x)$ in the first property are often called the
``reproducers of evaluation''. Accordingly, the second property is often called
the ``reproducing property''. In particular, note that the reproducing property
implies 
\[
k(x,y) = \langle k(\cdot, x), k(\cdot, y) \rangle_\cH,
\]
for any $x,y$. This means that the map $\phi(x) = k(\cdot, x)$ is a valid
feature map for $k$.  

\paragraph{Equivalent formulation of an RKHS.}

Here is an equivalent formulation of an RKHS that reveals a nice connection to
Sobolev spaces. For $x \in \cX$, denote by $\delta_x$ the corresponding
evaluation operator on $\cH$, which acts according to $\delta_x(f) = f(x)$. Then
a Hilbert space of functions $\cH$ is an RKHS if and only if all of the
evaluation operators are continuous.  

Recall that the Sobolev space $W^{m,2}(U)$, where $U \subseteq \R^d$, is a
Hilbert space, for any $m \geq 0$. Recall also that point evaluation in a
Sobolev space is continuous if and only if $2m > d$. Therefore, we learn the
following:       
\begin{quote}
\centering\it
A Sobolev space $W^{m,2}(U)$ is an RKHS if and only if $2m > d$.
\end{quote}
This is an amazing fact. A natural follow-up question: what is the kernel $k$
corresponding to the Sobolev space, for $2m > d$? Well, it's a bit
complicated. Are you ready for it? For $U = \R^d$, it's\footnote{The integral 
  here---hence the kernel---only takes real values, because $\exp(2\pi i
  (x-y)^\T u) = \cos(2\pi(x-y)^\T u) + i (2\pi i (x-y)^\T u)$ and the integral
  of the imaginary part is zero since the corresponding integrand is odd.}  
\[
k(x,y) = \int \frac{\exp(2\pi i (x-y)^\T u)}{1 + \sum_{0 < |\alpha| \leq m}
  \prod_{j=1}^d (2\pi u_j)^{2\alpha_j}} \, du.
\]
While this certainly looks complicated, it simplifies in a few important cases.  
% For $d=1$ and $m=1$, we get:
% \[
% k(x,y) = \frac{1}{2} \exp(-|x-y|).
% \]
For $d=1$ and $m=2$, we get the ``smoothing spline kernel'': 
\begin{equation}
\label{eq:ss_rkhs}
k(x,y) = \frac{\sqrt{3}}{3} \exp\bigg( -\frac{\sqrt{3}|x-y|}{2} \bigg) 
\sin\bigg( \frac{|x-y|}{2} + \frac{\pi}{6} \bigg).
\end{equation}
Meanwhile, for $d=2$ and $m=2$, we get the ``thin plate spline kernel'': 
\begin{equation}
\label{eq:tp_rkhs}
k(x,y) = \frac{1}{16 \pi} \|x-y\|_2^2 \, \log \|x-y\|_2^2.
\end{equation}

\subsection{RKHS regression}

After all that build up, this next part is going to be (surprisingly?)
simple. Let $\cH$ be an RKHS, and let $k : \cX \times \cX \to \R$ be its
kernel. We can now define an \emph{RKHS regression estimator} by solving 
\begin{equation}
\label{eq:rkhs}
\minimize_f \; \sum_{i=1}^n (y_i - f(x_i))^2 + \lambda \|f\|_\cH^2,
\end{equation}
where $\|\cdot\|_\cH$ is the norm associated with $\cH$. Note: here the input
points satisfy $x_i \in \cX$, and $\cX$ can be completely arbitrary (it doesn't
need to be a subset of $\R^d$). 

Popular choices of $k$, as we already mentioned, include the polynomial kernel,
Gaussian kernel, and exponential kernel, which gives novel nonparametric
regression estimators---novel in the sense that you haven't seem them thus far
in these lectures on nonparametric regresion. We can also choose particular
kernels, as given in \eqref{eq:ss_rkhs} and \eqref{eq:tp_rkhs}, to ``reproduce''
(no pun in intended) smoothing splines and thin plate splines. 

% RJT note: this is kind of cheating, but by putting it in quotes, I feel as if
% I'm being sufficiently vague about what the actual equivalence is. 

\subsection{Representer theorem}

One of the main points of appeal of RKHS regression is the fact that we have a
\emph{representer theorem} for the infinite-dimensional problem \eqref{eq:rkhs}.
This effectively generalizes the results stated thus far on splines. The
solution in \eqref{eq:rkhs} is unique and achieved at a function of the form   
\[
f(x) = \sum_{i=1}^n \beta_i k(x, x_i).
\]
Note that this is a linear combination of the representers of evaluation. The
proof of this result is actually not very hard. We'll walk through it on the
homework.  

\subsection{Finite-dimensional form}

From the representer result, we can rewrite \eqref{eq:rkhs} as
\begin{equation}
\label{eq:rkhs_finite_dim1}
\minimize_\beta \; \sum_{i=1}^n \bigg( y_i - \sum_{j=1}^n \beta_j k(x_i, x_j) 
\bigg)^2 + \lambda \sum_{i,j=1}^n \beta_i \beta_j \langle k(\cdot, x_i) ,
k(\cdot, x_j) \rangle_\cH.
\end{equation}
Using the reproducing property $\langle k(\cdot, x_i), k(\cdot, x_j) \rangle_\cH
= k(x_i, x_j)$, and denoting by $K \in \R^{n \times n}$ the kernel matrix with
entries $K_{ij} = k(x_i, x_j)$, problem \eqref{eq:rkhs_finite_dim1} simplifies
even further to
\begin{equation}
\label{eq:rkhs_finite_dim2}
\minimize_\beta \; \|Y - K\beta\|_2^2 + \lambda \beta^\T K \beta.
\end{equation}
This is (once again) a generalized ridge regression. The solution in
\eqref{eq:rkhs_finite_dim2} is 
\[
\hbeta = (K^\T K + \lambda K)^{-1} K^\T Y = (K + \lambda I)^{-1} Y,
\]
where we have used the fact that $K$ is symmetric and invertible. The RKHS
estimator makes predictions according to \smash{$\hf(x) = \sum_{j=1}^d \hbeta_j
  k(x, x_j)$}, and since \smash{$\hbeta$} is linear in $Y$, it is (once again) a
linear smoother. 

\bibliographystyle{plainnat}
\bibliography{../../common/ryantibs.bib}

\end{document}
