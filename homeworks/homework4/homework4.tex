\documentclass{article}

\def\ParSkip{} 
% Packages
\usepackage{amssymb,amsmath,amsthm,bbm}
\usepackage{verbatim,float,url,dsfont}
\usepackage{graphicx,subfigure,psfrag}
\usepackage{algorithm,algorithmic}
\usepackage{mathtools,enumitem}
\usepackage{multirow}
\usepackage{ragged2e}
\usepackage{xr-hyper}

\usepackage[colorlinks=true,citecolor=blue,urlcolor=blue,linkcolor=blue]{hyperref}
\usepackage[margin=1in]{geometry}
\usepackage[round]{natbib}

\usepackage[utf8]{inputenc} % allow utf-8 input
\usepackage[T1]{fontenc}    % use 8-bit T1 fonts
\usepackage{booktabs}       % professional-quality tables
\usepackage{nicefrac}         % compact symbols for 1/2, etc.
\usepackage{microtype}      % microtypography

\ifdefined\TimesFont 
\usepackage{times} % use times font
\fi

\ifdefined\ParSkip 
\usepackage{parskip} % use par skip
\fi

% Theorems and such
\newtheorem{theorem}{Theorem}
\newtheorem{lemma}{Lemma}
\newtheorem{corollary}{Corollary}
\newtheorem{proposition}{Proposition}
\theoremstyle{definition}
\newtheorem{remark}{Remark}
\newtheorem{definition}{Definition}

% Assumption
\newtheorem*{assumption*}{\assumptionnumber}
\providecommand{\assumptionnumber}{}
\makeatletter
\newenvironment{assumption}[2]{
  \renewcommand{\assumptionnumber}{Assumption #1#2}
  \begin{assumption*}
  \protected@edef\@currentlabel{#1#2}}
{\end{assumption*}}
\makeatother

% Widebar
\makeatletter
\newcommand*\rel@kern[1]{\kern#1\dimexpr\macc@kerna}
\newcommand*\widebar[1]{%
  \begingroup
  \def\mathaccent##1##2{%
    \rel@kern{0.8}%
    \overline{\rel@kern{-0.8}\macc@nucleus\rel@kern{0.2}}%
    \rel@kern{-0.2}%
  }%
  \macc@depth\@ne
  \let\math@bgroup\@empty \let\math@egroup\macc@set@skewchar
  \mathsurround\z@ \frozen@everymath{\mathgroup\macc@group\relax}%
  \macc@set@skewchar\relax
  \let\mathaccentV\macc@nested@a
  \macc@nested@a\relax111{#1}%
  \endgroup
}
\makeatother

% Min and max
\newcommand{\argmin}{\mathop{\mathrm{argmin}}}
\newcommand{\argmax}{\mathop{\mathrm{argmax}}}
\newcommand{\minimize}{\mathop{\mathrm{minimize}}}
\newcommand{\maximize}{\mathop{\mathrm{maximize}}}
\newcommand{\st}{\mathop{\mathrm{subject\,\,to}}}

% Shortcuts
\def\R{\mathbb{R}}
\def\C{\mathbb{C}}
\def\Z{\mathbb{Z}}
\def\N{\mathbb{N}}
\def\E{\mathbb{E}}
\def\P{\mathbb{P}}
\def\T{\mathsf{T}}
\def\Cov{\mathrm{Cov}}
\def\Var{\mathrm{Var}}
\def\indep{\perp\!\!\!\perp}
\def\th{^{\text{th}}}
\def\tr{\mathrm{tr}}
\def\df{\mathrm{df}}
\def\dim{\mathrm{dim}}
\def\col{\mathrm{col}}
\def\row{\mathrm{row}}
\def\nul{\mathrm{null}}
\def\rank{\mathrm{rank}}
\def\nuli{\mathrm{nullity}}
\def\spa{\mathrm{span}}
\def\sign{\mathrm{sign}}
\def\supp{\mathrm{supp}}
\def\diag{\mathrm{diag}}
\def\aff{\mathrm{aff}}
\def\conv{\mathrm{conv}}
\def\dom{\mathrm{dom}}
\def\hy{\hat{y}}
\def\hf{\hat{f}}
\def\hmu{\hat{\mu}}
\def\halpha{\hat{\alpha}}
\def\hbeta{\hat{\beta}}
\def\htheta{\hat{\theta}}
\def\cA{\mathcal{A}}
\def\cB{\mathcal{B}}
\def\cD{\mathcal{D}}
\def\cE{\mathcal{E}}
\def\cF{\mathcal{F}}
\def\cG{\mathcal{G}}
\def\cK{\mathcal{K}}
\def\cH{\mathcal{H}}
\def\cI{\mathcal{I}}
\def\cL{\mathcal{L}}
\def\cM{\mathcal{M}}
\def\cN{\mathcal{N}}
\def\cP{\mathcal{P}}
\def\cS{\mathcal{S}}
\def\cT{\mathcal{T}}
\def\cW{\mathcal{W}}
\def\cX{\mathcal{X}}
\def\cY{\mathcal{Y}}
\def\cZ{\mathcal{Z}}


\title{Homework 4 \\ \smallskip
\large Advanced Topics in Statistical Learning, Spring 2023 \\ \smallskip
Due Friday April 14 at 5pm}
\date{}

\begin{document}
\maketitle
\RaggedRight
\vspace{-50pt}

\section{Basic fact about CDFs and quantiles [20 pts]}

In this exercise, we'll walk through a number of basic but important facts about 
quantiles and cumulative distribution functions (CDFs). Let $F$ be a CDF, of the
form 
\[
F(x) = \P(X \leq x), \quad x \in \R,
\]
for some real-valued random variable $X$. Let $Q$ be the corresponding quantile 
function,
\[
Q(t) = \inf \{ x : F(x) \geq t \}, \quad t \in [0,1].
\]
(This is often denoted as $Q = F^{-1}$, even when the inverse of $F$ does not
exist in the usual sense.) We note that $F$ is always nondecreasing and 
right-continuous; the latter says, for any $x$,    
\[
F(x) = \lim_{y \to x^+} F(y)
\]
(where $y \to x^+$ means that $y$ approaches $x$ from the right).
Similarly, $Q$ is always nonincreasing and left-continuous; the latter says,
for any $t$,  
\[
Q(t) = \lim_{u \to t^-} Q(u)
\]
(where $u \to t^-$ means that $u$ approaches $t$ from the left).

\begin{enumerate}[label=(\alph*)]
\item Prove that for any $x$ and any $t$, 
  \marginpar{\small [3 pts]}
  \[
  F(x) \geq t \iff Q(t) \leq x.
  \]
  This is sometimes called the \emph{Galois inequality} for the quantile
  function. Hint: one direction follows from the definition of $Q$, and the
  other is a consequence of right-continuity of $F$.  

\item Use part (a) to prove that if $U \sim \mathrm{Unif}(0,1)$, then $Q(U)$ is
  distributed according to $F$ (meaning, it has $F$ as its CDF). 
  \marginpar{\small [2 pts]}

\item Use part (a) to prove that for any $t$, 
  \marginpar{\small [2 pts]}
  \[
  F(Q(t)) \geq t,
  \]
  with strict inequality if and only if $t$ is not in the range of $F$. 

\item Use parts (b) and (c) to prove that if $X$ is distributed according to
  $F$, then $F(X)$ is sub-uniform, which means that for any $t$, 
  \marginpar{\small [3 pts]}
  \[
  \P(F(X) \leq t) \leq t.
  \]
  Hint: you may start by replacing $X$ with $Q(U)$, for $U \sim
  \mathrm{Unif}(0,1)$, as they have the same distribution.  
  
\item Give an example to show that, in general, equality may fail in the result
  in part (d).  
  \marginpar{\small [2 pts]}

\item Show that we can always achieve equality in part (d) via auxiliary
  randomization: define 
  \[
  F^*(x; v) =  \lim_{y \to x^-} F(y) + v \cdot \Big( F(x) - \lim_{y \to x^-}
  F(y) \Big), 
  \]
  and prove that for $V \sim \mathrm{Unif}(0,1)$, independent of $X$, and for
  any $t$,  
  \marginpar{\small [3 pts]}
  \[
  \P(F^*(X; V) \leq t) = t.
  \]
  Hint: there are different ways to go about this; one way is to show that, with
  respect to the randomness in $(X,V)$ jointly, we can think of $F^*$ as a CDF 
  that is continuous at every point, and hence one for which we always have an
  inequality in part (c). 
\end{enumerate}

\section{Calibration-conditional beta coverage}

derive beta result do simulations

\begin{enumerate}[label=(\alph*)]
\item
  \marginpar{\small [2 pts]}
\end{enumerate}

\section{X-conditional coverage: impossible!}

\begin{enumerate}[label=(\alph*)]
\item
  \marginpar{\small [2 pts]}
\end{enumerate}


\section{}

\bibliographystyle{plainnat}
\bibliography{../../common/ryantibs}

\end{document}