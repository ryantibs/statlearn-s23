\documentclass{article}

\def\ParSkip{} 
% Packages
\usepackage{amssymb,amsmath,amsthm,bbm}
\usepackage{verbatim,float,url,dsfont}
\usepackage{graphicx,subfigure,psfrag}
\usepackage{algorithm,algorithmic}
\usepackage{mathtools,enumitem}
\usepackage{multirow}
\usepackage{ragged2e}
\usepackage{xr-hyper}

\usepackage[colorlinks=true,citecolor=blue,urlcolor=blue,linkcolor=blue]{hyperref}
\usepackage[margin=1in]{geometry}
\usepackage[round]{natbib}

\usepackage[utf8]{inputenc} % allow utf-8 input
\usepackage[T1]{fontenc}    % use 8-bit T1 fonts
\usepackage{booktabs}       % professional-quality tables
\usepackage{nicefrac}         % compact symbols for 1/2, etc.
\usepackage{microtype}      % microtypography

\ifdefined\TimesFont 
\usepackage{times} % use times font
\fi

\ifdefined\ParSkip 
\usepackage{parskip} % use par skip
\fi

% Theorems and such
\newtheorem{theorem}{Theorem}
\newtheorem{lemma}{Lemma}
\newtheorem{corollary}{Corollary}
\newtheorem{proposition}{Proposition}
\theoremstyle{definition}
\newtheorem{remark}{Remark}
\newtheorem{definition}{Definition}

% Assumption
\newtheorem*{assumption*}{\assumptionnumber}
\providecommand{\assumptionnumber}{}
\makeatletter
\newenvironment{assumption}[2]{
  \renewcommand{\assumptionnumber}{Assumption #1#2}
  \begin{assumption*}
  \protected@edef\@currentlabel{#1#2}}
{\end{assumption*}}
\makeatother

% Widebar
\makeatletter
\newcommand*\rel@kern[1]{\kern#1\dimexpr\macc@kerna}
\newcommand*\widebar[1]{%
  \begingroup
  \def\mathaccent##1##2{%
    \rel@kern{0.8}%
    \overline{\rel@kern{-0.8}\macc@nucleus\rel@kern{0.2}}%
    \rel@kern{-0.2}%
  }%
  \macc@depth\@ne
  \let\math@bgroup\@empty \let\math@egroup\macc@set@skewchar
  \mathsurround\z@ \frozen@everymath{\mathgroup\macc@group\relax}%
  \macc@set@skewchar\relax
  \let\mathaccentV\macc@nested@a
  \macc@nested@a\relax111{#1}%
  \endgroup
}
\makeatother

% Min and max
\newcommand{\argmin}{\mathop{\mathrm{argmin}}}
\newcommand{\argmax}{\mathop{\mathrm{argmax}}}
\newcommand{\minimize}{\mathop{\mathrm{minimize}}}
\newcommand{\maximize}{\mathop{\mathrm{maximize}}}
\newcommand{\st}{\mathop{\mathrm{subject\,\,to}}}

% Shortcuts
\def\R{\mathbb{R}}
\def\C{\mathbb{C}}
\def\Z{\mathbb{Z}}
\def\N{\mathbb{N}}
\def\E{\mathbb{E}}
\def\P{\mathbb{P}}
\def\T{\mathsf{T}}
\def\Cov{\mathrm{Cov}}
\def\Var{\mathrm{Var}}
\def\indep{\perp\!\!\!\perp}
\def\th{^{\text{th}}}
\def\tr{\mathrm{tr}}
\def\df{\mathrm{df}}
\def\dim{\mathrm{dim}}
\def\col{\mathrm{col}}
\def\row{\mathrm{row}}
\def\nul{\mathrm{null}}
\def\rank{\mathrm{rank}}
\def\nuli{\mathrm{nullity}}
\def\spa{\mathrm{span}}
\def\sign{\mathrm{sign}}
\def\supp{\mathrm{supp}}
\def\diag{\mathrm{diag}}
\def\aff{\mathrm{aff}}
\def\conv{\mathrm{conv}}
\def\dom{\mathrm{dom}}
\def\hy{\hat{y}}
\def\hf{\hat{f}}
\def\hmu{\hat{\mu}}
\def\halpha{\hat{\alpha}}
\def\hbeta{\hat{\beta}}
\def\htheta{\hat{\theta}}
\def\cA{\mathcal{A}}
\def\cB{\mathcal{B}}
\def\cD{\mathcal{D}}
\def\cE{\mathcal{E}}
\def\cF{\mathcal{F}}
\def\cG{\mathcal{G}}
\def\cK{\mathcal{K}}
\def\cH{\mathcal{H}}
\def\cI{\mathcal{I}}
\def\cL{\mathcal{L}}
\def\cM{\mathcal{M}}
\def\cN{\mathcal{N}}
\def\cP{\mathcal{P}}
\def\cS{\mathcal{S}}
\def\cT{\mathcal{T}}
\def\cW{\mathcal{W}}
\def\cX{\mathcal{X}}
\def\cY{\mathcal{Y}}
\def\cZ{\mathcal{Z}}


\title{Homework 2 \\ \smallskip
\large Advanced Topics in Statistical Learning, Spring 2023 \\ \smallskip
Due Friday March 3 at 5pm}
\date{}

\begin{document}
\maketitle
\RaggedRight
\vspace{-50pt}

\section{Properties of RKHS regression [10 points]}

In this exercise, we will work out some facts about RKHS regression. 

\begin{enumerate}[label=(\alph*)]
\item First, let $k : \cX \times \cX \to \R$ be an arbitrary kernel. Recall that
  this means there exists a feature map $\phi : \cX \to \cH$, and a Hilbert
  space $\cH$ with inner product $\langle \,\cdot\,, \,\cdot\, \rangle_\cH$,
  such that for any $x,y \in \cX$,    
  \[
  k(x,y) = \langle \phi(x), \phi(y) \rangle_\cH.
  \]
  Prove that $k$ is positive semidefinite, meaning, it is symmetric, and for any 
  $n \geq 1$ and $x_1,\dots,x_n \in \cX$, if we define a matrix $K \in \R^{n
    \times n}$ to have entries $K_{ij} = k(x_i, x_j)$, then  
  \marginpar{\small [2 pts]}
  \[
  a^\T K a \geq 0, \quad \text{for all $a \in \R^n$}.
  \]
  Hint: express $a^\T K a$ in terms of the feature map $\phi$.

\item Henceforth, suppose that $k$ is the reproducing kernel for $\cH$ (and
  hence $\cH$ is an RKHS). Recall that this means the following two properties
  are satisfied: 
  \begin{enumerate}
  \item for any $x \in \cX$, the function $k(\cdot, x)$ is an element of $\cH$;   
  \item for any function $f \in \cH$ and $x \in \cX$, it holds that $\langle f,
    k(\cdot, x) \rangle_\cH = f(x)$. 
  \end{enumerate}
  Let $f$ be a function of the form 
  \[  
  f(x) = \sum_{i=1}^n \beta_i k(x, x_i),
  \]
  for coefficients $\beta_1,\dots,\beta_n \in \R$. Show that 
  \marginpar{\small [2 pts]}
  \[
  \|f\|_\cH^2 = \sum_{i,j=1}^n \beta_i \beta_j k(x_i,x_j).
  \]

\item Let $h$ be any function (in $\cH$) that is orthogonal (with respect to
  $\langle \,\cdot\,, \,\cdot\, \rangle_\cH$) to the linear space of functions
  of the form in part (b). Prove that 
  \marginpar{\small [2 pts]}
  \[
  h(x_i) = 0, \quad i=1,\dots,n,
  \]
  and \marginpar{\small [2 pts]}
  \[
  \|h\|_\cH \geq \|f\|_\cH, \quad \text{with equality iff $h=0$}. 
  \]

\item Argue that for any $\lambda > 0$, the infinite-dimensional RKHS
  optimization problem 
  \[
  \minimize_f \; \sum_{i=1}^n (y_i - f(x_i))^2 + \lambda \|f\|_\cH^2
  \]
  (where the minimization is implicitly over $f \in \cH$) has a unique solution
  of the form in part (b), and we can rewrite it as
  \marginpar{\small [2 pts]}
  \[
  \minimize_\beta \; \|Y - K\beta\|_2^2 + \lambda \beta^\T K \beta. 
  \]
  for the matrix $K \in \R^{n \times n}$ with entries $K_{ij} = k(x_i, x_j)$. 

  Hint: let $g = f+h$ and use the results in part (c) to argue that $g$ has a
  larger criterion value, unless $h=0$. Use part (b) to complete the reduction
  to finite-dimensional form. 

\item Finally, we establish a cool fact about leave-one-out cross-validation
  (LOOCV) in RKHS regression problems. Recall that in general, the LOOCV error
  of an estimator \smash{$\hf$} is defined as
  \[
  \mathrm{CV}(\hf) = \frac{1}{n} \sum_{i=1}^n (y_i - \hf^{-i}(x_i))^2,
  \]
  where \smash{$\hf^{-i}$} is the estimator trained on all but the $i\th$ pair 
  $(x_i,y_i)$. Prove the following shortcut formula for LOOCV with an RKHS
  regression estimator \smash{$\hf$}:  
  \marginpar{\small [6 pts]}
  \[
  \frac{1}{n} \sum_{i=1}^n (y_i - \hf^{-i}(x_i))^2 = 
  \frac{1}{n} \sum_{i=1}^n \bigg(\frac{y_i - \hf(x_i)}{1-S_{ii}}\bigg)^2,
  \]
  where $S=(K + \lambda I)^{-1}$ is the smoother matrix for the RKHS estimator
  (so that the vector of fitted values is given by \smash{$\hat{Y} = SY$}). 

  Hint: prove that for each $i$,
  \[
  \hf^{-i}(x_i) = \frac{1}{1-S_{ii}} [\hf(x_i) - S_{ii}y_i].
  \]
  The desired result will follow by rearranging, squaring both sides, and
  summing over $i=1,\dots,n$. There are different ways to establish the above
  display; one nice way is as follows. Consider solving the RKHS regression
  problem on 
  \[
    (x_1,y_1), \dots, (x_{i-1},y_{i-1}), (x_{i+1},y_{i+1}), \dots, (x_n,y_n), 
  \]
  and consider solving it on 
  \[
    (x_1,y_1), \dots, (x_{i-1},y_{i-1}), (x_i, \tilde{y}_i), (x_{i+1},y_{i+1}),
    \dots, (x_n,y_n), 
  \]
  where \smash{$\tilde{y}_i = \hf^{-i}(x_i)$}. Argue that these should produce
  the same solutions. Derive the zero gradient condition for optimality
  (differentiate the criterion and set it equal to zero) for each problem, and
  use these to solve for \smash{$\tilde{y}_i = \hf^{-i}(x_i)$}. 
\end{enumerate}

\section{Sub-Gaussian maximal inequalities [15 points]}

\begin{enumerate}[label=(\alph*)]
\item Average tail bound 
\marginpar{\small [5 pts]}
\item Max tail bound in probability 
\marginpar{\small [7 pts]}
\item Max tail bound in expectation
\marginpar{\small [3 pts]}

% See ~/Dropbox/teaching/s17-702/homework/homework4.tex
% Also https://math.stackexchange.com/questions/1676407/tail-bounds-for-maximum-of-sub-gaussian-random-variables
\end{enumerate}

\section{Some lasso question}

\section{Elementary facts about ridge}

\bibliographystyle{plainnat}
\bibliography{../../common/ryantibs}

\end{document}