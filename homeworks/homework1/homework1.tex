\documentclass{article}

\def\ParSkip{} 
% Packages
\usepackage{amssymb,amsmath,amsthm,bbm}
\usepackage{verbatim,float,url,dsfont}
\usepackage{graphicx,subfigure,psfrag}
\usepackage{algorithm,algorithmic}
\usepackage{mathtools,enumitem}
\usepackage{multirow}
\usepackage{ragged2e}
\usepackage{xr-hyper}

\usepackage[colorlinks=true,citecolor=blue,urlcolor=blue,linkcolor=blue]{hyperref}
\usepackage[margin=1in]{geometry}
\usepackage[round]{natbib}

\usepackage[utf8]{inputenc} % allow utf-8 input
\usepackage[T1]{fontenc}    % use 8-bit T1 fonts
\usepackage{booktabs}       % professional-quality tables
\usepackage{nicefrac}         % compact symbols for 1/2, etc.
\usepackage{microtype}      % microtypography

\ifdefined\TimesFont 
\usepackage{times} % use times font
\fi

\ifdefined\ParSkip 
\usepackage{parskip} % use par skip
\fi

% Theorems and such
\newtheorem{theorem}{Theorem}
\newtheorem{lemma}{Lemma}
\newtheorem{corollary}{Corollary}
\newtheorem{proposition}{Proposition}
\theoremstyle{definition}
\newtheorem{remark}{Remark}
\newtheorem{definition}{Definition}

% Assumption
\newtheorem*{assumption*}{\assumptionnumber}
\providecommand{\assumptionnumber}{}
\makeatletter
\newenvironment{assumption}[2]{
  \renewcommand{\assumptionnumber}{Assumption #1#2}
  \begin{assumption*}
  \protected@edef\@currentlabel{#1#2}}
{\end{assumption*}}
\makeatother

% Widebar
\makeatletter
\newcommand*\rel@kern[1]{\kern#1\dimexpr\macc@kerna}
\newcommand*\widebar[1]{%
  \begingroup
  \def\mathaccent##1##2{%
    \rel@kern{0.8}%
    \overline{\rel@kern{-0.8}\macc@nucleus\rel@kern{0.2}}%
    \rel@kern{-0.2}%
  }%
  \macc@depth\@ne
  \let\math@bgroup\@empty \let\math@egroup\macc@set@skewchar
  \mathsurround\z@ \frozen@everymath{\mathgroup\macc@group\relax}%
  \macc@set@skewchar\relax
  \let\mathaccentV\macc@nested@a
  \macc@nested@a\relax111{#1}%
  \endgroup
}
\makeatother

% Min and max
\newcommand{\argmin}{\mathop{\mathrm{argmin}}}
\newcommand{\argmax}{\mathop{\mathrm{argmax}}}
\newcommand{\minimize}{\mathop{\mathrm{minimize}}}
\newcommand{\maximize}{\mathop{\mathrm{maximize}}}
\newcommand{\st}{\mathop{\mathrm{subject\,\,to}}}

% Shortcuts
\def\R{\mathbb{R}}
\def\C{\mathbb{C}}
\def\Z{\mathbb{Z}}
\def\N{\mathbb{N}}
\def\E{\mathbb{E}}
\def\P{\mathbb{P}}
\def\T{\mathsf{T}}
\def\Cov{\mathrm{Cov}}
\def\Var{\mathrm{Var}}
\def\indep{\perp\!\!\!\perp}
\def\th{^{\text{th}}}
\def\tr{\mathrm{tr}}
\def\df{\mathrm{df}}
\def\dim{\mathrm{dim}}
\def\col{\mathrm{col}}
\def\row{\mathrm{row}}
\def\nul{\mathrm{null}}
\def\rank{\mathrm{rank}}
\def\nuli{\mathrm{nullity}}
\def\spa{\mathrm{span}}
\def\sign{\mathrm{sign}}
\def\supp{\mathrm{supp}}
\def\diag{\mathrm{diag}}
\def\aff{\mathrm{aff}}
\def\conv{\mathrm{conv}}
\def\dom{\mathrm{dom}}
\def\hy{\hat{y}}
\def\hf{\hat{f}}
\def\hmu{\hat{\mu}}
\def\halpha{\hat{\alpha}}
\def\hbeta{\hat{\beta}}
\def\htheta{\hat{\theta}}
\def\cA{\mathcal{A}}
\def\cB{\mathcal{B}}
\def\cD{\mathcal{D}}
\def\cE{\mathcal{E}}
\def\cF{\mathcal{F}}
\def\cG{\mathcal{G}}
\def\cK{\mathcal{K}}
\def\cH{\mathcal{H}}
\def\cI{\mathcal{I}}
\def\cL{\mathcal{L}}
\def\cM{\mathcal{M}}
\def\cN{\mathcal{N}}
\def\cP{\mathcal{P}}
\def\cS{\mathcal{S}}
\def\cT{\mathcal{T}}
\def\cW{\mathcal{W}}
\def\cX{\mathcal{X}}
\def\cY{\mathcal{Y}}
\def\cZ{\mathcal{Z}}


\title{Homework 1 \\ \smallskip
\large Advanced Topics in Statistical Learning, Spring 2023 \\ \smallskip
Due Friday February 10 at 5pm}
\date{}

\begin{document}
\maketitle
\RaggedRight
\vspace{-50pt}

\section{Mathematical statistics warm-up [15 points]}

\def\pto{\overset{p}{\to}}
\def\dto{\overset{d}{\to}}

\begin{enumerate}[label=(\alph*)]
\item Suppose that $X_n \geq 0$ and $\E[X_n]=O(r_n)$. Prove that $X_n=O_p(r_n)$.
  \marginpar{\small [1 pt]} 

\item Suppose that $X_n \geq 0$ and $X_n=O_p(r_n)$.  Give an example to show
  that in general, this does not  imply that $\E[X_n]=O(r_n)$.   
  \marginpar{\small [1 pt]} 

\item Prove that for $X \geq 0$, it holds that 
  \[
  \E[X] = \int_0^\infty \P(X>t) \, dt.
  \]  
  You may assume that $X$ is continuously distributed and hence has a probability
  density function. 
  \marginpar{\small [2 pts]} 

\item Suppose that $X_n \geq 0$ and $X_n=O_p(r_n)$, the latter bound holding
  ``exponentially fast'', meaning that there are constants $\gamma_0,n_0>0$ such 
  that for all $\gamma \geq \gamma_0$ and $n \geq n_0$, we have    
  \[
  X_n \leq \gamma r_n, \quad \text{with probability at least $1-\exp(-\gamma)$}.    
  \]
  Prove that $\E[X_n]=O(r_n)$. Hint: use the formulation for $\E[X_n]$ from the
  last question. 
  \marginpar{\small [3 pts]} 

\item Let $X_1,\ldots, X_n \sim P$, i.i.d., with $\mu = \E[X_i]$ and $\sigma^2 = 
  \Var[X_i]$. Define   
  \[
  \bar{X}_n = \frac{1}{n}\sum_{i=1}^n X_i, \quad
  s_n^2 = \frac{1}{n}\sum_{i=1}^n (X_i - \bar{X}_n)^2.
  \]

(i) Prove that $s_n^2 \pto \sigma^2$.
\marginpar{\small [1 pt]} 

(ii) Prove that \smash{$\sqrt{n}(\bar{X}_n-\mu)/s_n \dto N(0,1)$}.
\marginpar{\small [1 pt]} 

\item Let $X  \in \R^d$ and $Y \in \R$.

(i) Prove that $\E[(Y - f(X))^2]$ is minimized at $f(x) = \E[Y|X=x]$.
\marginpar{\small [1 pt]} 

(ii) Prove that $\E[(Y - X^T \beta)^2]$ is minimized at $\beta = \Sigma^{-1} 
\alpha$, where $\Sigma = \E[X X^T]$ and $\alpha = \E[Y X]$.
\marginpar{\small [1 pt]} 

\item This part will involve a small bit of coding. Attach your code in an
  appendix. 

(i) Simulate Brownian motion on $[0,1]$, and a Brownian bridge on $[0,1]$, and
plot them. 
\marginpar{\small [1 pt]} 

(ii) Simulate the 95th percentile of the supremum of the Brownian bridge, i.e.,
the value $q$ such that   
\[
\P\Big( \sup_{t \in [0,1]} B(t)\geq q \Big) = 0.05.
\]
where $B(t)$, $t \in[0,1]$ is the Brownian bridge. 
\marginpar{\small [1 pt]} 

(iii) Draw $X_1,\dots,X_n \sim F$ from any distribution $F$ of your liking
(uniform, normal, etc.), calculate the Kolmogorov-Smirnov (KS) test statistic    
  \[
  T = \sqrt{n} \sup_x \, |F_n(x) - F(x)|,
  \]
  where $F_n$ is the empirical distribution of $X_1,\dots,X_n$, and calculate
  the proportion of times out of (say) 1000 repetitions that $T$ exceeds the 
  threshold $q$ computed in part (ii).  
\marginpar{\small [2 pts]} 
\end{enumerate}

\section{Risk analysis for least squares [15 points]}

\def\Bias{\mathrm{Bias}}
\def\Risk{\mathrm{Risk}}

In this exercise, we will work on risk calculations for least squares regression.

\begin{enumerate}[label=(\alph*)]
\item First, we start with an algebraic fact. Suppose that $A,B \succeq 0$, 
  which we write to mean that are positive semidefinite matrices (symmetric with
  nonnegative eigenvalues). Prove that $\tr(AB) \geq 0$. 
  \marginpar{\small [2 pts]} 

  Hint: there are many ways to prove this, but for one, take an
  eigendecomposition of $B$, and expand the trace as a sum of products involving
  its eigenvectors. 

\item For this part and the next, suppose that we observe i.i.d.\ $(x_i,y_i) \in
  \R^d \times \R$. We write $f(x) = \E[y_i|x_i = x]$, $\epsilon_i = y_i -
  f(x_i)$, and assume that each $x_i \indep \epsilon_i$. We denote $\sigma^2 = 
  \Var[\epsilon_i]$. 

  Let $Y \in \R^n$ be the response vector and $X \in \R^{n \times d}$ the
  predictor matrix (whose $i\th$ row is $x_i$). Let \smash{$\hbeta = (X^\T
    X)^{-1} X^\T Y$} be the least squares solution of $Y$ on $X$ (where we 
  assume $X^T X$ is invertible, which requires $d \leq n$), and let
  \smash{$\hf(x) = x^\T \hbeta$}.  

  Follow/reproduce the calculations in the review lecture to show that
  \marginpar{\small [3 pts]} 
  \[
  \frac{1}{n} \sum_{i=1}^n \Var(\hf(x_i)) = \sigma^2 \frac{d}{n},
  \]
  and that, for an independent draw $x_0$ from the predictor distribution,
  \marginpar{\small [3 pts]} 
  \[
  \E[\Var(\hf(x_0) \, | X, x_0)] = \frac{\sigma^2}{n} \tr\Big(\E [X^\T X] \, \E
  [(X^\T X)^{-1}] \Big). 
  \]
  Therefore, using part (a), argue that
  \marginpar{\small [1 pt]}
  \[
  \E[\Var(\hf(x_0) \, | X, x_0)] \geq \frac{1}{n} \sum_{i=1}^n \Var(\hf(x_i)).
  \]

  Hint: the calculations in lecture assumed the underlying model was linear and
  hence the bias (both in- and out-of-sample) was zero. But if you look back 
  carefully, the variance calculations are unaffected by whether the true mean
  is linear or not.    

\item Follow/reproduce the calculations leading up to Theorem 1 in
  \citet{rosset2020from} to prove the inequality: 
  \marginpar{\small [6 pts]} 
  \[
  \E[\Bias^2(\hf(x_0) \, | X, x_0)] \geq  \frac{1}{n} \sum_{\ell=1}^n 
  \E[\Bias^2(\hf(x_i) \,|\, X)].
  \]
  Note that you have shown that
  \[
  \Risk(\hf) \geq \E[\Risk(\hf; x_{1:n})].
  \]
  In other words, the out-of-sample risk of least squares is always at least as 
  large as the in-sample risk (integrated over the feature values). To
  emphasize, this assumes nothing really at all (i.e., no underlying linear
  model) about the data model, except for the independence of $x_i$ and
  $\epsilon_i$.   

\item As a bonus, prove or disprove: there is a predictor distribution such that
  we get an equality in the last display, i.e., the out-of-sample and in-sample
  risks are equal. Note that we are still talking about standard least squares
  regression, so we are restricting attention to distributions such that $X^\T
  X$ is almost surely invertible.  
\end{enumerate}

\section{Risk analysis for wavelet denoising [22 points]}

\def\ttheta{\tilde\theta}
\def\TV{\mathrm{TV}}

In this exercise, we will analyze the risk of wavelet denoising. 

\begin{enumerate}[label=(\alph*)]
\item Assume for now that we observe data according to the normal sequence model  
  \begin{equation}
  \label{eq:sequence_model}
  z_\ell = \theta_\ell + \delta_\ell, \quad \ell=1,\dots,N,  
  \end{equation}
  where $\delta_\ell \sim N(0,\tau^2)$, independently, for
  $\ell=1,\dots,N$. Consider the soft-thresholding estimator,
  \[
  \htheta_\ell = S_\lambda(z_\ell) 
  = \begin{cases}
    z_\ell-\lambda & z_\ell > \lambda \\
    0 & |z_\ell| \leq \lambda \\
    z_\ell+\lambda & z_\ell < -\lambda
  \end{cases}, 
  \quad \ell=1,\dots,N.
  \]
  Here $\lambda \geq 0$ is a tuning parameter. For arbitrary $\lambda$, prove
  that we have the exact risk expression:
  \marginpar{\small [3 pts]}
  \[
  \E \|\theta - \htheta\|_2^2 = \sum_{\ell=1}^N r(\theta_\ell, \lambda),   
  \]
  where
  \[
    r(\mu,\lambda) = \mu^2
    \int_{\frac{-\lambda-\mu}{\tau}}^{\frac{\lambda-\mu}{\tau}} \phi(z) \, dz +  
    \int_{\frac{\lambda-\mu}{\tau}}^\infty (\tau z- \lambda)^2 \phi(z) \, dz + 
    \int_{-\infty}^{\frac{-\lambda-\mu}{\tau}} (\tau z + \lambda)^2 \phi(z) \,
    dz, 
  \]
    and $\phi$ denotes the standard (univariate) normal density function. 

  \item Prove that for \smash{$\lambda = \tau\sqrt{2\log{N}}$}, we have the risk
    upper bound: 
  \marginpar{\small [5 pts]}
  \[
    \E \|\theta - \htheta\|_2^2 \leq \tau^2 + (2\log{N}+1) \sum_{\ell=1}^N
    \min\{\theta_\ell^2, \tau^2\}.   
  \]

  Hint: start with $\tau^2=1$ for simplicity. Prove that, for any $\mu,\lambda
  \geq 0$, we have $0 \leq \partial r(\mu,\lambda) / \partial \mu \leq
  2\mu$. From this, argue that $r(\mu,\lambda)$ is monotone increasing in $\mu$,
  and further    
  \[
  r(\mu,\lambda) \leq r(0,\lambda) + \min\{\mu^2, r(\infty,\lambda)\}.  
  \]
  Then, derive upper bounds on $r(0,\lambda)$ and $r(\infty,\lambda)$ (for the
  former you can use Mills' ratio, for the latter you can use direct arguments)
  to give     
  \[
  r(\mu,\lambda) \leq e^{-\lambda^2/2} + \min\{\mu^2,1+\lambda^2\}.  
  \]
  Plug in \smash{$\lambda = \sqrt{2\log{N}}$}; show that an analogous bound
  holds for general $\tau^2>0$; and sum the bound over $\mu = \theta_\ell$,
  $\ell=1,\ldots,N$ to give the result.    

\item Now consider the nonparametric regression model 
  \begin{equation}
  \label{eq:nonpar_model}
  y_i = f(x_i) + \epsilon_i, \quad i=1,\dots,n,
  \end{equation}
  where $\epsilon_i \sim N(0,\sigma^2)$, independently, for $i=1,\dots,n$, and
  $x_i \in [0,1]$, $i=1,\dots,n$ are fixed (more on them later). We are going to
  analyze the $L^2([0,1])$ risk of a wavelet smoothing estimator \smash{$\hf$},      
  \[
  \E \|f - \hf\|_{L^2([0,1])}^2 = \E \bigg[ \int_0^1 (f(x) -  \hf(x))^2 \, dx \bigg]. 
  \]
  The estimator \smash{$\hf$} will be defined by 
  \begin{equation}
  \label{eq:wavelet_smoothing}
  \hf(x) = \sum_{j,k} \ttheta_{jk}(y) \psi_{jk}, 
  \end{equation}
  where each $\psi_{jk}$ is a Haar wavelet function, and each
  \smash{$\ttheta_{jk}(y)$} is a noisy empirical wavelet coefficient.

  We begin with a simple Haar calculation. To recall the Haar basis on $[0,1]$, 
  first define $\psi(x) = 1\{x \in [0,1/2]\} - 1\{x \in (1/2,1]\}$. Then the
  Haar basis is given by the collection          
  \[
   1, \; \psi_{jk}, \; \text{for $k = 0, \dots, 2^j-1$ and $j = 0,1,2,\dots$},     
  \]
  where $\psi_{jk}(x) = 2^{j/2} \psi(2^j x - k)$. (For notational convenience,
  we let $\psi_{-10} = 1$, and implicitly index all basis calculations starting 
  from $j=-1$.) Verify that this collection is orthonormal in $L^2([0,1])$: the
  functions are pairwise orthogonal and unit norm, with respect to the
  $L^2([0,1])$ inner product    
  \marginpar{\small [2 pts]}
  \[
  \langle f, g \rangle = \int_0^1 f(x) g(x) \, dx.
  \]
  (Accordingly the $L^2([0,1])$ norm is given by \smash{$\|f\|_{L^2([0,1])}^2
    = \langle f, f \rangle = \int_0^1 f(x)^2 \, dx$}.) 

\item Explain why it is that we can write
  \marginpar{\small [1 pt]}
  \[
  \E \|f - \hf\|_{L^2([0,1])}^2 = \sum_{j,k} (\theta_{jk}(f) - \ttheta_{jk}(y))^2,
  \]
  where the wavelet coefficients of $f$ are 
  \[
  \theta_{jk}(f) = \langle f, \psi_{jk} \rangle = \int_0^1 f(x) \psi_{jk}(x) \,
  dx,
  \]
  and \smash{$\ttheta_{jk}(y)$} are the coefficients to define the estimator
  \smash{$\hf$} in its Haar basis expansion \eqref{eq:wavelet_smoothing}.    

  Hint: by orthonormality, observe that \smash{$f = \sum_{j,k} \theta_{jk}(f) 
    \psi_{jk}$}. It suffices to just name the theorem that relates the
  $L^2([0,1])$ of a function to the norm of its coefficients.  

\item We define the last few parts needed to understand \smash{$\hf$} and 
  analyze its risk. For each $j,k$, we define the empirical wavelet coefficient 
  \[
  \ttheta_{jk}(f) = \frac{1}{n} \sum_{i=1}^n f(x_i) \psi_{jk}(x_i).
  \] 
  We also define a noisy empirical wavelet coefficient 
  \[
  \ttheta_{jk}(y) = \begin{cases}
  \displaystyle
  S_\lambda \bigg( \frac{1}{n} \sum_{i=1}^n y_i \psi_{jk}(x_i) 
  \bigg) & j \leq j^* \\
  0 & j > j ^*
  \end{cases},
  \]
  where $S_\lambda$ is the soft-thresholding operator, as before, and $j^*$ is a
  truncation level, to be chosen. 

  By part (d), and the inequality $(a+b)^2 \leq 2a^2 + 2b^2$ (applied twice), we
  have 
  \[
  \E \|f - \hf\|_{L^2([0,1])}^2 \leq 
  2\underbrace{\sum_{j > j^*,k} \theta_{jk}^2(f)}_{e_1} 
  \,+\,
  4\underbrace{\sum_{j \leq j^*,k} (\theta_{jk}(f) - \ttheta_{jk}(f))^2}_{e_2}
  \,+\, 
  4\underbrace{\sum_{j \leq j^*,k} (\ttheta_{jk}(f) - \ttheta_{jk}(y))^2}_{e_3}.
  \]
  We can interpret $e_1$ as the \emph{truncation error}, $e_2$ as the
  \emph{discretization error} (between population and empirical wavelet
  coefficients), and $e_3$ as the \emph{estimation error} (in estimating the
  empirical wavelet coefficients from noisy data). 
  
  Denote by $\theta_{j \cdot}(f)$ the vector $(\theta_{jk}(f) : k =
  0,\dots,2^j-1)$. Assume that $\TV(f) \leq 1$, and assume that the design
  points $x_i = i/n$, $i=1,\dots,n$ are evenly-spaced. It can be shown that 
  \begin{equation}
  \label{eq:besov_condition}
  \| \theta_{j \cdot}(f) \|_1 \leq c_1 2^{-j/2}, \quad
  \| \ttheta_{j \cdot}(f) \|_1 \leq c_2 2^{-j/2}, \quad \text{and} \quad 
  \| \theta_{j \cdot}(f) - \ttheta_{j \cdot}(f) \|_1 \leq c_3 \frac{2^{j/2}}{n}, 
  \end{equation}
  for constants $c_1,c_2,c_3> 0$. Use the first and third inequalities to show 
  that there is a truncation level $j^*$ such that sum of truncation and
  discretization errors satisfy $e_1 + e_2 \leq C/n$, for another constant
  $C>0$.       
  \marginpar{\small [2 pts]}

\item It remains to study the estimation error. Assume that $n$ is a power of
  2. Show that, starting from the nonparametric regression model
  \eqref{eq:nonpar_model}, we may transform this to a model of the form   
  \marginpar{\small [3 pts]}
  \[
  z_\ell = \ttheta_\ell(f) + \delta_\ell, \quad \ell=1,\dots,n,
  \]
  where $\delta_\ell \sim N(0,\sigma^2/n)$, independently, for
  $\ell=1,\dots,n$. Note that here, in indexing wavelet coefficients, we
  collapse the pair $j,k$ into a single index $\ell$. 

  Hint: use the appropriate truncation level $j^*$, from part (e), and only
  consider $j \leq j^*$. Then define a matrix $\Psi$ with elements $[\Psi]_{i
    \ell} = \psi_\ell(x_i)/n$, where in indexing the Haar wavelets, we again
  collapse the pair $j,k$ into a single index $\ell$. Using the fact we have an
  evenly-spaced design $x_i = i/n$, $i=1,\dots,n$, show that \smash{$\Psi
    \Psi^\T = \frac{1}{n} I$}, where $I$ is the $n \times n$ identity matrix.        

\item Finally, note that from the transformation in part (f) you have brought
  yourself back to the problem studied in parts (a), (b): soft-thresholding
  under the sequence model \eqref{eq:sequence_model}, with noise level $\tau^2 =
  \sigma^2/n$. 

  From the risk bound from part (b), note that we have
  \[
  \sum_{j \leq j^*,k} (\ttheta_{jk}(f) - \ttheta_{jk}(y))^2 \leq 
  \frac{\sigma^2}{n} + (2 \log{n} + 1) \sum_{j \leq j^*,k} \min 
  \bigg \{ \ttheta_{jk}^2(f), \frac{\sigma^2}{n} \bigg\}.
  \]
  Use the second inequality in \eqref{eq:besov_condition}, on the empirical
  wavelet coefficients, to establish that for each $j$, 
  \marginpar{\small [4 pts]}
  \[
  \sum_k \min \bigg\{ \ttheta_{jk}^2(f), \frac{\sigma^2}{n} \bigg\} \leq
  C \frac{\sigma^2}{n} 2^j \min\bigg\{ 1, 2^{-3j/2} \frac{\sqrt{n}}{\sigma}
  \bigg\},  
  \]
  for a constant $C>0$. Show that gives the estimation error bound, 
  \marginpar{\small [2 pts]}
  \[
  e_3 \leq C \log{n} \bigg( \frac{\sigma^2}{n} \bigg)^{2/3}.
  \]
  for a constant $C>0$, redefined as needed.

  Hint: the first bound (second-to-last display) is a bit tricky, whereas the
  second (last display) is more of a straight algebraic calculation, summing the
  first bound over $j$. To prove the first, argue that 
  \[
  \sup_{\|\ttheta_{j \cdot}\|_1 \leq c_j} \; \sum_k \min \bigg\{
  \ttheta_{jk}^2, \frac{\sigma^2}{n} \bigg\}   
  \]
  will be achieved at a vector \smash{$\ttheta_{j \cdot}$} for which each entry
  is equal to $0$ or \smash{$\sigma/\sqrt{n}$}, except for (possibly) one
  entry, which is defined so that we hit the constraint \smash{$\|\ttheta_{j
      \cdot}\|_1 =  c_j$}. For the current problem, note that we have $c_j = c_2
  2^{-j/2}$.   

  Concluding note: the risk bound you have shown, redefining the constant $C>0$
  as needed, is  
  \[
  \E \|f - \hf\|_{L^2([0,1])}^2\leq C\bigg[ \frac{1}{n} + \log{n} \bigg(
  \frac{\sigma^2}{n} \bigg)^{2/3} \bigg],   
  \]
  for estimating a function with $\TV(f) \leq 1$ using Haar wavelet
  denoising. This is minimax rate optimal for the class of functions with
  bounded TV, ignoring log factors (which could be removed from the upper bound
  with a slightly finer analysis).
\end{enumerate}

\section{Minimax lower bounds for H{\"o}lder classes [8 points]} 

Generalize the univariate calculation given in the minimax lecture to the
multivariate case, to show that for the $d$-dimensional H{\"o}lder ball 
\[
C^k(L; [0,1]^d) = \Big\{ f : [0,1]^d \to \R \,:\, | D^\alpha f (x) - D^\alpha
f(y) | \leq L \|x-y\|_2 \; \text{for all $\|\alpha\|_1 < k$, and $x,y \in
  [0,1]^d$} \Big\},
\]
where $L>0$ is a constant (not growing with $n$), the following holds. Given
data 
\[
y_i = f(x_i) + \epsilon_i, \quad i=1,\dots,n,
\]
where $\epsilon_i \sim N(0,1)$, independently, for $i=1,\dots,n$, and $x_i \in
[0,1]^d$, $i=1,\dots,n$ are fixed, there exists constants $C,N>0$ such that
for all $n \geq N$,   
\[
\inf_{\hf} \, \sup_{f \in C^k_d(L; [0,1]^d)} \, \E \big \|f - \hf
\big\|_{L^2([0,1]^d)}^2 \geq C n^{-\frac{2k}{2k+d}}. 
\]
Make sure to write the full calculation from start to finish (even detailing the 
steps that may be analogous to those in lecture). 
\marginpar{\small [7 pts]}

Explain why this certifies that the upper bound given in the last question for
estimation over a univariate TV ball is rate optimal (ignoring log factors). 
\marginpar{\small [1 pt]}

\bibliographystyle{plainnat}
\bibliography{../../common/ryantibs}

\end{document}